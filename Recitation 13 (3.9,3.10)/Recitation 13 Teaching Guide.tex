\documentclass[handout,nooutcomes]{ximera}
%% handout
%% space
%% newpage
%% numbers
%% nooutcomes


\newcommand{\RR}{\mathbb R}
\renewcommand{\d}{\,d}
\newcommand{\dd}[2][]{\frac{d #1}{d #2}}
\renewcommand{\l}{\ell}
\newcommand{\ddx}{\frac{d}{dx}}
\newcommand{\dfn}{\textbf}
\newcommand{\eval}[1]{\bigg[ #1 \bigg]}

\usepackage{multicol}

\renewenvironment{freeResponse}{
\ifhandout\setbox0\vbox\bgroup\else
\begin{trivlist}\item[\hskip \labelsep\bfseries Solution:\hspace{2ex}]
\fi}
{\ifhandout\egroup\else
\end{trivlist}
\fi} %% we can turn off input when making a master document

\title{Recitation \#13 - 3.9 Derivatives of Logarithmic and Exponential Functions and 3.10 Derivatives of Inverse Trig Functions (Teaching Guide)}  

\begin{document}
\begin{abstract}		\end{abstract}
\maketitle


\section*{Warm up:} 
	
	\begin{itemize}
	
	\item  \emph{5 minutes}:  Ask students to think about the warm-up as they are waiting for class to begin.  Then discuss the warm-up as a class when class begins.
	
	
	
	\end{itemize}


\section*{Problem 1:}

	\begin{itemize}
	
	\item  \emph{10 minutes}:  Allow students to work on \#1 in groups. Have different groups do parts a,b, and c  If they finish early, they can work on the other parts.  While students are still working, have students start writing their solutions on the board to present.
	
	\item  \emph{5 minutes}:  Let a group present their solution(s).
	
	\end{itemize}




\section*{Problem 2:}

	\begin{itemize}
	
	\item  \emph{5 minutes}:  Divide up the problems in \#2 among the groups and let the students work in groups.  Have them work on the other parts if they finish their part early. 
		
	\item  \emph{5 minutes}:  Let groups present their solution(s).
			
	\end{itemize}
	
	
	
\section*{Problem 3:}

	\begin{itemize}
	
	\item  \emph{10 minutes}:  Allow students to work on \#3 in groups.  If is very likely that you are going to have to remind them of the formula for $(f^{-1})'(x)$. 
	
	\item  \emph{5 minutes}:  Allow a group to present their solutions to \#3.
	
	\end{itemize}
	


\section*{Problem 4:}

	\begin{itemize}
	
	\item  \emph{5 minutes}:  Allow students to work on \#4 in groups.  
	
	\item  \emph{5 minutes}:  Allow a group to present their solution to \#4, or discuss as a class if you are low on time.
		
	\end{itemize}	
	
	
	

	
	

	
	
	

	
	
	
















\end{document}