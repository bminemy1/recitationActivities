\documentclass[handout,nooutcomes]{ximera}
\usepackage{booktabs}
%% handout
%% space
%% newpage
%% numbers
%% nooutcomes

\renewcommand{\outcome}[1]{\marginpar{\null\vspace{2ex}\scriptsize\framebox{\parbox{0.75in}{\begin{raggedright}P\arabic{problem} Outcome: #1\end{raggedright}}}}}

\renewenvironment{freeResponse}{
\ifhandout\setbox0\vbox\bgroup\else
\begin{trivlist}\item[\hskip \labelsep\bfseries Solution:\hspace{2ex}]
\fi}
{\ifhandout\egroup\else
\end{trivlist}
\fi}

\newcommand{\RR}{\mathbb R}
\renewcommand{\d}{\,d}
\newcommand{\dd}[2][]{\frac{d #1}{d #2}}
\renewcommand{\l}{\ell}
\newcommand{\ddx}{\frac{d}{dx}}
\everymath{\displaystyle}
\newcommand{\dfn}{\textbf}
\newcommand{\eval}[1]{\bigg[ #1 \bigg]}


\title{Breakout Session 22 Teaching Guide}

\begin{document}
\begin{abstract}

\end{abstract}
\maketitle

\section{Notes for Problem 1}
For this problem, you should probably remind students about the definitions of Riemann sum, left Riemann sum, right Riemann sum, and midpoint Riemann sum.
(It may also be helpful to draw a graph of a decreasing function that is positive.)

When working the solution I suggest you draw two graphs: a decreasing concave down function and a decreasing concave up function.
(Or, you can also spin this into another problem and ask ``does the right Riemann sum of a concave up positive continuous function overestimates, underestimates, or there is not enough information to tell the area under the graph?'')

\section{Notes for Problem 2}
I would skip part (a) and work directly on part (b).
This problem gives students practice in reshaping sigma notation and is part of the usual first step when computing integrals via limits.

\section{Notes for Problem 3}
Please make sure you and the students work this problem completely!
Remind the students about the important summation formulas.

This problem will take an extremely long time to solve: students will have difficulties with the notation, algebra, and they may forget how the limits are evaluated.
Therefore it’s important to go through part (b) slowly and clearly.

\section{Notes for Problem 4}
This problem is from the AU15 Final Exam and simply tests if students can identify the corresponding components in the limit of (general) Riemann sums and the definite integrals.
(You can mention that they will learn how to (easily) evaluate such integrals in Wednesday's lecture.)

\section{Notes for Problem 5}
Before working this problem remind students about the basic properties of integrals (with the appropriate graphs!).
Part (c) will give students trouble (since they may assume that absolute ``composed'' with a definite integral ``commutes'').
\end{document} 
