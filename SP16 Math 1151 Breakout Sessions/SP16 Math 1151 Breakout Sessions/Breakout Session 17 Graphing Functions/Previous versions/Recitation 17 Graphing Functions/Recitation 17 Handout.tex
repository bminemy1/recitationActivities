\documentclass[handout, nooutcomes]{ximera}
%% handout
%% space
%% newpage
%% numbers
%% nooutcomes

%I added the commands here so that I would't have to keep looking them up
\newcommand{\RR}{\mathbb R}
\renewcommand{\d}{\,d}
\newcommand{\dd}[2][]{\frac{d #1}{d #2}}
\renewcommand{\l}{\ell}
\newcommand{\ddx}{\frac{d}{dx}}
\everymath{\displaystyle}
\newcommand{\dfn}{\textbf}
\newcommand{\eval}[1]{\bigg[ #1 \bigg]}

%\begin{image}
%\includegraphics[trim= 170 420 250 180]{Figure1.pdf}
%\end{image}

\renewenvironment{freeResponse}{
\ifhandout\setbox0\vbox\bgroup\else
\begin{trivlist}\item[\hskip \labelsep\bfseries Solution:\hspace{2ex}]
\fi}
{\ifhandout\egroup\else
\end{trivlist}
\fi}


% \newcommand{\RR}{\mathbb R}
\renewcommand{\d}{\,d}
\newcommand{\dd}[2][]{\frac{d #1}{d #2}}
\renewcommand{\l}{\ell}
\newcommand{\ddx}{\frac{d}{dx}}
\newcommand{\dfn}{\textbf}
\newcommand{\eval}[1]{\bigg[ #1 \bigg]}

\renewenvironment{freeResponse}{
\ifhandout\setbox0\vbox\bgroup\else
\begin{trivlist}\item[\hskip \labelsep\bfseries Solution:\hspace{2ex}]
\fi}
{\ifhandout\egroup\else
\end{trivlist}
\fi} %% we can turn off input when making a master document

\title{Recitation 17}  
\begin{document}
\begin{abstract}		\end{abstract}
\maketitle

\section{Graphing Functions}

\begin{problem}
  \mbox{}
  \begin{enumerate}
	
    % part 1
  \item[1.]  You are given that $f''(x) > 0$ for all $x$.  Which of
    the following must be true about $f(x)$ on the region
    $0 \leq x \leq 2$?
    \begin{enumerate}
		
    \item There is a critical point between $0$ and $2$.
    \item An absolute maximum occurs at either $x=0$ or $x=2$.
    \item There is a local maximum, but not enough information is
      given to determine where.
    \item $f$ need not have a local maximum.
		
    \end{enumerate}

			\begin{freeResponse}
                          Only (b) and (d) must be true.  The
                          following picture provides a counterexample
                          for both (a) and (c).
			
                          \begin{image}
                            \includegraphics[trim= 70 470 250
                            190]{Figure1.pdf}
                          \end{image}
						
			\end{freeResponse}
		
		
		
                        % part 2
                      \item[2.]  You are told that $f''(x) > 0$ for
                        all $x$.  Which of the following must be true
                        about the graph of $y=f(x)$?
                        \begin{enumerate}
		
                        \item The graph is a straight line.
                        \item The graph crosses the $x$-axis at most
                          once.
                        \item The graph is concave down.
                        \item The graph crosses the $y$-axis more than
                          once.
                        \item The graph is concave up.
		
                        \end{enumerate}

			\begin{freeResponse}
                          Only (e) must be true.  For (a), the
                          function $f(x) = e^x$ provides a
                          counterexample.
			
                          \begin{image}
                            \includegraphics[trim= 70 470 250
                            190]{Figure2.pdf}
                          \end{image}
			
                          For part (b), the function $f(x) = x^2 -2$
                          provides a counterexample.
			
                          \begin{image}
                            \includegraphics[trim= 70 470 250
                            190]{Figure3.pdf}
                          \end{image}
			
                          Part (c) is clearly false since $f''(x) > 0$
                          means that $f$ is concave up.  Part (d) is
                          false for any function.
			
			\end{freeResponse}
		
		
		
                      \end{enumerate}
                    \end{problem}
		
		

	
\newpage

%problem 1
\begin{problem}
Given that:
$$ \lim_{x \to - \infty} f(x) = 0 \, , \, \lim_{x \to -3^-} f(x) = \infty \, , \, \lim_{x \to -3^+} f(x) = - \infty \, , \, \lim_{x \to 4^-} f(x) = \infty $$
$$ \lim_{x \to 4^+} f(x) = -\infty \, , \, f(1) = 1 \, , \, f(5) = -2 \, , \, f(-3) \text{ is undefined} \, , \, f(4) \text{ is undefined} $$
$$ f(9) \text{ is undefined} \, , \, f \text{ is continuous except at } x=-3, 4, \text{ and } 9 $$
$$ f'(1) \ne 0\, ,f'(7) = 0 \, , \, f'(x) = 2 \text{ for } x > 9 \, , \, f''(1) = 0 $$
and the following sign chart for the first and second derivatives of $f$:

	\begin{image}
	\includegraphics[trim= 70 530 250 190]{Figure4.pdf}
	\end{image}
	
find the following:
	\begin{enumerate}
	
	\item  Critical points.
	\item  Intervals where $f$ is increasing and decreasing.
	\item  Local extrema.
	\item  Inflection points.
	\item  Intervals of concavity.
	\item  Sketch the graph of $f$.
	
	\end{enumerate}
	
		\begin{freeResponse}
		
			\begin{enumerate}
			
			%part a
			\item  Critical points. \\
			The critical points of $f$ occur at points in the domain of $f$ where either $f'(x)=0$ or where $f'(x)$ does not exist.  We are given that $f'(7)=0$, and so $x=7$ is a critical point of $f$.  Even though $f'(-3), f'(4),$ and $f'(9)$ do not exist, all three of those points are not in the domain of $f$.  Therefore, $x=7$ is the only critical point of $f$.  
			
			%part b
			\item  Intervals where $f$ is increasing and decreasing.  \\
			$f$ is increasing when $f'(x)>0$.  From the sign chart and our critical points, these are the intervals $(-\infty ,-3)$, $(-3,4)$, $(4,7]$, and $(9,\infty )$. 
$f(x)$ is decreasing when $f'(x)<0$.  From the sign chart, this is on the interval $[7,9)$.
			
			%part c
			\item  Local extrema.  \\
			Using the first derivative test, $f(7)$ is a local maximum because the derivative changes sign from positive to negative.  Although the derivative changes sign from negative to positive at $f(9)$, this is not a local minimum because the function is not defined at this point.  Therefore, $x=7$ is the only local extremum of $f$, and it is a local maximum.
			
			%part d
			\item  Inflection points.  \\
			Possible inflection points occur where $f''(x)=0$ or where $f''(x)$ does not exist.  We are given that $f''(1)=0$.  In addition, $f''(x)$ does not exist at $x=-3,4,9$ .  However, these are not inflection points because $f$ is not defined at these points.  Since $f''$ changes sign at $x=1$, this is in fact an inflection point of $f$.  We are given that $f(1) = 1$, and so the only inflection point of $f$ is the point $(1,1)$.  
			
			%part e
			\item  Intervals of concavity.  \\
			$f(x)$ is concave up when $f''(x)>0$.  From the sign chart, this is on the intervals $(-\infty ,-3)$ and $(1,4)$.  $f(x)$ is concave down when $f''(x)<0$.  From the sign chart, this is on the interval $(-3,1)$ and $(4,9)$. 
			
			%part f
			\item  Sketch the graph of $f$.
			
			\begin{image}
			\includegraphics[trim= 220 530 250 120]{Figure5.pdf}
			\end{image}
			
			\end{enumerate}
			
		\end{freeResponse}
		
		
\end{problem}
















%problem 2
\begin{problem}
Follow all the steps on the ``Graph Sketching Summary Sheet" and then graph the given function:
$$ f(x) = \frac{x^2 + x + 1}{x^2} $$

		\begin{freeResponse}  

		\begin{itemize}
		
			\item  \dfn{Domain}  \\
			The function is a rational function, and so the domain of the function is all real numbers except where the denominator equals zero.
			$$ x^2 = 0 \quad \Longrightarrow \quad x=0 $$
			So the domain of $f$ is $(-\infty ,0)\cup (0,\infty )$.
			
			
			
			\item  \dfn{$x,\, y$-intercepts}  \\
			To find any $x$-intercept(s), set $y=0$ and solve:
			$$ \frac{x^2 + x + 1}{x^2} = 0 $$
			$$ x^2 + x + 1 = 0 $$
			$$ x = \frac{-1 \pm \sqrt{1-4(1)(1)}}{2(1)} $$
			which has no real solutions.  Thus, $f$ has no $x$-intercepts.
			
			Since $x=0$ is not in the domain of $f$, $f$ has no $y$-intercepts as well.
			
			
			
			\item  \dfn{Symmetry}  \\
			
			It is clear that $f$ is not periodic.  Note that $f(1) = 3$ and $f(-1) = 1$.  So it cannot be for all values of $x$ that either $f(-x) = f(x)$ or $f(-x) = -f(x)$.  So $f$ is neither even nor odd, and therefore $f$ has no symmetry.
			
			\item  \dfn{Asymptotes}  \\
			\dfn{Vertical Asymptotes:}  Our only candidate is $x=0$, and so we compute the sided limits.
			$$ \lim_{x \to 0^-} \frac{x^2+x+1}{x^2} = \infty $$
			$$ \lim_{x \to 0^+} \frac{x^2+x+1}{x^2} = \infty $$
			Therefore, $x=0$ is the only vertical asymptote of $f$.
			
			\dfn{Horizontal Asymptotes:}  We compute the following limits:
			$$ \lim_{x \to \infty} \frac{x^2+x+1}{x^2} = 1 $$
			$$ \lim_{x \to -\infty} \frac{x^2+x+1}{x^2} = 1 $$
			and so the only horizontal asymptote of $f$ is $y=1$.
			
			\dfn{Slant Asymptote:}  Since our function is rational, we need to check for slant asymptotes.  Since the degree of the numerator is not one greater than the degree of the denominator, we do not have a slant asymptote.
			
			
			
			\item  \dfn{Increasing/Decreasing}  \\
			
			\begin{align*}
			f'(x) &= \frac{x^2(2x+1) - (x^2+x+1)(2x)}{x^4} \\
			&= \frac{2x^3 + x^2 - 2x^3 - 2x^2 - 2x}{x^4} \\
			&= \frac{-x^2 - 2x}{x^4} \\
			&= \frac{-x-2}{x^3}
			\end{align*}
			
			To find where $f'$ is both positive and negative, we need to find where $f'(x) = 0$ and where $f'(x)$ does not exist.  Clearly, $f'(x)$ does not exist when $x=0$.  To find when $f'(x) = 0$, we solve:
			$$ \frac{-x-2}{x^3} = 0 $$
			$$ -x-2 = 0 $$
			$$ -x = 2 $$
			$$ x = -2 $$
			
			Since $x=0$ is not in the domain of $f$, $x=-2$ is the only critical point of $f$.  To see where $f$ is increasing and decreasing, consider the following sign chart for $f'$:
			
%%%%%%%%%%%%%%%%%%%%%%%%%%%%%%%%%%%%%%%%%%%%%%%%%%%%%%%%%%%%%%%%%%%%%%%%%%%%%%%%%%%%%%
		
\begin{center}
\begin{image}
\begin{tikzpicture}

\draw [<->] (-4,0) -- (2,0);
\draw (0,0.1) -- (0,-0.1);
\draw (-2,0.1) -- (-2,-0.1);
\draw (-2,-0.3)node[below]{$-2$};
\draw (0,-0.3)node[below]{$0$};
\draw (-3.5,-0.8)node[below]{$f'(-3) = \frac{-1}{27}$};
\draw (-1,-1)node[below]{$f'(-1) = 1$};
\draw (1.2,-1)node[below]{$f'(1) = -3$};
\draw[red] (-1,1)node[below]{(+)};
\draw[blue] (1,1)node[below]{(-)};
\draw[blue] (-3,1)node[below]{(-)};
\draw (2.5,0)node[above]{$f'$};



\end{tikzpicture}
\end{image}
\end{center}

%%%%%%%%%%%%%%%%%%%%%%%%%%%%%%%%%%%%%%%%%%%%%%%%%%%%%%%%%%%%%%%%%%%%%%%%%%%%%%%%%%%%%%%%

So we see that $f$ is increasing on $[-2,0)$, and $f$ is decreasing on $(-\infty, -2] \cup (0,\infty)$.

			
			
			
			\item  \dfn{Local Extrema}  \\
			$f'$ changes from negative to positive at $x=-2$, so this is the location of a local minimum.  $f'$ also changes from positive to negative at $x=0$, but $f$ is not defined at $x=0$ and so this is not a local extreme value. $f$ has a local minimum at $\left( -2,\frac{3}{4} \right)$.
			
			
			
			\item  \dfn{Concavity}
			\begin{align*}
			f''(x) &= \frac{x^3(-1) - (-x-2)(3x^2)}{x^6} \\
			&= \frac{-x^3 + 3x^3 + 6x^2}{x^6} \\
			&= \frac{2x^3 + 6x^2}{x^6} \\
			&= \frac{2x+6}{x^4} \\
			&= \frac{2(x+3)}{x^4}
			\end{align*}
			
			To find where $f''$ is both positive and negative, we need to find where $f''(x) = 0$ and where $f''(x)$ does not exist.  Clearly, $f''(x)$ does not exist when $x=0$.  To find when $f''(x) = 0$, we solve:
			$$ \frac{2(x+3)}{x^4} = 0 $$
			$$ 2(x+3) = 0 $$
			$$ x=-3 $$
			
			To see where $f$ is concave up and concave down, consider the following sign chart for $f''$:
			
%%%%%%%%%%%%%%%%%%%%%%%%%%%%%%%%%%%%%%%%%%%%%%%%%%%%%%%%%%%%%%%%%%%%%%%%%%%%%%%%%%%%%%
		
\begin{center}
\begin{image}
\begin{tikzpicture}

\draw [<->] (-5,0) -- (2,0);
\draw (0,0.1) -- (0,-0.1);
\draw (-3,0.1) -- (-3,-0.1);
\draw (-3,-0.3)node[below]{$-3$};
\draw (0,-0.3)node[below]{$0$};
\draw (-4.5,-0.8)node[below]{$f''(-4) = \frac{-1}{128}$};
\draw (-1.5,-1)node[below]{$f''(-1) = 4$};
\draw (1.2,-1)node[below]{$f''(1) = 8$};
\draw[red] (-1.5,1)node[below]{(+)};
\draw[red] (1,1)node[below]{(+)};
\draw[blue] (-4,1)node[below]{(-)};
\draw (2.5,0)node[above]{$f''$};



\end{tikzpicture}
\end{image}
\end{center}

%%%%%%%%%%%%%%%%%%%%%%%%%%%%%%%%%%%%%%%%%%%%%%%%%%%%%%%%%%%%%%%%%%%%%%%%%%%%%%%%%%%%%%%%

			So we see that $f$ is concave up on $(-3,0) \cup (0,\infty)$, and $f$ is concave down on $(-\infty, -3)$.

			
			
			
			\item  \dfn{Inflection Points}  \\
			
			$f''(x)$ changes sign from negative to positive at $x=-3$, and $f$ is continuous at $x=-3$.  So $f$ has an inflection point at $\left( -3, \frac{7}{9} \right)$
			
\newpage

			\item  \dfn{The graph of $f$}
			
			\begin{image}
			\includegraphics[trim= 220 330 250 180]{Figure6.pdf}
			\end{image}
			
		\end{itemize}
		
		\end{freeResponse}
		
		
		

\end{problem}









\newpage









\begin{center}  \dfn{Graph Sketching Summary Sheet}  \end{center}

\begin{enumerate}

\item[1.]  \dfn{Domain}  \\
	Find the domain of the function $f$.

\item[2.]  \dfn{$x, \, y$ - intercepts}
	\begin{itemize}
		\item $x$-int: set $f(x)=0$ and solve for $x$.
		\item $y$-int: plug in 0 for $x$ and solve for $y$.
	\end{itemize}

\item[3.]  \dfn{Symmetry} 
	\begin{itemize}
		\item  Odd:  $f(-x)=-f(x)$, symmetric about the origin.
		\item  Even:  $f(-x)=f(x)$, symmetric about the y-axis.
		\item  Periodic:  $f(x+k)=f(x)$ for all x, period is k.
	\end{itemize}

\item[4.]  \dfn{Asymptotes} 
	\begin{itemize}
		\item  Vertical Asymptotes:  lines $x=a$ where either $\lim_{x \to a^-} f(x) = \pm \infty$ or $\lim_{x \to a^+} f(x) = \pm \infty$.
		\item  Horizontal Asymptotes:  lines $y=b$ where either $\lim_{x \to \infty} f(x) = b $ or $\lim_{x \to -\infty} f(x) = b $.
	\end{itemize}
	
\item[5.]  \dfn{Increasing/decreasing}
	\begin{itemize}
		\item  $f'(x) > 0 \quad \Longrightarrow \quad f$ is increasing.
		\item  $f'(x) < 0 \quad \Longrightarrow \quad f$ is decreasing.
	\end{itemize}
\item[6.]  \dfn{Maxima/Minima}  \\
	Use the First and/or Second Derivative Test(s) to determine whether a critical point of $f$ is a local maximum, minimum, or neither.
	
\item[7.]  \dfn{Concavity}
	\begin{itemize}
		\item  $f''(x) > 0 \quad \Longrightarrow \quad f$ is concave up.
		\item  $f''(x) < 0 \quad \Longrightarrow \quad f$ is concave down.
	\end{itemize}

\item[8.]  \dfn{Inflection Points}  \\
	$x=a$ is an inflection point of $f$ if BOTH $f''(x)$ changes sign at $x=a$ AND $f$ is continuous at $x=a$.

\end{enumerate}

\end{document} 