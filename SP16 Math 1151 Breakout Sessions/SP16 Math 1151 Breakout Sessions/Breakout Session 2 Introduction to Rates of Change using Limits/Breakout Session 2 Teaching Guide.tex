\documentclass[handout, nooutcomes]{ximera}
\usepackage{booktabs}
%% handout
%% space
%% newpage
%% numbers
%% nooutcomes

\renewcommand{\outcome}[1]{\marginpar{\null\vspace{2ex}\scriptsize\framebox{\parbox{0.75in}{\begin{raggedright}P\arabic{problem} Outcome: #1\end{raggedright}}}}}

\newcommand{\RR}{\mathbb R}
\renewcommand{\d}{\,d}
\newcommand{\dd}[2][]{\frac{d #1}{d #2}}
\renewcommand{\l}{\ell}
\newcommand{\ddx}{\frac{d}{dx}}
\everymath{\displaystyle}
\newcommand{\dfn}{\textbf}
\newcommand{\eval}[1]{\bigg[ #1 \bigg]}

\title{Breakout Session 2 Teaching Guide}  

\begin{document}
\begin{abstract}
 \textbf{Theme of calculus:} Calculus is the application of  \href{https://en.wikipedia.org/wiki/Derivative}{rates of change} and \href{https://en.wikipedia.org/wiki/Integral}{accumulation} to understand \href{https://en.wikipedia.org/wiki/Elementary_function}{famous functions} in their application to both real world and mathematical processes.

  \textbf{Goal of Math 1151 course:} Promote and cultivate an environment which improves students' ability to construct, organize, and demonstrate their knowledge of calculus.
\end{abstract}
\maketitle

\section{Teaching notes for problems}
\subsection*{Notes for problem 1}
For part (a), it may be useful to remind students the slope of a secant line gives the average rate of change of a function between two points and a tangent line gives the instantaneous rate of change at a particular point.
Since a linear function, by definition, has a constant rate of change the function will be identical to its secant and tangent lines at every point.

You could demonstrate this fact analytically, but it is probably quicker and easier just to argue using a graph.

For part (b), this example can be used to demonstrate the differences between a secant line and a tangent line.
Intuitively secant lines only give us information about a function at a ``fixed zoom level'', while tangent lines give us information by zooming infinitely close to a point.
You may also want to preview the fact that the slope of a tangent line on an interval gives us certain qualitative information on the graph of a function.

For part (c), this example can be used to demonstrate that a tangent line may intersect a graph at more than one point.
(Many students will mistakenly believe a tangent line can only intersect a graph at exactly one point.)
        
I think it's important to emphasize again that one visible difference between tangent lines and secant lines is that secant lines work at a ``fixed zoom level'' while tangent lines zoom infinitely close to a point.

\subsection*{Notes for problem 2}
For part (b), it may be useful to demonstrate the path of the ball by throwing a crumpled piece of paper up and down.
Many students may mistakenly think this graph gives the path of the ball: they be mentally thinking of a basketball being shot toward the net.

It may also be useful to draw a ``stop frame picture'' of the path of the ball and line each up picture its corresponding spot on the position-time graph.

For part (d), parts (a), (b) and (c) all lead up to this part.
The students should recognize that the given graph at the start of the problem needs to be modified to give the actual position-time graph of the ball.

Emphasize the important of knowing the domain (and range) of a function and remind the students of Breakout Session 1.


For part (f), it's important to emphasize the formula for average velocity, the formula's connection to the slope of a secant line, and how we can conjecture the instantaneous velocity using a table of values.

For part (g), it may be useful to draw, on a separate graph, the velocity-time graph of this function.
(Of course if you do this, tell students that they will learn how to construct this graph for themselves later!)
Emphasize that the shape of the position-time graph can gives us clues to the qualitative aspects of velocity.

For part (h), if you draw the velocity-time graph for part (g), you should refer to it.

\subsection*{Notes for problem 3}
This problem is from the Autumn 2015 Exam 1.
Many students incorrectly computed the average velocity in part (III), so before working this problem I suggest reminding students of the formula average velocity over an interval.

Many students also had problems in part (IV): several simply differentiated and others didn't see the connection between part (III) and this problem.
It's helpful to remind students that instantaneous velocity is the defined by limits.

For part (V).(c), many students didn't notice the connection between the slope of the tangent line and the instantaneous velocity at a point.
It's important to make this connection explicit.
\end{document} 
