\documentclass[nooutcomes]{ximera}
%% handout
%% space
%% newpage
%% numbers
%% nooutcomes


\renewcommand{\outcome}[1]{\marginpar{\null\vspace{2ex}\scriptsize\framebox{\parbox{0.75in}{\begin{raggedright}\textbf{P\arabic{problem} Outcome:} #1\end{raggedright}}}}}

\renewenvironment{freeResponse}{
\ifhandout\setbox0\vbox\bgroup\else
\begin{trivlist}\item[\hskip \labelsep\bfseries Solution:\hspace{2ex}]
\fi}
{\ifhandout\egroup\else
\end{trivlist}
\fi}

\newcommand{\RR}{\mathbb R}
\renewcommand{\d}{\,d}
\newcommand{\dd}[2][]{\frac{d #1}{d #2}}
\renewcommand{\l}{\ell}
\newcommand{\ddx}{\frac{d}{dx}}
\everymath{\displaystyle}
\newcommand{\dfn}{\textbf}
\newcommand{\eval}[1]{\bigg[ #1 \bigg]}

\title{Breakout Session 5 Teaching Guide}  

\begin{document}
\begin{abstract}
 \textbf{Theme of calculus:} Calculus is the application of  \href{https://en.wikipedia.org/wiki/Derivative}{rates of change} and \href{https://en.wikipedia.org/wiki/Integral}{accumulation} to understand \href{https://en.wikipedia.org/wiki/Elementary_function}{famous functions} in their application to both real world and mathematical processes.

  \textbf{Goal of Math 1151 course:} Promote and cultivate an environment which improves students' ability to construct, organize, and demonstrate their knowledge of calculus.
\end{abstract}
\maketitle

\section{Notes for problem 1}
The point of this warmup is to have students start thinking about different forms and the possible behavior of the limits.

Students will have difficulty constructing examples to illustrate each of the given properties.
(You may start with a graph, illustrate things algebraically, then compare and contrast the different functions.)

It's important to emphasize to the students that when writing these types of limits they should indicate the form of the limit after every reshaping.

\section{Notes for problem 2}
One important point of these problems is to give the students practice in correctly writing and justifying their solutions to limits.
They should \emph{nearly always} indicate the form of the limit before reshaping the limit.

Also when they indicate the form, if the denominator is $0$ students should also illustrate whether it's approaching from the right or left.

\section{Notes for problem 3}
When matching the formulas to the graphs, I suggest you emphasize the connections between the limit calculations and the properties of the graph.
(For instance, be sure to note that the limit tells how a function approaches its vertical and horizontal asymptotes.)

\section{Notes for problem 4}
It's important to stress that students \emph{must} indicate the form when justifying their limit calculations for vertical and horizontal asymptotes.
Make sure when you go over the answer, you're modeling how students should write their answers on quizzes and exams.

\end{document} 
