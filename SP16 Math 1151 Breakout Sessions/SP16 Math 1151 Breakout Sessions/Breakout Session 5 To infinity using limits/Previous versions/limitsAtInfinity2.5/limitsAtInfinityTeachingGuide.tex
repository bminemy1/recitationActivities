\documentclass[handout,nooutcomes]{ximera}
%% handout
%% space
%% newpage
%% numbers
%% nooutcomes


\newcommand{\RR}{\mathbb R}
\renewcommand{\d}{\,d}
\newcommand{\dd}[2][]{\frac{d #1}{d #2}}
\renewcommand{\l}{\ell}
\newcommand{\ddx}{\frac{d}{dx}}
\newcommand{\dfn}{\textbf}
\newcommand{\eval}[1]{\bigg[ #1 \bigg]}

\usepackage{multicol}

\renewenvironment{freeResponse}{
\ifhandout\setbox0\vbox\bgroup\else
\begin{trivlist}\item[\hskip \labelsep\bfseries Solution:\hspace{2ex}]
\fi}
{\ifhandout\egroup\else
\end{trivlist}
\fi} %% we can turn off input when making a master document

\title{Section - 2.5:  Limits at Infinity (Teaching Guide)}  

\begin{document}
\begin{abstract}		\end{abstract}
\maketitle

Here is a suggested structure for this recitation

\section*{Warm up:} 
	
	\begin{itemize}
	
	\item  \emph{10 minutes}:  Ask students to think about the Warm-up as they are waiting for class to begin.  Then discuss the Warm-Up as a class when class begins. Show students how to use the information about the limits to determine asymptotes as well as the behavior near them to determine which function matches which graph.
	
	\end{itemize}


\section*{Problem 1:}

	\begin{itemize}
	
	\item  \emph{10 minutes}:  Allow students to work on problem 1 in groups.  Assign different groups to start with (a), and (b) and then encourage the students to complete the other problem if time allows.  
	
	\item  \emph{10 minutes}:  Discuss problem 1 as a class.  Ask for input from the students in the groups who started with that problem.  Be sure to explain why the limit is negative as the limit approaches negative infinity.  It may help to give a concrete example like $\sqrt{(-2)^2} = 2$.  
		
	\end{itemize}
	
	
	
\section*{Problem 2:}

	\begin{itemize}
	
	\item  \emph{5 minutes}:  Allow students to work on problem 2 in groups. 
	
	\item  \emph{5 minutes}:  Discuss problem 2 as a class.   
	
	\end{itemize}
	
	
	
\section*{Problem 3:}

	\begin{itemize}
	
	\item  \emph{5 minutes}:  Allow students to work on problem 3 in groups.
	
	\item  \emph{10 minutes}:  Talk about \# 3 as a class.  Help give students good strategies for organizing the information before trying to graph it.  Be sure students understand that functions often cross horizontal asymptotes.  If the problem did not specify that $f(x)=3$ for $x>9$, what are the other possible ways that $\lim_{x \to \infty} f(x) = 3$?  Also specify to the students that it is not contradictory for $f(x) = 3$ for $x>9$ and $\lim_{x \to \infty} f(x) = 3$ (a similar idea could happen with slant asymptotes).  Talk about why, in general, it makes sense that the graph of a function can cross horizontal or slant asymptotes, but not vertical asymptotes.
	
	\end{itemize}
	
	
	

	
	
	
















\end{document}