\documentclass[handout]{ximera}
%% handout
%% space
%% newpage
%% numbers
%% nooutcomes

%I added the commands here so that I would't have to keep looking them up
\newcommand{\RR}{\mathbb R}
\renewcommand{\d}{\,d}
\newcommand{\dd}[2][]{\frac{d #1}{d #2}}
\renewcommand{\l}{\ell}
\newcommand{\ddx}{\frac{d}{dx}}
\everymath{\displaystyle}
\newcommand{\dfn}{\textbf}
\newcommand{\eval}[1]{\bigg[ #1 \bigg]}

\renewenvironment{freeResponse}{
\ifhandout\setbox0\vbox\bgroup\else
\begin{trivlist}\item[\hskip \labelsep\bfseries Solution:\hspace{2ex}]
\fi}
{\ifhandout\egroup\else
\end{trivlist}
\fi}


% \newcommand{\RR}{\mathbb R}
\renewcommand{\d}{\,d}
\newcommand{\dd}[2][]{\frac{d #1}{d #2}}
\renewcommand{\l}{\ell}
\newcommand{\ddx}{\frac{d}{dx}}
\newcommand{\dfn}{\textbf}
\newcommand{\eval}[1]{\bigg[ #1 \bigg]}

\renewenvironment{freeResponse}{
\ifhandout\setbox0\vbox\bgroup\else
\begin{trivlist}\item[\hskip \labelsep\bfseries Solution:\hspace{2ex}]
\fi}
{\ifhandout\egroup\else
\end{trivlist}
\fi} %% we can turn off input when making a master document

\title{Recitation 13}  

\begin{document}
\begin{abstract}		\end{abstract}
\maketitle

\section{Implicit Differentiation}
\begin{problem}[Warmup]
  Explain why both the $x$-coordinate and the $y$-coordinate are generally needed to find the slope of the tangent line at a point on the graph of an implicitly defined function?
\end{problem}
	\begin{freeResponse}
          If a function is defined \emph{implicitly} rather than \emph{explicitly}, then its graph will usually not pass the ``vertical line test".
          So fixing a value for $x$ will not typically specify a unique point on the graph because there will be more than one corresponding $y$-value.
          That is why you need to specify both the $x$ and the $y$ coordinate, to make sure that you are giving the coordinates for a unique point on the graph.
	\end{freeResponse}	

%problem 1
\begin{problem}
Consider the equation $x^2 + 4xy + 9y^2 = 9$

\begin{image}
\includegraphics[trim= 140 450 290 220]{Figure1.pdf}
\end{image}

	\begin{enumerate}
	
	%part a
	\item  Find $\dd[y]{x}$.
		\begin{freeResponse}
		$$ \ddx(x^2 + 4xy + 9y^2) = \ddx(9) $$
		$$ 2x + \left(4y + 4x \dd[y]{x} \right) + 18 y \dd[y]{x} = 0 $$
		$$ 4x \dd[y]{x} + 18y \dd[y]{x} = -2x - 4y $$
		$$ (4x+18y) \dd[y]{x} = -2x-4y $$
		$$ \dd[y]{x} = \frac{-2x-4y}{4x+18y} $$
		provided that $4x+18y \neq 0$.
		\end{freeResponse}
		
		
		
%part b
	\item  Find the equation(s) of the tangent line(s) when $x=0$.  Draw the tangent line(s) on the above picture.
		\begin{freeResponse}
		Plugging into the original equation, we see that when $x=0$ we have that 
		$$ 0^2 + 4(0)y + 9y^2 = 9 \qquad \Longrightarrow \qquad y^2 = 1 \qquad \Longrightarrow \qquad y = \pm 1 $$
		
		Then,
		$$\eval{\dd[y]{x}}_{(0,1)} = \frac{-4}{18} = -\frac{2}{9}$$
		$$\eval{\dd[y]{x}}_{(0,-1)} = \frac{4}{-18} = -\frac{2}{9} $$
		
		So there are two tangent lines to the graph of the solution set to the given equation when $x=0$.  The two solutions are $(0,1)$ and $(0,-1)$, and the tangent lines at both points have slope $-\frac{2}{9}$.  Thus, the equations to the two tangent lines are
		$$ y - 1 = -\frac{2}{9}(x-0)  \qquad \Longrightarrow \qquad y = -\frac{2}{9}x + 1 $$
		$$ y + 1 = -\frac{2}{9}(x-0) \qquad \Longrightarrow \qquad y = -\frac{2}{9}x - 1 $$
		\end{freeResponse}
		
		
		
%part c
	\item  Find the point(s) where the tangent line is horizontal.  Draw the point(s) and line(s) on the above picture.
		\begin{freeResponse}
		A line is horizontal if and only if its slope is 0.  So we are looking for the points $(x_0,y_0)$ such that $\eval{\dd[y]{x}}_{(x_0,y_0)} = 0$.  So we solve:
		$$ \frac{-2x-4y}{4x+18y} = 0\qquad \Longrightarrow \qquad -2x - 4y = 0 \qquad \Longrightarrow \qquad x=-2y$$
		
		But we also need the point to satisfy the given equation.  So, letting $x=-2y$, we solve:
		$$x^2 + 4xy + 9y^2 = 9 $$
		$$ (-2y)^2 + 4(-2y)y + 9y^2 = 9 $$
		$$ 4y^2 - 8y^2 + 9y^2 = 9 $$
		$$ 5y^2 = 9 $$
		$$ y^2 = \frac{9}{5} $$
		$$ y = \pm \frac{3}{\sqrt{5}} $$
		
		So there exists two solutions to the given equation where the tangent line to the graph has 0 slope.  Those two points are $\left( -\frac{6}{\sqrt{5}}, \frac{3}{\sqrt{5}} \right)$ and $\left( \frac{6}{\sqrt{5}}, - \frac{3}{\sqrt{5}} \right)$.
		\end{freeResponse}
	\end{enumerate}
\end{problem}

%problem 2
\begin{problem}
Find the slope of the tangent line (at any point $(x,y)$) to the graph of the solution set to the given equation

	\begin{enumerate}
	
	%part a
	\item  $e^{x^2 y^3} - 5x + 7y = 36$
			\begin{freeResponse}
			$$ e^{x^2 y^3} \left( 2xy^3 + x^2 (3y^2)\dd[y]{x} \right) - 5 + 7\dd[y]{x} = 0 $$
			$$ 2xy^3 e^{x^2y^3} + 3x^2y^2e^{x^2y^3}\dd[y]{x} - 5 + 7\dd[y]{x} = 0 $$
			$$ 3x^2y^2e^{x^2y^3}\dd[y]{x} + 7\dd[y]{x} = -2xy^3 e^{x^2y^3} + 5 $$
			$$ \left( 3x^2y^2e^{x^2y^3} + 7 \right) \dd[y]{x} = -2xy^3 e^{x^2y^3} + 5 $$
			$$  \dd[y]{x} = \frac{-2xy^3 e^{x^2y^3} + 5}{3x^2 y^2 e^{x^2y^3} + 7} $$
			\end{freeResponse}
			
			
			
	%part b
	\item  $7 = 22 \tan(y) + \frac{4}{x} - \frac{7}{y}$
			\begin{freeResponse}
			$$ 0 = 22 \sec^2 (y) \dd[y]{x} - \frac{4}{x^2} + \frac{7}{y^2} \dd[y]{x} $$
			$$ 22 \sec^2(y) \dd[y]{x} + \frac{7}{y^2} \dd[y]{x} = \frac{4}{x^2} $$
			$$ \dd[y]{x} = \frac{\frac{4}{x^2}}{22 \sec^2(y) + \frac{7}{y^2}} $$
			\end{freeResponse}
			
			
			
	%part c
	\item  $\cos(xy) - x^3 = 5y^3$
			\begin{freeResponse}
			$$ -\sin(xy) \cdot \left(y + x \dd[y]{x} \right) - 3x^2 = 15y^2 \dd[y]{x} $$
			$$ -y \sin(xy) - x \sin(xy) \dd[y]{x} - 3x^2 = 15y^2 \dd[y]{x} $$
			$$ x \sin(xy) \dd[y]{x} + 15y^2 \dd[y]{x} = -y \sin(xy) - 3x^2 $$
			$$ \dd[y]{x} = \frac{-y \sin(xy) - 3x^2}{x \sin(xy) + 15y^2} $$
			
			Provided that $x \sin(xy) + 15y^2 \neq 0$.  It is worth pointing out that this condition was not necessary for parts (a) and (b) because the denominator of those solutions cannot be $0$.  
			\end{freeResponse}
			
			
			
	\end{enumerate}
\end{problem}
	
%problem 3			
\begin{problem}
If $9x^2 + y^2 = 9$, find $\dd[^2 y]{x^2}$ as a function of $x$ and $y$.
		\begin{freeResponse}
		$18x + 2y \dd[y]{x} = 0 \qquad \Longrightarrow \qquad \dd[y]{x} = - \frac{18x}{2y} = - \frac{9x}{y}$.
		
		Then,
		$$ \dd[^2y]{x^2} = \ddx \dd[y]{x} = \ddx \left( - \frac{9x}{y} \right) = - \frac{(y)(9) - (9x)(\dd[y]{x} )}{y^2} = \frac{9x\dd[y]{x} - 9y}{y^2}. $$
		
		This is not a sufficient answer to the question since $\dd[y]{x}$ is a part of the solution.  But we know that $\dd[y]{x} = -\frac{9x}{y}$, and so we can substitute to get
		
		$$ \dd[^2y]{x^2} = \frac{(9x) \left(- \frac{9x}{y} \right) - 9y}{y^2} = \frac{-81x^2 - 9y^2}{y^3}. $$
		
		Provided that $y \neq 0$.  
		\end{freeResponse}
\end{problem}

\subsection*{Extra Problems}
\begin{problem}
  The volume of a doughnut with an inner radius of $a$ and an outer radius of $b$ is 
  \[
    V = \pi^2 \frac{(b+a)(b-a)^2}{4}.
  \]
  Find $db/da$ when the volume of a doughnut is $64\pi^2$.
  (This is problem 78 on page 202 of Briggs's \textit{Calculus: For Scientists and Engineers}.)
\end{problem}


\section{Derivatives of Logarithmic and Exponential Functions}

\subsection*{Things to know and eventually remember}
\begin{enumerate}
  \item[(1)]
    $e^{\ln x} = x$ for $x > 0$ and $\ln(e^x) = x$ for all $x$.

  \item[(2)]
    $y = \ln x \iff e^y = x$.

  \item[(3)]
    For every $x$ and $b >0$ we have
    $
      b^x = e^{x \ln b}
    $.

  \item[(4)]
    $\ln(xy) = \ln(x) + \ln(y)$.

  \item[(5)]
    $\ln(x/y) = \ln(x) - \ln(y)$.

  \item[(6)]
    $\ln(x^z) = z \ln(x)$.

  \item[(7)]
    $\ln(e) = 1$.
\end{enumerate}

\begin{problem}[warmup]
  \mbox{}
  	\begin{enumerate}
	
	%part 1
	\item[(1)]  If $f(x) = (x-2)^x$, then $f'(x) = x (x-2)^{x-1}$.

		\begin{freeResponse}
		False.  Any time that you have a function of $x$ raised to a function of $x$, in order to compute the derivative you need to use logarithmic differentiation (or something equivalent).
		\end{freeResponse}	
		
		
		
	%part 2
	\item[(2)]  If $f(x) = (3x)^x$, then $f'(x) = (3x)^x \ln (3x)$.

		\begin{freeResponse}
		False.  Same as part (1).  
		\end{freeResponse}	
	\end{enumerate}
\end{problem}

\begin{problem}
Find the derivatives of the following functions:
	\begin{enumerate}
	
	%part a
	\item  $f(x) = x^{e^x} + 7x$
		\begin{freeResponse}
		$f'(x) = \ddx \left(x^{e^x} \right) + \ddx(7x) = \ddx \left(x^{e^x} \right) + 7$.  So the real problem is to find $\ddx \left(x^{e^x} \right)$.  
		
		\begin{align*}
		\ddx \left( x^{e^x} \right) &= \ddx \left( e^{\ln x^{e^x}} \right) \\
		&= \ddx \left( e^{e^x \ln x} \right) \\
		&= e^{e^x \ln x} \left( e^x \ln x + \frac{e^x}{x} \right) \\
		&= x^{e^x} \left( e^x \ln x + \frac{e^x}{x} \right).
		\end{align*}
		
		Thus, $f'(x) = x^{e^x} \left( e^x \ln x + \frac{e^x}{x} \right) + 7$.  
		
		\end{freeResponse}
		
		
		
	%part b
	\item  $g(x) = (\ln x + 9)^{\sec(x^4)}$
		\begin{freeResponse}
			\begin{align*}
			g'(x) &= \ddx \left( (\ln x + 9)^{\sec(x^4)} \right) \\
			&= \ddx \left( e^{\sec(x^4) \ln ( \ln x + 9) } \right) \\
			&= e^{\sec(x^4) \ln ( \ln x + 9)} \left( 4x^3 \sec(x^4) \tan(x^4) \ln(\ln x + 9) + \sec(x^4) \frac{\frac{1}{x}}{\ln x + 9} \right) \\
			&= (\ln x + 9)^{\sec(x^4)} \left( 4x^3 \sec(x^4) \tan(x^4) \ln(\ln x + 9) + \frac{\sec(x^4)}{x(\ln x + 9)} \right) .
			\end{align*}
		\end{freeResponse}
		
		
		
	%part c
	\item  $h(x) = \sqrt[4]{\frac{(x^2 - 7)^5 \ln x}{\cos^7(x^2 - 5)}}$
		\begin{freeResponse}
		$$h'(x) = \ddx \left( \left( \frac{(x^2 - 7)^5 \ln x}{\cos^7(x^2 - 5)} \right)^{\frac{1}{4}} \right) $$
		
		$$=  \frac{1}{4} \left( \frac{(x^2 - 7)^5 \ln x}{\cos^7(x^2 - 5)} \right)^{\frac{-3}{4}} \cdot $$
		$$\left( \frac{\cos^7(x^2 - 5)(5(x^2-7)^4(2x) \ln x + (x^2-7)^5 \frac{1}{x}) - (x^2-7)^5 \ln x (7\cos^6(x^2-5) (-\sin(x^2 - 5))(2x))}{\cos^{14}(x^2 - 5)} \right)$$
		\end{freeResponse}
		
		
		
	\end{enumerate}
		
		
\end{problem}

\end{document} 