\documentclass[nooutcomes]{ximera}
\usepackage{booktabs}
%% handout
%% space
%% newpage
%% numbers
%% nooutcomes

\renewcommand{\outcome}[1]{\marginpar{\null\vspace{2ex}\scriptsize\framebox{\parbox{0.75in}{\begin{raggedright}\textbf{P\arabic{problem} Outcome:} #1\end{raggedright}}}}}

\renewenvironment{freeResponse}{
\ifhandout\setbox0\vbox\bgroup\else
\begin{trivlist}\item[\hskip \labelsep\bfseries Solution:\hspace{2ex}]
\fi}
{\ifhandout\egroup\else
\end{trivlist}
\fi}

\newcommand{\RR}{\mathbb R}
\renewcommand{\d}{\,d}
\newcommand{\dd}[2][]{\frac{d #1}{d #2}}
\renewcommand{\l}{\ell}
\newcommand{\ddx}{\frac{d}{dx}}
\everymath{\displaystyle}
\newcommand{\dfn}{\textbf}
\newcommand{\eval}[1]{\bigg[ #1 \bigg]}


\title{Breakout Session 12 Solutions}  

\begin{document}
\begin{abstract}
  % \textbf{A look back:} In the previous (February 16, 2016) Breakout Session you were introduced to a few physical interpretations of rates of change and practiced the last major differentiation technique~---~the chain rule.

  % \textbf{Overview:} In today's (February 18, 2016) Breakout Session you'll practice implicit differentiation and the geometrical interpretation of $dy/dx$ for an implicit equation.

  % \textbf{A look ahead:} In the next (February 23, 2016) Breakout Session you will practice logarithmic differentiation~---~a technique that helps simplify computations of some derivatives~---~and the connection between derivatives and derivatives of inverse functions.
\end{abstract}
\maketitle

% \section{Learning Outcomes}
% \label{section:learning-outcomes}
% The following outcomes are \emph{not an exhaustive} list of the skills you will need to develop and integrate for demonstration on quizzes and exams.
% This list is meant to be a starting point for conversation (with your Lecturer, Breakout Session Instructor, and fellow learners) for organizing your knowledge and monitoring the development of your skills.

% \begin{itemize}
%   \item 
%     Implicitly differentiate expressions. 

%   \item
%     Solve equations for $dy/dx$.

%   \item
%     Find the equation of the tangent line for curves that are not graphs of functions.

%   \item
%     Find derivatives of functions with rational exponents. 

%   \item 
%     Understand how changing the variable changes how we take the derivative.

%   \item 
%     Understand the derivatives of expressions that are not functions or not solved for $y$. 

% \end{itemize}
% \newpage

\begin{problem}
  \outcome{Implicitly differentiate expressions.}
  \outcome{Solve equations for $dy/dx$.}
  \outcome{Find the equation of the tangent line for curves that are not graphs of functions.}
  Consider the equation $x^2 + 4xy + 9y^2 = 9$
  \begin{image}
    \includegraphics[scale = 0.6]{Images/"Graph of skewed ellipse".png}
  \end{image}
  \begin{enumerate}
    \item
      Find $\dd[y]{x}$.
      \begin{freeResponse}
        Implicitly differentiate equation:
        \begin{align*}
          \ddx(x^2 + 4xy + 9y^2) = \ddx(9)
          &\implies 2x + \left(4y + 4x \dd[y]{x} \right) + 18 y \dd[y]{x} = 0
        \end{align*}

        Solving for $\dd[y]{x}$:
        \begin{align*}
          2x + \left(4y + 4x \dd[y]{x} \right) + 18 y \dd[y]{x} = 0
          &\implies 4x \dd[y]{x} + 18y \dd[y]{x} = -2x - 4y\\
          &\implies (4x+18y) \dd[y]{x} = -2x-4y\\
          &\implies \dd[y]{x} = \frac{-2x-4y}{4x+18y} \hspace{2em}\mbox{provided that $4x+18y \neq 0$.}
        \end{align*}
      \end{freeResponse}
		
    \item  Find the equation(s) of the tangent line(s) when $x=0$.  Draw the tangent line(s) on the above picture.
      \begin{freeResponse}
        Finding points on tangent lines:
        \begin{align*}
          x = 0 &\implies 0^2 + 4(0)y + 9y^2 = 9\\
                &\implies y^2 = 1 \\
                &\implies y = \pm 1\\
          &\implies \mbox{$(0, 1)$ or $(0,-1)$}
        \end{align*}

        Slope of tangent line at $(0,1)$:
        \begin{align*}
          \eval{\dd[y]{x}}_{(0,1)} &= \frac{-4}{18} \\
                                  &= -\frac{2}{9}
        \end{align*}

        Slope of tangent line at $(0, -1)$:
        \begin{align*}
          \eval{\dd[y]{x}}_{(0,-1)} &= \frac{4}{-18}\\
                                    &= -\frac{2}{9}
        \end{align*}

        Equation of tangent line at $(0, 1)$:
        \begin{align*}
          y - 1 = -\frac{2}{9}(x-0)  &\implies y = -\frac{2}{9}x + 1 
        \end{align*}
        
        Equation of tangent line at $(0, -1)$:
        \begin{align*}
           y + 1 = -\frac{2}{9}(x-0) &\implies y = -\frac{2}{9}x - 1 
        \end{align*}

        Graph with these two tangent lines:
        \begin{image}
          \includegraphics[scale = 0.6]{Images/"Graph of skewed ellipse with two tangents".png}
        \end{image}
      \end{freeResponse}

    \item  Find the point(s) where the tangent line is horizontal.  Draw the point(s) and line(s) on the above picture.
      \begin{freeResponse}
        A line is horizontal if and only if its slope is 0.  So we are looking for the points $(x_0,y_0)$ such that $\eval{\dd[y]{x}}_{(x_0,y_0)} = 0$.  So we solve:
	$$ \frac{-2x-4y}{4x+18y} = 0\qquad \Longrightarrow \qquad -2x - 4y = 0 \qquad \Longrightarrow \qquad x=-2y$$
		
	But we also need the point to satisfy the given equation.  So, letting $x=-2y$, we solve:
	$$x^2 + 4xy + 9y^2 = 9 $$
	$$ (-2y)^2 + 4(-2y)y + 9y^2 = 9 $$
	$$ 4y^2 - 8y^2 + 9y^2 = 9 $$
	$$ 5y^2 = 9 $$
	$$ y^2 = \frac{9}{5} $$
	$$ y = \pm \frac{3}{\sqrt{5}} $$
		
	So there exists two solutions to the given equation where the tangent line to the graph has 0 slope.  Those two points are $\left( -\frac{6}{\sqrt{5}}, \frac{3}{\sqrt{5}} \right)$ and $\left( \frac{6}{\sqrt{5}}, - \frac{3}{\sqrt{5}} \right)$.

        Graph with horizontal tangent lines:
        \begin{image}
          \includegraphics[scale = 0.6]{Images/"Graph of skewed ellipse with horizontal tangents".png}
        \end{image}
	\end{freeResponse}
  \end{enumerate}
\end{problem}

\begin{problem}
  \outcome{Implicitly differentiate expressions.}
  \outcome{Solve equations for $dy/dx$.}
  \outcome{Find the equation of the tangent line for curves that are not graphs of functions.}  
  A part of the curve with equation $\cos(\pi x y) + x + y = 1$ is sketched below.
  \begin{image}
    \includegraphics[scale = 0.3]{Images/"Bean graph".png}
  \end{image}
  \begin{itemize}
    \item[(a)]
      Use the implicit differentiation to find the derivative $dy/dx$.
      \begin{freeResponse}
        Implicitly differentiate equation:
        \begin{align*}
          \ddx( \cos(\pi x y) + x + y) = \ddx(1) &\implies -\sin(\pi x y)\cdot(\pi y + \pi x \dd[y]{x}) + 1 + \dd[y]{x} = 0 
        \end{align*}

        Solve for $\dd[y]{x}$:
        \begin{align*}
          -\sin(\pi x y)\cdot(\pi y + \pi x \dd[y]{x}) + 1 + \dd[y]{x} = 0 &\implies -\pi y \sin(\pi x y) - \pi x \dd[y]{x} \sin(\pi x y) + 1 + \dd[y]{x} = 0 \\
          &\implies  \bigl(1 - \pi x \sin(\pi x y)\bigr)\dd[y]{x} = \pi y \sin(\pi x y) - 1 \\
          &\implies \dd[y]{x} = \frac{\pi y \sin(\pi x y) - 1}{1 - \pi x \sin(\pi x y)} \hspace{2em} \mbox{if $1 - \pi x \sin(\pi x y) \ne 0$}\\
        \end{align*}
      \end{freeResponse}

    \item[(b)]
      Consider the point $(1, 1)$.
      Show (algebraically) that this point lies on the curve.
      \begin{freeResponse}
        $(1, 1)$ lies on this curve:
        \begin{align*}
          \cos(\pi \cdot 1 \cdot 1) + 1 + 1 &= \cos(\pi) + 2\\
          &= -1 + 2 = 1
        \end{align*}
      \end{freeResponse}

    \item[(c)]
      Find the equation of the line tangent to the curve at $(1,1)$.
      Draw this line in the figure above.
      \begin{freeResponse}
        Slope of tangent line:
        \begin{align*}
          \eval{\dd[y]{x}}_{(1,1)} &= \frac{\pi \cdot 1 \cdot\sin(\pi 1 \cdot 1) - 1}{1 - \pi \cdot 1 \cdot \sin(\pi 1 \cdot 1)} \\
          &= \frac{\pi\sin(\pi) - 1}{1 - \pi \sin(\pi)}\\
          &= \frac{-1}{1} = -1
        \end{align*}

        Equation of tangent line:
        \begin{align*}
          y - 1 = -1(x - 1) &\implies y = -x + 2
        \end{align*}

        Graph of tangent line:
        \begin{image}
          \includegraphics[scale = 0.7]{Images/"Bean graph with tangent".png}
        \end{image}
      \end{freeResponse}

  \end{itemize}
\end{problem}

\begin{problem}
  \outcome{Implicitly differentiate expressions.}
  \outcome{Solve equations for $dy/dx$.}
  \outcome{Understand the derivatives of expressions that are not functions or not solved for $y$.}
  For the each of the following implicit equations, find a formula for the slope of the tangent line at a point $(x, y)$.
  \begin{enumerate}
    \item
      $e^{x^2 y^3} - 5x + 7y = 36$
			\begin{freeResponse}
			$$ e^{x^2 y^3} \left( 2xy^3 + x^2 (3y^2)\dd[y]{x} \right) - 5 + 7\dd[y]{x} = 0 $$
			$$ 2xy^3 e^{x^2y^3} + 3x^2y^2e^{x^2y^3}\dd[y]{x} - 5 + 7\dd[y]{x} = 0 $$
			$$ 3x^2y^2e^{x^2y^3}\dd[y]{x} + 7\dd[y]{x} = -2xy^3 e^{x^2y^3} + 5 $$
			$$ \left( 3x^2y^2e^{x^2y^3} + 7 \right) \dd[y]{x} = -2xy^3 e^{x^2y^3} + 5 $$
			$$  \dd[y]{x} = \frac{-2xy^3 e^{x^2y^3} + 5}{3x^2 y^2 e^{x^2y^3} + 7} $$
			\end{freeResponse}
			
     \item
       $7 = 22 \tan(y) + \frac{4}{x} - \frac{7}{y}$
			\begin{freeResponse}
			$$ 0 = 22 \sec^2 (y) \dd[y]{x} - \frac{4}{x^2} + \frac{7}{y^2} \dd[y]{x} $$
			$$ 22 \sec^2(y) \dd[y]{x} + \frac{7}{y^2} \dd[y]{x} = \frac{4}{x^2} $$
			$$ \dd[y]{x} = \frac{\frac{4}{x^2}}{22 \sec^2(y) + \frac{7}{y^2}} $$
			\end{freeResponse}

     \item
       $\cos(xy) - x^3 = 5y^3$
			\begin{freeResponse}
			$$ -\sin(xy) \cdot \left(y + x \dd[y]{x} \right) - 3x^2 = 15y^2 \dd[y]{x} $$
			$$ -y \sin(xy) - x \sin(xy) \dd[y]{x} - 3x^2 = 15y^2 \dd[y]{x} $$
			$$ x \sin(xy) \dd[y]{x} + 15y^2 \dd[y]{x} = -y \sin(xy) - 3x^2 $$
			$$ \dd[y]{x} = \frac{-y \sin(xy) - 3x^2}{x \sin(xy) + 15y^2} $$
			
			Provided that $x \sin(xy) + 15y^2 \neq 0$.  It is worth pointing out that this condition was not necessary for parts (a) and (b) because the denominator of those solutions cannot be $0$.  
			\end{freeResponse}
	\end{enumerate}
\end{problem}

\begin{problem}
  If $9x^2 + y^2 = 9$, find $\dd[^2 y]{x^2}$ as a function of $x$ and $y$.
  		\begin{freeResponse}
		$18x + 2y \dd[y]{x} = 0 \qquad \Longrightarrow \qquad \dd[y]{x} = - \frac{18x}{2y} = - \frac{9x}{y}$.
		
		Then,
		$$ \dd[^2y]{x^2} = \ddx \dd[y]{x} = \ddx \left( - \frac{9x}{y} \right) = - \frac{(y)(9) - (9x)(\dd[y]{x} )}{y^2} = \frac{9x\dd[y]{x} - 9y}{y^2}. $$
		
		This is not a sufficient answer to the question since $\dd[y]{x}$ is a part of the solution.  But we know that $\dd[y]{x} = -\frac{9x}{y}$, and so we can substitute to get
		
		$$ \dd[^2y]{x^2} = \frac{(9x) \left(- \frac{9x}{y} \right) - 9y}{y^2} = \frac{-81x^2 - 9y^2}{y^3}. $$
		
		Provided that $y \neq 0$.  
		\end{freeResponse}
\end{problem}

\section{Extra Problems for Personal Practice}
\begin{problem}
  Explain why both the $x$-coordinate and the $y$-coordinate are generally needed to find the slope of the tangent line at a point on the graph of an implicitly defined function?
	\begin{freeResponse}
          If a function is defined \emph{implicitly} rather than \emph{explicitly}, then its graph will usually not pass the ``vertical line test".
          So fixing a value for $x$ will not typically specify a unique point on the graph because there will be more than one corresponding $y$-value.
          That is why you need to specify both the $x$ and the $y$ coordinate, to make sure that you are giving the coordinates for a unique point on the graph.
	\end{freeResponse}	
\end{problem}

\begin{problem}
  The volume of a doughnut with an inner radius of $a$ and an outer radius of $b$ is 
  \[
    V = \pi^2 \frac{(b+a)(b-a)^2}{4}.
  \]
  Find $db/da$ when the volume of a doughnut is $64\pi^2$.
\begin{freeResponse}
  We have to (implicitly) differentiate
  \[
    64\pi^2 = \pi^2 \frac{(b+a)(b-a)^2}{4}.
  \]
  
  Before doing this we'll perform a bit of algebra to simplify our calculuations:
  \begin{align*}
    64\pi^2 = \pi^2 \frac{(b+a)(b-a)^2}{4} &\implies 256 =  (b+a)(b-a)^2.
  \end{align*}

  Now we'll differentiate with respect to $a$:
  \begin{align*}
    256 = (b+a)(b-a)^2 &\implies 0 = \left(\frac{db}{da} + 1\right)(b - a)^2 + (b + a) 2 \cdot (b - a) \cdot \left(\frac{db}{da} - a\right) \\
    &\implies 0 = \frac{db}{da}(b-a)^2 + (b-a)^2 + 2(b^2-a^2) \frac{db}{da} - 2(b^2 - a^2).
  \end{align*}

  To finish we solve for $\frac{db}{da}$:
  \begin{align*}
    0 &= \frac{db}{da}(b-a)^2 + (b-a)^2 + 2(b^2-a^2) \frac{db}{da} - 2(b^2 - a^2) \\
      &\implies
        - \left(\frac{db}{da}(b-a)^2 + 2(b^2-a^2) \frac{db}{da}\right) = (b-a)^2 - 2(b^2 - a^2)\\
    &\implies
      \frac{db}{da} = \frac{(b-a)^2 - 2(b^2 - a^2)}{-((b-a)^2 + 2(b^2-a^2))}.
  \end{align*}
\end{freeResponse}	
\end{problem}
\end{document} 
