\documentclass[handout,nooutcomes]{ximera}
\usepackage{booktabs}
%% handout
%% space
%% newpage
%% numbers
%% nooutcomes

\renewcommand{\outcome}[1]{\marginpar{\null\vspace{2ex}\scriptsize\framebox{\parbox{0.75in}{\begin{raggedright}P\arabic{problem} Outcome: #1\end{raggedright}}}}}

\renewenvironment{freeResponse}{
\ifhandout\setbox0\vbox\bgroup\else
\begin{trivlist}\item[\hskip \labelsep\bfseries Solution:\hspace{2ex}]
\fi}
{\ifhandout\egroup\else
\end{trivlist}
\fi}

\newcommand{\RR}{\mathbb R}
\renewcommand{\d}{\,d}
\newcommand{\dd}[2][]{\frac{d #1}{d #2}}
\renewcommand{\l}{\ell}
\newcommand{\ddx}{\frac{d}{dx}}
\everymath{\displaystyle}
\newcommand{\dfn}{\textbf}
\newcommand{\eval}[1]{\bigg[ #1 \bigg]}


\title{Breakout Session 10 Teaching Guide}  

\begin{document}
\begin{abstract}
  % \textbf{A look back:} In the previous (February 9, 2016) Breakout Session you were introduced to the sum, product, and quotient rules for computing derivatives.
  % These rules help us to compute derivatives quickly and accurately.

  % \textbf{Overview:} In today's (February 11, 2016) Breakout Session you'll practice how to evaluate special trigonometric limits and the basic derivatives of trigonometric functions.

  % \textbf{A look ahead:} In the next (February 16, 2016) Breakout Session you will practice real world interpretations of derivatives (as rates of change) and learn the last major differentiation rule.
\end{abstract}
\maketitle

\section{Notes for problem 1}
Before working these problems remind students about the two standard limits they'll be using.

For part (a), when working through this problem you should emphasize the substitution of $u$ for $8x$.

For part (b), you should remind students of the Pythagorean trigonometric identity $\cos^2(x) + \sin^2(x) = 1$ when evaluating this limit.
Also, you can solve this problem by factoring the numerator.
(It's good to show students both ways of solving the problem.)

For part (c), you should work carefully through this problem and remind the students that $\lim_{x \to 0} \sin(x)/x = 1$ implies $\lim_{x \to 0} x/\sin(x) = 1$.
Also when computing $\lim_{x \to 0} 5x/\sin(5x)$ be explicit about the use of substitution.

\section{Notes for problem 2}
These are fairly standard differentiation problems involving trigonometric functions.
Before working through them you should remind students about the derivatives of sin, cos, tan, csc, sec, and cot and the connection between the derivative of a trigonometric function and its co-function.



\section{Notes for problem 4}
This problem will be difficult for the students to work through.
Carefully go through the steps involved using the continuity checklist and the ``differentiability checklist''.
(Unfortunately, there is no explicit differentiability checklist given in the book, but you should remind students that a two-sided limit exists if and only if the right-sided limit and left-sided limit both exist and are equal.

\end{document} 
