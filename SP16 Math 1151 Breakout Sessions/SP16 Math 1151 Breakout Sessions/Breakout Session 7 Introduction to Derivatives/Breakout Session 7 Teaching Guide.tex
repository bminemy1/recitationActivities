\documentclass[nooutcomes]{ximera}
\usepackage{booktabs}
%% handout
%% space
%% newpage
%% numbers
%% nooutcomes

\renewcommand{\outcome}[1]{\marginpar{\null\vspace{2ex}\scriptsize\framebox{\parbox{0.75in}{\begin{raggedright}\textbf{P\arabic{problem} Outcome:} #1\end{raggedright}}}}}

\renewenvironment{freeResponse}{
\ifhandout\setbox0\vbox\bgroup\else
\begin{trivlist}\item[\hskip \labelsep\bfseries Solution:\hspace{2ex}]
\fi}
{\ifhandout\egroup\else
\end{trivlist}
\fi}

\newcommand{\RR}{\mathbb R}
\renewcommand{\d}{\,d}
\newcommand{\dd}[2][]{\frac{d #1}{d #2}}
\renewcommand{\l}{\ell}
\newcommand{\ddx}{\frac{d}{dx}}
\everymath{\displaystyle}
\newcommand{\dfn}{\textbf}
\newcommand{\eval}[1]{\bigg[ #1 \bigg]}


\title{Breakout Session 7 Teaching Guide}  

\begin{document}
\begin{abstract}
 \textbf{Theme of calculus:} Calculus is the application of  \href{https://en.wikipedia.org/wiki/Derivative}{rates of change} and \href{https://en.wikipedia.org/wiki/Integral}{accumulation} to understand \href{https://en.wikipedia.org/wiki/Elementary_function}{famous functions} in their application to both real world and mathematical processes.

  \textbf{Goal of Math 1151 course:} Promote and cultivate an environment which improves students' ability to construct, organize, and demonstrate their knowledge of calculus.
\end{abstract}
\maketitle

\section{Notes for Problem 1}
For part (a), one visually based definition you could remind students of is
\begin{quote}
  A line is \textsl{\textbf{tangent to a graph at a particular point}} if when we zoom in on the point the graph is nearly indistinguishable from the tangent line.
\end{quote}
This definition may help those students who may be tempted to incorrectly state ``a tangent line is a line that intersects the graph at only one point.''

For part (b), you should show that $f'(0)$ doesn't exist both graphically (there is a vertical tangent line at $x = 0$) and, if you feel there is time, from the definition of a derivative.
        
Taking the latter approach will review infinite limits.
\textbf{Important Note:} However, be careful not to confuse students into thinking that $f$ has a vertical asymptote based on this calculation!
(Of course $f'$ is the function that has a vertical asymptote at $x = 0$, but, at this stage, some students may not be clear on the distinction between $f$ and $f'$.)

\section{Notes for Problem 2}
These figures are interactive in the eBook version our textbook. 
If you have a computer and projector setup in you class, it may be helpful to preload these interactive figures to demonstrate to students when they work on this problem.


This will be a tricky conceptual problems for the students: some of them will be intimated by the differences in notation.
(Also, some may be tempted to use both notations simultaneously.)

It's important to emphasize that both notations are mathematically equivalent (they produce the same mathematical information) \emph{but} psychologically inequivalent (one form may be easier to apply in a certain context).

\section{Notes for Problem 3}
These problems help illustrate the ``mathematically equivalent but psychologically inequivalent'' point mentioned above.
Also, note which definition of instantaneous rate of change you will use before starting a problem.
If you think there is time (there probably won't be), you can show how to find the slope of the tangent line using each of the equivalent definitions above.

You should take the algebra slow here and \emph{do not} skip steps.
No matter which form (of the definition of instantaneous rate of change) you use students will probably find the algebra tough going.


\section{Notes for Problem 4}
This is a trick question, since the derivative doesn't exist at $x = 5$.
You should go through the demonstration that $\lim_{x \to 5} (f(x) - f(5))/(x - 5)$ does not exist slowly and emphasize how you remove the absolute value sign in your computations by using right-sided and left-sided limits.
In general, the absolute value sign will be tricky for the students.

For parts (a) and (b), the students will have some trouble computing $f'$ for a general $a$, so also take things slow when you work through this problem.


When you're done with this problem, I suggest you draw the graph of $f'$ (but not on the same graph as $f$!).
As a preview of what's to come, you can say sometime similar to
\begin{quote}
  Limits give qualitative information about functions.
  Derivatives are defined in terms of limits, so they too must give new qualitative information about functions.
  At the simplest level, they can tell you where functions are increasing, decreasing, and constant.
  At a deeper level, they can help identify the ``peaks'' and ``valleys'' of the graph of a function.
\end{quote}
\end{document} 
