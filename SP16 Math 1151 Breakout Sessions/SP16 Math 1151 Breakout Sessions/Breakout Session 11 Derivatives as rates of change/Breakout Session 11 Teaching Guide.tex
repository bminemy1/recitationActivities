\documentclass[nooutcomes]{ximera}
\usepackage{booktabs}
%% handout
%% space
%% newpage
%% numbers
%% nooutcomes

\renewcommand{\outcome}[1]{\marginpar{\null\vspace{2ex}\scriptsize\framebox{\parbox{0.75in}{\begin{raggedright}\textbf{P\arabic{problem} Outcome:} #1\end{raggedright}}}}}

\renewenvironment{freeResponse}{
\ifhandout\setbox0\vbox\bgroup\else
\begin{trivlist}\item[\hskip \labelsep\bfseries Solution:\hspace{2ex}]
\fi}
{\ifhandout\egroup\else
\end{trivlist}
\fi}

\newcommand{\RR}{\mathbb R}
\renewcommand{\d}{\,d}
\newcommand{\dd}[2][]{\frac{d #1}{d #2}}
\renewcommand{\l}{\ell}
\newcommand{\ddx}{\frac{d}{dx}}
\everymath{\displaystyle}
\newcommand{\dfn}{\textbf}
\newcommand{\eval}[1]{\bigg[ #1 \bigg]}


\title{Breakout Session 11 Teaching Guide}  

\begin{document}
\begin{abstract}

\end{abstract}
\maketitle

\section{Notes for problem 1}
This is a good warmup problem to test if students can identify the
``inner'' function and the ``outer'' function.
For time, I suggest you only work through one part.

\section{Notes for problem 3}
This will be a difficult but good problem for students to work through.
Be explicit in showing how the values of the derivatives are computed.
It may be helpful to build a table of values based off the graph.

\section{Notes for problem 4}
As usual when solving tangent line problems, stress the importance of finding the slope (using derivatives), finding a point on the line, and writing the final answer in point-slope form.

\section{Notes for problem 5}
Stress which function is the ``inner'' function and which function is the ``outer'' function.
This problem will test students ability to recognize composition of functions and application of the chain rule to functions whose formulas look similar but are different.

\section{Notes for problem 6}
When solving this problem it’s helpful to use the story in the solutions to illustrate the problem.
I read it to my class last semester and I think they both enjoyed it (in a humerous way) and it helped them understand what is going on as well.




\end{document} 
