\documentclass[nooutcomes]{ximera}
\usepackage{booktabs}
%% handout
%% space
%% newpage
%% numbers
%% nooutcomes

\renewcommand{\outcome}[1]{\marginpar{\null\vspace{2ex}\scriptsize\framebox{\parbox{0.75in}{\begin{raggedright}\textbf{P\arabic{problem} Outcome:} #1\end{raggedright}}}}}

\renewenvironment{freeResponse}{
\ifhandout\setbox0\vbox\bgroup\else
\begin{trivlist}\item[\hskip \labelsep\bfseries Solution:\hspace{2ex}]
\fi}
{\ifhandout\egroup\else
\end{trivlist}
\fi}

\newcommand{\RR}{\mathbb R}
\renewcommand{\d}{\,d}
\newcommand{\dd}[2][]{\frac{d #1}{d #2}}
\renewcommand{\l}{\ell}
\newcommand{\ddx}{\frac{d}{dx}}
\everymath{\displaystyle}
\newcommand{\dfn}{\textbf}
\newcommand{\eval}[1]{\bigg[ #1 \bigg]}


\title{Breakout Session 13 Solutions}  

\begin{document}
\begin{abstract}
  % \textbf{A look back:} In the previous (February 18, 2016) Breakout Session you practiced differentiating implicit equations and geometrically interpertating $dy/dx$.

  % \textbf{Overview:} In today's (February 23, 2016) Breakout Session you'll practice differentiating inverse (trigonometric) functions and the connection between the derivative of $f^{-1}$ with the derivative of $f$.

  % \textbf{A look ahead:} In the next (February 25, 2016) Breakout Session you be introduced to related rates and take the computation derivatives quiz.
\end{abstract}
\maketitle

% \section{Learning Outcomes}
% \label{section:learning-outcomes}
% The following outcomes are \emph{not an exhaustive} list of the skills you will need to develop and integrate for demonstration on quizzes and exams.
% This list is meant to be a starting point for conversation (with your Lecturer, Breakout Session Instructor, and fellow learners) for organizing your knowledge and monitoring the development of your skills.

% \begin{itemize}
%   \item
%     Take derivatives of logarithms and exponents of all bases. 
%   \item
%     Take derivatives of functions raised to functions.
%   \item
%     Apply the generalized power rule. 
%   \item 
%     Recognize the difference between a variable as the base and a variable as the exponent.
%   \item
%     Identify situations where logs can be used to help find derivatives. 
%   \item
%     Use logarithmic differentiation to simplify taking derivatives. 
%   \item
%     Find derivatives of inverse functions in general. 
%   \item
%     Understand how the derivative of an inverse function relates to the original derivative. 
%   \item
%     Derive the derivatives of inverse trig functions.
%   \item
%     Recall the meaning and properties of inverse trig functions. 
%   \item
%     Take derivatives which involve inverse trig functions. 
% \end{itemize}
% \newpage

% \subsection*{Facts to know and eventually remember}
% \begin{enumerate}
%   \item[(1)]
%     $e^{\ln x} = x$ for $x > 0$ and $\ln(e^x) = x$ for all $x$.

%   \item[(2)]
%     $y = \ln x \iff e^y = x$.

%   \item[(3)]
%     For every $x$ and $b >0$ we have
%     $
%       b^x = e^{x \ln b}
%     $.

%   \item[(4)]
%     $\ln(xy) = \ln(x) + \ln(y)$.

%   \item[(5)]
%     $\ln(x/y) = \ln(x) - \ln(y)$.

%   \item[(6)]
%     $\ln(x^z) = z \ln(x)$.

%   \item[(7)]
%     $\ln(e) = 1$.
% \end{enumerate}

\begin{problem}
  \outcome{Recognize the difference between a variable as the base and a variable as the exponent.}
  True or False:
  \begin{enumerate}
    \item[(1)]
      If $f(x) = (x-2)^x$, then $f'(x) = x (x-2)^{x-1}$.
      \begin{freeResponse}
        False.
        Any time that you have a function of $x$ raised to a function of $x$, in order to compute the derivative you need to use logarithmic differentiation (or something equivalent).

        Correct derivative of $f$:
        \begin{align*}
          f(x) &= (x-2)^x \implies f(x) = e^{x \ln(x-2)} \\
          &\implies f'(x) = e^{x \ln(x-2)} \cdot \left(1\cdot\ln(x-2) + x \cdot \frac{1}{x-2}\right) \\
          &\implies f'(x) = (x-2)^x \cdot \left(\ln(x-2) + \frac{x}{x-2}\right)
        \end{align*}
        \end{freeResponse}	
		
      \item[(2)]
        If $f(x) = (3x)^x$, then $f'(x) = (3x)^x \ln (3x)$.
	\begin{freeResponse}
          False.  Same as part (1).

          Correct derivative of $f$:
          \begin{align*}
            f(x) &= (3x)^x \implies f(x) = e^{x \cdot \ln(3x)} \\
            &\implies f'(x)  = e^{x \cdot \ln(3x)} \left(1 \cdot \ln(3x) + x \cdot \frac{3}{3x} \right) \\
            &\implies f'(x)  = (3x)^x \left(\ln(3x) + 1 \right)
          \end{align*}
	\end{freeResponse}	
	\end{enumerate}
\end{problem}

\begin{problem}
\outcome{Recognize the difference between a variable as the base and a variable as the exponent.}
\outcome{Identify situations where logs can be used to help find derivatives.}
Find the derivatives of the following functions:
	\begin{enumerate}
	
	%part a
	\item  $f(x) = x^{e^x} + 7x$
		\begin{freeResponse}
		$f'(x) = \ddx \left(x^{e^x} \right) + \ddx(7x) = \ddx \left(x^{e^x} \right) + 7$.  So the real problem is to find $\ddx \left(x^{e^x} \right)$.  
		
		\begin{align*}
		\ddx \left( x^{e^x} \right) &= \ddx \left( e^{\ln x^{e^x}} \right) \\
		&= \ddx \left( e^{e^x \ln x} \right) \\
		&= e^{e^x \ln x} \left( e^x \ln x + \frac{e^x}{x} \right) \\
		&= x^{e^x} \left( e^x \ln x + \frac{e^x}{x} \right).
		\end{align*}
		
		Thus, $f'(x) = x^{e^x} \left( e^x \ln x + \frac{e^x}{x} \right) + 7$.  
		
		\end{freeResponse}
		
		
		
\outcome{Take derivatives of functions raised to functions.}
\outcome{Take derivatives of logarithms and exponents of all bases.}
	\item  $g(x) = (\ln x + 9)^{\sec(x^4)}$
		\begin{freeResponse}
			\begin{align*}
			g'(x) &= \ddx \left( (\ln x + 9)^{\sec(x^4)} \right) \\
			&= \ddx \left( e^{\sec(x^4) \ln ( \ln x + 9) } \right) \\
			&= e^{\sec(x^4) \ln ( \ln x + 9)} \left( 4x^3 \sec(x^4) \tan(x^4) \ln(\ln x + 9) + \sec(x^4) \frac{\frac{1}{x}}{\ln x + 9} \right) \\
			&= (\ln x + 9)^{\sec(x^4)} \left( 4x^3 \sec(x^4) \tan(x^4) \ln(\ln x + 9) + \frac{\sec(x^4)}{x(\ln x + 9)} \right) .
			\end{align*}
		\end{freeResponse}
		
		
		
	%part c
	\item  $h(x) = \frac{(x^2 - 7)^5}{\cos^7(x^2 - 5)}$
          \begin{freeResponse}
            Rewrite $h(x)$ using properties of logarithms:
            \begin{align*}
              \ln h(x) &= \ln \left( \frac{(x^2 - 7)^5}{\cos^7(x^2 - 5)}\right) \\
                       &= 5 \cdot \ln (x^2 - 7) + 7 \cdot \ln(\cos(x^2-5))
            \end{align*}
            % \begin{align*}
            %   h(x) &= e^{\ln(h(x))} = e^{\ln\left((x^2 - 7)^5/\cos^7(x^2 - 5)\right)}\\
            %   &= e^{\ln((x^2-7)^5) - \ln(\cos^7(x^2-5))} \\
            %   &= e^{5\cdot \ln((x^2-7)) - 7 \cdot \ln(\cos(x^2-5))}
            % \end{align*}

            Derivative of $h$:
            \begin{align*}
              \ddx \ln h(x) &\implies \frac{h'(x)}{h(x)} = 5 \cdot \frac{1}{x^2 - 7} \cdot 2x + 7 \cdot\frac{1}{\cos(x^2 - 5)} \cdot - \sin^2(x^2-5) \cdot 2x\\
                            &= \frac{10x}{x^2-7} - \frac{14x \cdot \sin(x^2 - 5)}{\cos(x^2 - 5)}\\
              &= \frac{10x}{x^2-7} - 14x \tan(x^2-5) \\
              &\implies h'(x) = h(x)\cdot \left( \frac{10x}{x^2-7} - 14x \tan(x^2-5) \right) \\
              &= h'(x) = \frac{(x^2 - 7)^5}{\cos^7(x^2 - 5)} \cdot \left( \frac{10x}{x^2-7} - 14x \tan(x^2-5) \right)
              % h(x) &= e^{5\cdot \ln((x^2-7)) - 7 \cdot \ln(\cos^7(x^2-5))}\\
              % &\implies h'(x) = e^{5\cdot \ln((x^2-7)) - 7 \cdot \ln(\cos(x^2-5))}\cdot \left(5\cdot \frac{1}{x^2-7} \cdot 2x - 7 \cdot -\sin(x^2-5)\cdot 2x \right) \\
              % &\implies h'(x) = h(x) \cdot \left( \frac{10x-35}{x^2-7} + 14x\sin(x^2-5)\right)
            \end{align*}
	\end{freeResponse}
	\end{enumerate}
\end{problem}


\begin{problem}
  \outcome{Find derivatives of inverse functions in general.}
  \outcome{Understand how the derivative of an inverse function relates to the original derivative.}
  A table of values for $f$ and $f'$ is shown below.
  Suppose that $f$ is a one-to-one function and $f^{-1}$ is its inverse.
  \begin{center}
    \begin{tabular}{ccc}
      \toprule
      $x$ & $f(x)$ & $f'(x)$\\
      \midrule
      1 & 3 & 4\\
      3 & 4 & 5\\
      4 & 6 & 3\\
      \bottomrule
    \end{tabular}
  \end{center}

  \begin{itemize}
    \item[(I)] Evaluate $f^{-1}(f(x))$ at $x = 3$.
      \begin{freeResponse}
        (b) is the correct answer:
        \begin{align*}
          f^{-1}(f(3)) &= f^{-1}(4) \\
          &= 3
        \end{align*}
      \end{freeResponse}
      \begin{itemize}
        \item[(a)] 1
        \item[(b)] 3
        \item[(c)] 6
        \item[(d)] 4
        \item[(e)] DNE
        \item[(f)] None of the previous answers
      \end{itemize}

    \item[(II)] Evaluate $\ddx f(f(x))$ at $x = 3$.
      \begin{freeResponse}
        (d) is correct answer:
        \begin{align*}
          \ddx f(f(x)) = f'(f(x)) \cdot f'(x) &\implies \eval{\ddx}_{x = 3} f(f(x)) = f'(f(3)) \cdot f'(3)\\
          &= f'(4) \cdot 5 = 3 \cdot 5 = 15
        \end{align*}
      \end{freeResponse}
      \begin{itemize}
        \item[(a)] 6
        \item[(b)] 25
        \item[(c)] 5
        \item[(d)] 15
        \item[(e)] DNE
        \item[(f)] None of the previous answers
      \end{itemize}

    \item[(III)] Evaluate $\ddx\ln((f(x))$ at $x = 3$.
      \begin{freeResponse}
        (c) is the correct answer:
        \begin{align*}
          \ddx\ln((f(x)) = \frac{f'(x)}{f(x)} \\
          &\implies \eval{\ddx \ln(f(x))}_{x = 3} = \frac{f'(3)}{f(3)} = \frac{5}{4}
        \end{align*}
      \end{freeResponse}
      \begin{itemize}
        \item[(a)] $1/4$
        \item[(b)] 5
        \item[(c)] $5/4$
        \item[(d)] $1/5$
        \item[(e)] DNE
        \item[(f)] None of the previous answers
      \end{itemize}

    \item[(IV)] Evaluate $f^{-1}(x)$ at $x = 3$.
      \begin{freeResponse}
        (b) is the correct answer:
        \begin{align*}
          f^{-1}(3) = 1 \iff 3 = f(1)
        \end{align*}
      \end{freeResponse}
      \begin{itemize}
        \item[(a)] 4
        \item[(b)] 1
        \item[(c)] $1/3$
        \item[(d)] 5
        \item[(e)] DNE
        \item[(f)] None of the previous answers
      \end{itemize}

    \item[(V)] Evaluate $\ddx f^{-1}(x)$ at $x = 3$.
      \begin{freeResponse}
        (d) is the correct answer:
        \begin{align*}
          \ddx f^{-1}(x) &= \frac{1}{f'(f^{-1}(x))} \\
          &\implies \eval{\ddx f^{-1}(x)}_{x = 3} = \frac{1}{f'(f^{-1}(3))} \\
          &= \frac{1}{f'(1)} = \frac{1}{4}
        \end{align*}
      \end{freeResponse}
      \begin{itemize}
        \item[(a)] 1
        \item[(b)] 4
        \item[(c)] $1/5$
        \item[(d)] $1/4$
        \item[(e)] 5
        \item[(f)] None of the previous answers
      \end{itemize}
  \end{itemize}
\end{problem}

%problem 1
\begin{problem}
  \outcome{Take derivatives which involve inverse trig functions.}
Find the derivatives of the following functions:
	\begin{enumerate}
	
	%part a
	\item  $f(x) = \sec^{-1} (\sqrt{x})$.
		\begin{freeResponse}
		$f'(x) = \frac{1}{\sqrt{x} \sqrt{x - 1}} \cdot \frac{1}{2} x^{-\frac{1}{2}} = \frac{1}{2x\sqrt{x-1}}$
		\end{freeResponse}
		
		
		
	%part b
	\item  $g(x) = \ln (\sin^{-1}(x))$.
		\begin{freeResponse}
		$g'(x) = \frac{1}{\sin^{-1}(x)} \cdot \frac{1}{\sqrt{1-x^2}}$.
		\end{freeResponse}
		
		
		
	%part c
	\item  $h(x) = \frac{1}{\tan^{-1}(x^2 + 4)}$.  
		\begin{freeResponse}
		$h'(x) = - \left( \tan^{-1}(x^2 + 4) \right)^{-2} \cdot \frac{1}{1 + (x^2 + 4)^2} \cdot (2x)$.
		\end{freeResponse}
	\end{enumerate}
\end{problem}

\begin{problem}
Find the derivative of $f^{-1}$ at the following points without solving for $f^{-1}$.
	\begin{enumerate}
	
	\item  $f(x) = x^2 + 1$ (for $x \geq 0$) at the point $(5,2)$.  
		\begin{freeResponse}
		$(f^{-1})'(5) = \frac{1}{f'(2)}$.  Since $f'(x) = 2x$, $f'(2) = 4$.  Thus, $(f^{-1})'(5) = \frac{1}{4}$.  
		\end{freeResponse}

	\item  $f(x) = x^2 - 2x - 3$ (for $x \leq 1$) at the point $(12, -3)$.  
		\begin{freeResponse}
		$(f^{-1})'(12) = \frac{1}{f'(-3)}$.  Since $f'(x) = 2x - 2$, $f'(-3) = -6 - 2 = -8$.  Thus, $(f^{-1})'(12) = - \frac{1}{8}$. 
		\end{freeResponse}
	\end{enumerate}
\end{problem}

\section{Extra Problems for Personal Practice}
Explain what each of the following means:
\begin{problem}[warmup]
  \outcome{Recall the meaning and properties of inverse trig functions.}
  \mbox{}
  \begin{enumerate}
	
    % part a
  \item $\sin^{-1}(x)$
    \begin{freeResponse}
      This denotes the inverse function to $sin(x)$, sometimes denoted
      by $\arcsin(x)$.
    \end{freeResponse}
		
    % part b
  \item $\left( \sin(x) \right)^{-1}$
    \begin{freeResponse}
      This means $\sin(x)$ raised to the $-1$ power,
      i.e. $\frac{1}{\sin(x)}$.
    \end{freeResponse}
		
    % part c
  \item $\sin \left(x^{-1} \right)$
    \begin{freeResponse}
      This means $\sin \left( \frac{1}{x} \right)$.
    \end{freeResponse}
		
    % part d
  \item $f^{-1}(x)$
    \begin{freeResponse}
      This denotes the inverse function of $f(x)$.
    \end{freeResponse}
		
    % part e
  \item $f(x^{-1})$
    \begin{freeResponse}
      This means $f \left( \frac{1}{x} \right)$.
    \end{freeResponse}
		
    % part f
  \item $\left( f(x) \right)^{-1}$
    \begin{freeResponse}
      This means $f(x)$ raised to the $-1$ power,
      i.e. $\frac{1}{f(x)}$.
    \end{freeResponse}
  \end{enumerate}
\end{problem}

%problem 3
\begin{problem}
  \outcome{Find derivatives of inverse functions in general.}
  \outcome{Understand how the derivative of an inverse function relates to the original derivative.}
Suppose that $f(x)$ is a differentiable function which is one-to-one.  Given the table of values below, find the value of $(f^{-1})'(7)$.  

\begin{tabular}{|c|c|c|c|}
\hline
$x$	&	1	&	7	&	11	\\
\hline
$f(x)$	&	7	&	11	&	1	\\
\hline
$f'(x)$	&	61	&	-17	&	71	\\
\hline
\end{tabular}

		\begin{freeResponse}
		$(f^{-1})'(7) = \frac{1}{f'(f^{-1}(7))}.$  Since $f(1) = 7$, $f^{-1}(7) = 1$.  Thus 
		$$(f^{-1})'(7) = \frac{1}{f'(1)} = \frac{1}{61}.$$
		\end{freeResponse}
\end{problem}

\begin{problem}
  \outcome{Find derivatives of inverse functions in general.}
  \outcome{Understand how the derivative of an inverse function relates to the original derivative.}
  Find the slope of the tangent line to the curve $y = f^{-1}(x)$ at $(4,7)$ if the slope of the tangent line to the curve $y=f(x)$ at $(7,4)$ is $\frac{2}{3}$.
  \begin{freeResponse}
    Note that the statement ``the slope of the tangent line to the curve $y=f(x)$ at $(7,4)$ is $\frac{2}{3}$" specifically means that $f'(7) = \frac{2}{3}$.
    The slope of the tangent line to the curve $y = f^{-1}(x)$ at $(4,7)$ is $(f^{-1})'(4)$, and so we use the formula for the derivative of the inverse function to compute: $(f^{-1})'(4) = \frac{1}{f'(7)} = \frac{1}{\frac{2}{3}} = \frac{3}{2}$.
  \end{freeResponse}
\end{problem}
\end{document} 
