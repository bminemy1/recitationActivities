\documentclass{article}

\usepackage{kpfonts}
\usepackage{amsthm, booktabs, epigraph, graphicx, mathtools, amssymb}
\usepackage{changepage}
\usepackage[framemethod = tikz]{mdframed} 
\usepackage[onehalfspacing]{setspace}

\usepackage[%
  backend = biber,%
  style = numeric,%
  natbib = true,%,
  url = false,%
  doi = false,%
  eprint = false,%
  backref=true,%
]{biblatex}

\usepackage{hyperref}

\definecolor{Green}{RGB}{82, 142, 0}
\hypersetup{%
  citecolor = Green,
  colorlinks,
  linkcolor = Green,
  urlcolor = Green%
}

\newtheoremstyle{plain}
  {3mm}                         % Space above theorem and previous line.
  {3mm}                         % Space below theorem box and next line.
  {\slshape}                    % Use slanted font in body of theorem.
  {}                            % Indent amount from margin. (Here no
                                % indent.)
  {\bfseries}                   % Theorem head font.
  {.}                           % Punctation after theorem head.
  {.5em}                        % Space after theorem head.
  {}                            % Theorem head specification. 

\newtheoremstyle{definition}
  {3mm}                         % Space above theorem and previous line.
  {3mm}                         % Space below theorem box and next line.
  {}                            % Use slanted font in body of theorem.
  {}                            % Indent amount from margin. (Here no
                                % indent.)
  {\bfseries}                   % Theorem head font.
  {.}                           % Punctation after theorem head.
  {.5em}                        % Space after theorem head.
  {}                            % Theorem head specification

\newtheorem*{Theme}{Themes}
\newtheorem*{FocusQuestion}{Focus Questions}
\newtheorem*{Keywords}{Important Terminology}
%\newtheorem*{Limit Laws}{Limit Laws}


\theoremstyle{definition}
\newtheorem{Problem}{Problem}


\definecolor{Off-White-Color}{rgb}{0.98, 0.99, 0.93}
\definecolor{Scarlett-Color}{rgb}{0.73, 0, 0}

\definecolor{Grey-Color}{rgb}{0.4, 0.4, 0.4}
\definecolor{Off-Gray-Color}{rgb}{0.99, 0.99, 0.98}
\definecolor{Pale-Yellow}{RGB}{252, 247, 235}


\mdfdefinestyle{Focus Question and Theme Style}{%
  linecolor = Scarlett-Color,%
  linewidth = 2pt,%
  roundcorner = 10pt
}
\newmdenv[style = Focus Question and Theme Style]{Focus Question and Theme}

\mdfdefinestyle{Limit Laws Style}{%
  frametitle = {Limit Laws},%
  linewidth = 2pt,%
  linecolor = Scarlett-Color,%
  roundcorner = 10pt,%
  backgroundcolor = Pale-Yellow%
}
\newmdenv[style = Limit Laws Style]{Limit Laws}

\mdfdefinestyle{Solution Style}{%
  frametitle = {Solution},%
  linewidth = 2pt,%
  linecolor = Scarlett-Color,%
  roundcorner = 10pt,%
  backgroundcolor = Off-White-Color%
}
\newmdenv[style = Solution Style]{Solution}

\mdfdefinestyle{Instructor Notes Style}{%
  frametitle = {Instructor Notes},%
  linewidth = 2pt,%
  linecolor = Grey-Color,%
  roundcorner = 10pt,%
  backgroundcolor = Off-Gray-Color%
}
\newmdenv[style = Instructor Notes Style]{Instructor Notes}


\title{\includegraphics{"Icon Breakout 7".pdf}\\
  Breakout Session 7\\
%  Recitation 7\\
  %Continuity\\
  %Solutions
  Teaching Guide
}
\date{15 September 2015}

\addbibresource{References.bib}
\begin{document}
\maketitle
\begin{Problem}[Warmup]
  \mbox{}
  \begin{itemize}
    \item[(a)]
      Explain why the Intermediate Value Theorem does not guarantee a zero for $f(x) = \frac{x-1}{x^2 - 5x}$ on the interval $(2,6)$, even though $f(2) < 0$ and $f(6) > 0$.
      \begin{Instructor Notes}  
        Some students will be tempted to say ``since $f$ is continuous on every point of its domain'' the intermediate value theorem applies in this case.
        It's important to clarify for students the hypotheses for the intermediate value theorem.
        (It may be useful to draw several pictures illustrating the various ways a function can fail the intermediate value theorem.)
      \end{Instructor Notes}

      \begin{Solution}
        Notice that $f$ is not defined on every point in the interval $[2, 6]$, in particular, $f$ is undefined for $x = 5$.
        Therefore $f$ does not satisfy the conditions for the intermediate value theorem.
        (For the intermediate value theorem to apply, $f$ needs to be continuous on the interval $[2,6]$.)

        To summarize: the graph of $f$ shows that $f$ has a zero outside the interval $[2, 6]$:
        \begin{center}
          \includegraphics[scale = 0.53]{"Graph for warmup problem".png}
        \end{center}
        However since $f$ is not continuous on the interval $[2, 6]$, the intermediate value theorem cannot guarantee that $f(x) = 0$ for some $x$ between 2 and 6.
      \end{Solution}

    \item[(b)]
      True or False: At some time since you were born your weight in pounds exactly equaled your height in inches.
      (From \cite[Question 7]{terrell_goodquestions_2005}.)
      \begin{Instructor Notes}
        This is a good conceptual question from \cite[Question 7]{terrell_goodquestions_2005} that gives the students practicing in applying the intermediate value theorem to a function that is not given to by an explicit formula.
        For this problem you should again review the necessary conditions for the intermediate value theorem.
        (According to \href{http://www.webmd.boots.com/children/baby/guide/normal-baby-size-at-birth}{WebMD} the average weight of a newborn baby is 7.7 pounds while the average height is 20 inches.)
      \end{Instructor Notes}

      \begin{Solution}
        This is true!

        Let $\mathrm{weight}(t)$ represent your weight in pounds at time $t$ and let $\mathrm{height}(t)$ represent your height in inches at time $t$.
        Each of these are continuous functions of time and so their difference $\mathrm{weight}(t) - \mathrm{height}(t)$ is also continuous.

        If $t = 0$ is the moment you where born and $t = T_0$ is the present time, then $\mathrm{weight}(0) - \mathrm{height}(0) > 0$ and $\mathrm{weight}(T_0) - \mathrm{height}(T_0) < 0$.
        Hence by the intermediate value theorem there is a point in the past where your weight in pounds is equal to your height in inches.
      \end{Solution}
  \end{itemize}
\end{Problem}

\begin{Problem}
  For the following function $g$ defined by
  \[
    g(t) =
    \begin{cases}
      5t + 7 & \mbox{if $t < -3$,}\\
      \displaystyle\frac{(t-1)(t+2)}{t+2} & \mbox{if $-3 \leq t < 1$ and $t \neq -2$, and}\\
      4 \ln(t) & \mbox{if $t \geq 1$,}
    \end{cases}
  \]
  find the intervals where $g$ is continuous.
  (\textbf{Important Note:} Write your answer as a list of intervals with each interval separated by a comma.)
\end{Problem}
\begin{Instructor Notes}
  Be sure to demonstrate proper limit notation and how we use this notation to continuity.
  Proper notation is important since students can easily lose points on the exam by writing their solutions improperly.
  (Some students may have a tendency to drop the limit notation when writing out their limits.)
  Also, mention the difference between listing intervals with \textbf{commas} verses listing intervals with \textbf{unions}.
  So you should carefully go through finding the intervals of continuity of this function.
\end{Instructor Notes}
\begin{Solution}
  The function $g$ is continuous on $(\infty, -3)$ since, in this interval, $g(t)=5t+7$ is a polynomial and therefore continuous on its domain.
  
  Note that
  \[
    \lim_{t \to -3^-} g(t) = \lim_{t \to -3^-} 5t+7 = 5(-3) + 7 = -8
  \]
  and
  \[
    g(-3) = \frac{(-3-1)(-3+2)}{-3+2} = \frac{4}{-1} = -4.
  \]
  Since $g(-3) = -4 \ne -8 = \lim_{t \to -3^-} g(t)$, we have that $g$ fails the continuity checklist at $t = -3$.
  Therefore $g$ is not continuous at $t = -3$.

  On the interval $(-3, 1)$, the function $g$ is defined by a rational function, $g(t) = \frac{(t-1)(t+2)}{t+2}$.
  A rational function is continuous on every point of its domain.
  Since $g(t) = \frac{(t-1)(t+2)}{t+2}$ is undefined for $t = -2$, on the interval $(-3, 1)$ we have that $g$ is continuous on $(-3, -2)$ and $(-2, 1)$.
  Since each of these intervals have endpoints, we must check for one-sided continuity.
  We have the right-sided limit
  \[
    \lim_{t \to -3^+} g(t) = \lim_{t \to -3^+} \frac{(t-1)(t+2)}{t+2} = \frac{(-3-1)(-3+2)}{-3+2} = -4 = g(-3).
  \]
  Since $g$ satisfies the right-continuity checklist at $t = -3$ we have that $g$ is right-continuous at $t = -3$.
  We now consider the left-sided limit
  \[
    \lim_{t \to 1^-} g(x) = \lim_{t \to 1^-} \frac{(x-1)(x+2)}{x+2} = \frac{0}{3} = 0.
  \]
  Since $g(1) = 4 \ln(1) = 0$, we have that $g$ satisfies the left-continuity checklist at $t = 1$.
  To summarize: on the interval $(-3, 1)$ we have that $g$ is continuous on $[-3, -2)$ and $(-2, 1]$.

  Finally, the function $t \mapsto 4 \ln(t)$ is continuous over the set of positive real numbers.
  Therefore $g$ is continuous on the interval $[1, \infty)$ since $g(t) = 4 \ln(t)$.

  To finish we can combine the interval $(-2, 1]$ with the interval $[1, \infty)$ to yield $(-2, \infty)$.
  To conclude: $g$ is continuous on the intervals \[(-\infty, -3), [-3, -2), (-2, \infty)\hspace{0.5em}\fbox{\parbox[c][]{1.5in}{\textbf{Important Note:} You \emph{should not} combine the intervals $(-\infty, -3)$ and $[-3, -2)$!}}\]
\end{Solution}


\begin{Problem}
  Find all real numbers $a$ and $b$ such that the function $h$ defined by
  \[
    h(z) =
    \begin{cases}
      az^2 + 38 &  \mbox{if $z < 3$,}\\
      a + b & \mbox{if $z = 3$, and }\\
      2bz - a &	\mbox{if $z > 3$,}
    \end{cases}
  \]
  is continuous for all real numbers $z$.
\end{Problem}
\begin{Instructor Notes}
  Again,  be sure to demonstrate proper limit notation and how the continuity checklist is used in this problem.
  The algebra may give some students some trouble, and you may need to remind students the usual procedure for solving two equations with two unknowns.
  If you feel mentioning the second item on the continuity checklist will be confusing to students, you can omit it and jump straight to the third checklist item.
\end{Instructor Notes}

\begin{Solution}
  First, notice that for $h$ is \emph{not} continuous at $z = 3$ for several different combinations of values of $a$ and $b$, but it is continuous on the intervals $(-\infty, 3)$ and $(3, \infty)$:
  \begin{center}
    \includegraphics[scale = 0.7]{"Graph for problem 3a".png}
  \end{center}
  \begin{center}
    \includegraphics[scale = 0.7]{"Graph for problem 3b".png}
  \end{center}
  \begin{center}
    \includegraphics[scale = 0.7]{"Graph for problem 3c".png}
  \end{center}

  To make sure $h$ is continuous at $z = 3$ we must satisfy the three requirements on the continuity checklist:
  \begin{itemize}
    \item[$\square$]
      $h$ is defined at $z = 3$,

    \item[$\square$]
      $\lim_{z \to 3} h(z)$ exists, and

    \item[$\square$]  
      $\lim_{z \to 3} h(z) = h(3)$.
  \end{itemize}
  
  \subsubsection*{Continuity Checklist: $\square$ $h$ is defined at $z = 3$}
  Since, by definition of $h$, we have $h(3) = a + b$ and so $h$ is defined at $z = 3$.
  We can then check this item off the checklist:
  \begin{itemize}
    \item[$\text{\rlap{$\checkmark$}}\square$]
      $h$ is defined at $z = 3$,

    \item[$\square$]
      $\lim_{z \to 3} h(z)$ exists, and

    \item[$\square$]  
      $\lim_{z \to 3} h(z) = h(3)$.
  \end{itemize}

  \subsubsection*{Continuity Checklist: $\square$ $\lim_{z \to 3} h(z)$ exists}
  To determine if $\lim_{z \to 3} h(z)$ exists we focus on the right-sided limit and left-sided limit as $z \to 3$.

  \begin{align*}
    \lim_{z \to 3^+} h(z) &= \lim_{z \to 3^+} 2bz - a\\
    &= 2b\cdot 3 - a \hspace{0.5em}\fbox{\parbox[c][]{1.5in}{Since $z \mapsto 2bz-a$ is a polynomial we can apply \cite[Theorem 2.4, page 71]{briggs_calculus_2015}.}} \\
    &= 6b - a.
  \end{align*}

  \begin{align*}
    \lim_{z \to 3^-} h(z) &= \lim_{z \to 3^+} az^2 + 38\\
    &= a\cdot 3^2 +38 \\
    &= 9a + 38.
  \end{align*}

  To satisfy this continuity checklist item we must then have $6b - a = 9a + 38 \implies a = (-1/10) \cdot (-6b + 38)$.
  We can then check this item off the checklist:
  \begin{itemize}
    \item[$\text{\rlap{$\checkmark$}}\square$]
      $h$ is defined at $z = 3$,

    \item[$\text{\rlap{$\checkmark$}}\square$]
      $\lim_{z \to 3} h(z)$ exists, and

    \item[$\square$]  
      $\lim_{z \to 3} h(z) = h(3)$.
  \end{itemize}

  \subsubsection*{Continuity Checklist: $\square$ $\lim_{z \to 3} h(z) = h(3)$}
  To finish we must also have $\lim_{z \to 3} h(z) = h(3)$.
  Again, we consider the right-sided limit and left-sided limit separately:
  \begin{align*}
    \lim_{z \to 3^+} h(z) &= a + b \\
    &\implies 6b - a = a + b \\
    &\implies 5b - 2a = 0\\
    &\implies 20b - 8a = 0 \hspace{0.5em}\fbox{\parbox[c][]{1.5in}{\textbf{Small Algebra Trick:} We multiply both sides by 4 to make solving for $a$ and $b$ easier.}}
  \end{align*}

  \begin{align*}
    \lim_{z \to 3^-} h(z) &= a + b \\
    &\implies 9a + 38 = a + b \\
    &\implies 8a - b  = -38.
  \end{align*}

  We can view
  \begin{align*}
    20b - 8a &= 0 \\
    8a - b  &= -38
  \end{align*}
  as a system of two equations with two unknowns.

  We can add these two equations:
  \begin{align*}
     20b - 8a &= 0 \\
   + (8a - b  &= -38)\\
     \cline{1-2}
    19b &= -38
  \end{align*}
  Therefore $b = -2$

  Using $a = (-1/10) \cdot (-6b + 38)$ and $b = -2$ we have that $a = -5$.

  We can now check the final item off the checklist:
  \begin{itemize}
    \item[$\text{\rlap{$\checkmark$}}\square$]
      $h$ is defined at $z = 3$,

    \item[$\text{\rlap{$\checkmark$}}\square$]
      $\lim_{z \to 3} h(z)$ exists, and

    \item[$\text{\rlap{$\checkmark$}}\square$]  
      $\lim_{z \to 3} h(z) = h(3)$.
  \end{itemize}
  
  Therefore $h$ is continuos at $z = 3$ when $a = -5$ and $b = -2$.

  With these choices for $a$ and $b$ we have that $h$ is continuous on every point of its domain:
  \begin{center}
    \includegraphics[scale = 0.7]{"Graph for problem 3d".png}
  \end{center}
\end{Solution}

\begin{Problem}
  Use the Intermediate Value Theorem to find an interval in which you can guarantee that there is a solution to the equation $x^3 = x + \sin x + 1$.
  Do not use a graphing device or calculator to solve this problem!
\end{Problem}
\begin{Instructor Notes}
  The hard part for students will be finding the appropriate intervals to test the function $f$ on; so, please remind students about the unit circle and the corresponding special angles as natural starting points.

  Also, if there is time, you can mention how a ``bisection type algorithm'' and the intermediate value theorem can be used to approximate a solution to this equation.
\end{Instructor Notes}
\begin{Solution}
  The intermediate value theorem is a result about functions.
  Since we are simply given an equation $x^3 = x + \sin x + 1$ we must transform this into a corresponding result using functions.

  Define the function $f$ by $f(x) = x^3 - x - \sin(x) - 1$.
  Since both $x \mapsto x^3 - x - 1$ and $x \mapsto \sin(x)$ are continuous over the set of all real numbers \cite[Theorems 2.10 and 2.15]{briggs_calculus_2015}, and $f$ is the difference of these two functions, we have $f$ is continuous everywhere \cite[Theorem 2.9]{briggs_calculus_2015}.

  To apply the intermediate value theorem we must find an interval $[a, b]$ with $f(a) < 0$ and $f(b) > 0$.
  (We could also find an interval $[c, d]$ with $f(c) > 0$ and $f(d) < 0$.)

  Using the unit circe we have $f(0) = 0^3 - 0 - 0 - 1 = -1 < 0$ and $f(\pi) = \pi^3 - \pi - \sin(\pi) - 1 = \pi(\pi^2 - 1) - 1 > 3(3^2 - 1) - 1 = 23 > 0$.
  Therefore, by the intermediate value theorem, there exists a number $c \in (0, \pi)$ such that $f(c) = 0$.

  % (\textbf{Potential application of intermediate value theorem:} To actually approximate this value of $c$ we could write a ``bisection algorithm'' that implicitly utilizes intermediate value theorem:
  % \begin{itemize}
  %   \item[(Step 1)]
  %     Divide interval $[0, \pi]$ into two intervals $[0, \pi/2]$ and $[\pi/2, \pi]$.

  %   \item[(Step 2)]
  %     Test sign of $f(0)$ and $f(\pi/2)$ and $f(\pi/2)$ and $f(\pi)$. 
  % \end{itemize}
  \end{Solution}
\printbibliography
\end{document}