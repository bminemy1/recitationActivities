\documentclass[handout,nooutcomes]{ximera}
%% handout
%% space
%% newpage
%% numbers
%% nooutcomes


\newcommand{\RR}{\mathbb R}
\renewcommand{\d}{\,d}
\newcommand{\dd}[2][]{\frac{d #1}{d #2}}
\renewcommand{\l}{\ell}
\newcommand{\ddx}{\frac{d}{dx}}
\newcommand{\dfn}{\textbf}
\newcommand{\eval}[1]{\bigg[ #1 \bigg]}

\renewenvironment{freeResponse}{
\ifhandout\setbox0\vbox\bgroup\else
\begin{trivlist}\item[\hskip \labelsep\bfseries Solution:\hspace{2ex}]
\fi}
{\ifhandout\egroup\else
\end{trivlist}
\fi} %% we can turn off input when making a master document

\title{Recitation \#6 - 2.6 Continuity (Teaching Guide)}  

\begin{document}
\begin{abstract}		\end{abstract}
\maketitle

Here is a suggested structure for this recitation

\section*{Warm up:} 
	
	\begin{itemize}
	
	\item  \emph{5 minutes}:  Ask students to think about the Warm-up as they are waiting for class to begin.  Then discuss the Warm-Up as a class when class begins. Help clarify for students the necessary hypotheses for the IVT.
	
	\end{itemize}


\section*{Problem 1:}

	\begin{itemize}
	
	\item  \emph{5 minutes}:  Allow students to work on problem 1 in groups.  
	
	\item  \emph{10 minutes}:  Discuss problem 1 as a class, asking for input from the students.  Be sure to demonstrate proper limit notation and clearly show of the criteria for continuity, as we would expect on an exam.  Also, mention the difference between listing intervals with \dfn{commas} verse listing intervals with \dfn{unions}.
		
	\end{itemize}
	
	
	
\section*{Problem 2:}

	\begin{itemize}
	
	\item  \emph{10 minutes}:  Allow students to work on problem 2 in groups. 
	
	\item  \emph{10 minutes}:  Talk about \#2 as a class.  Ask for input from the students in the groups who started with that problem.  Be sure to demonstrate proper limit notation and clearly show of the criteria for continuity, as we would expect on an exam.
	
	\end{itemize}
	
	
	
\section*{Problem 3:}

	\begin{itemize}
	
	\item  \emph{10 minutes}:  Allow students to work on problem 3 in groups.
	
	\item  \emph{5 minutes}:  Let a group present their solution to problem 3, or discuss as a class if the students are stuck.
	
	\end{itemize}
	
	
	

	
	
	
















\end{document}