\documentclass[nooutcomes]{ximera}
%% handout
%% space
%% newpage
%% numbers
%% nooutcomes


\renewcommand{\outcome}[1]{\marginpar{\null\vspace{2ex}\scriptsize\framebox{\parbox{0.75in}{\begin{raggedright}\textbf{P\arabic{problem} Outcome:} #1\end{raggedright}}}}}

\renewenvironment{freeResponse}{
\ifhandout\setbox0\vbox\bgroup\else
\begin{trivlist}\item[\hskip \labelsep\bfseries Solution:\hspace{2ex}]
\fi}
{\ifhandout\egroup\else
\end{trivlist}
\fi}

\newcommand{\RR}{\mathbb R}
\renewcommand{\d}{\,d}
\newcommand{\dd}[2][]{\frac{d #1}{d #2}}
\renewcommand{\l}{\ell}
\newcommand{\ddx}{\frac{d}{dx}}
\everymath{\displaystyle}
\newcommand{\dfn}{\textbf}
\newcommand{\eval}[1]{\bigg[ #1 \bigg]}

\title{Breakout Session 6 Teaching Guide}  

\begin{document}
\begin{abstract}
 \textbf{Theme of calculus:} Calculus is the application of  \href{https://en.wikipedia.org/wiki/Derivative}{rates of change} and \href{https://en.wikipedia.org/wiki/Integral}{accumulation} to understand \href{https://en.wikipedia.org/wiki/Elementary_function}{famous functions} in their application to both real world and mathematical processes.

  \textbf{Goal of Math 1151 course:} Promote and cultivate an environment which improves students' ability to construct, organize, and demonstrate their knowledge of calculus.
\end{abstract}
\maketitle

\section{Notes for problem 1}
For part (a), some students will be tempted to say ``since $f$ is continuous on every point of its domain'' the intermediate value theorem applies in this case.
It's important to clarify for students the hypotheses for the intermediate value theorem.
(It may be useful to draw several pictures illustrating the various ways a function can fail the intermediate value theorem.)

Part (b) is a good conceptual question from \href{http://www.math.cornell.edu/~GoodQuestions/}{Cornell's Good Questions} that gives the students practice in applying the intermediate value theorem to a function that is not given to by an explicit formula.
For this problem you should again review the necessary conditions for the intermediate value theorem.
(According to \href{http://www.webmd.boots.com/children/baby/guide/normal-baby-size-at-birth}{WebMD} the average weight of a newborn baby is 7.7 pounds while the average height is 20 inches.)

\section{Notes for problem 2}
The hard part for students will be finding the appropriate intervals to test the function $f$ on; so, please remind students about the unit circle and the corresponding special angles as natural starting points.

Also, if there is time, you can mention how a ``bisection type algorithm'' and the intermediate value theorem can be used to approximate a solution to this equation.


\section{Notes for problems 3, 4, 5, and 6}
These problems are from the Autumn 2015 Math 1151 Exam 1, and will be tricky for students: it's easy to guess the right answer and hard to construct counterexamples.
Emphasize that these problems are meant to test their knowledge and implication of the definitions of continuity and limits.
\end{document} 
