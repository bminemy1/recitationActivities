\documentclass[handout,nooutcomes]{ximera}
\usepackage{booktabs}
%% handout
%% space
%% newpage
%% numbers
%% nooutcomes

\renewcommand{\outcome}[1]{\marginpar{\null\vspace{2ex}\scriptsize\framebox{\parbox{0.75in}{\begin{raggedright}P\arabic{problem} Outcome: #1\end{raggedright}}}}}

\renewenvironment{freeResponse}{
\ifhandout\setbox0\vbox\bgroup\else
\begin{trivlist}\item[\hskip \labelsep\bfseries Solution:\hspace{2ex}]
\fi}
{\ifhandout\egroup\else
\end{trivlist}
\fi}

\newcommand{\RR}{\mathbb R}
\renewcommand{\d}{\,d}
\newcommand{\dd}[2][]{\frac{d #1}{d #2}}
\renewcommand{\l}{\ell}
\newcommand{\ddx}{\frac{d}{dx}}
\everymath{\displaystyle}
\newcommand{\dfn}{\textbf}
\newcommand{\eval}[1]{\bigg[ #1 \bigg]}


\title{Breakout Session 6: Continuity}  

\begin{document}
\begin{abstract}
  \textbf{A look back:} In the previous (January 26, 2016) Breakout Session you practiced how to interpert and compute infinite limits and limits at infinity.
  Infinite limits are how we defined vertical asymptotes and limits at infinity are how we defined horizontal asymptotes.

  \textbf{Overview:} In today's (January 28, 2016) Breakout Session you will practice working with continuous functions.
  Similar to asymptotes, continuity is defined in terms of limits and gives us information on the behavior of a graph of a function.

  \textbf{A look ahead:} In the next (February 2, 2016) Breakout Session you will resume the study of rates of change and review for Exam 1.
\end{abstract}
\maketitle

\section{Learning Outcomes}
\label{section:learning-outcomes}
The following outcomes are \emph{not an exhaustive} list of the skills you will need to develop and integrate for demonstration on quizzes and exams.
This list is meant to be a starting point for conversation (with your Lecturer, Breakout Session Instructor, and fellow learners) for organizing your knowledge and monitoring the development of your skills.
\begin{itemize}
  \item 
    Find where a function is and is not continuous.

  \item 
    Understand what it means for a function to be continuous at a point.

  \item
    Understand the connection between continuity of a function at a point and the value of a limit.

  \item
    Compute limits using continuity.

  \item
    Determine the intervals of continuity.

  \item
    State the Intermediate value theorem.

  \item 
    Apply the Intermediate value theorem.
\end{itemize}

\newpage

\begin{problem}
  \label{problem:application-of-ivt}
  \mbox{}
  \begin{itemize}
    \item[(a)]
      Explain why the Intermediate Value Theorem does not guarantee a zero for $f(x) = \frac{x-1}{x^2 - 5x}$ on the interval $(2,6)$, even though $f(2) < 0$ and $f(6) > 0$.

    \item[(b)]
      True or False: At some time since you were born your weight in pounds exactly equaled your height in inches.
  \end{itemize}
\end{problem}

\begin{problem}
  \label{problem:finding-a-root}
  Use the Intermediate Value Theorem to find an interval in which you can guarantee that there is a solution to the equation $x^3 = x + \sin x + 1$.
  Do not use a graphing device or calculator to solve this problem!
\end{problem}

\begin{problem}
  \label{problem:can-we-apply-limit-laws}
  True or False: If $f$ and $g$ are two functions defined on $(-1, 1)$, and if $\displaystyle \lim_{x \to 0} g(x) = 0$, then it must be true that $\displaystyle \lim_{x \to 0} (f(x) \cdot g(x)) = 0$.
\end{problem}

\begin{problem}
  \label{problem:cont-and-limit-laws}
  True or False: If $f$ is continuous on $(-1, 1)$, and if $f(0) = 10$ and $\displaystyle \lim_{x \to 0} g(x) = 2$, then
  \[
    \lim_{x \to 0} \frac{f(x)}{g(x)} = 5.
  \]
\end{problem}

\begin{problem}
  \label{problem:cont-and-limit-laws}
  True or False: If $f$ is continuous on $[1, 3]$, and if $f(1) = 0$ and $f(3) = 4$, then the equation $f(x) = \pi$ has a solution in $(1, 3)$.
\end{problem}

\begin{problem}
  \label{problem:vertical-asymptote}
  True or False: Let $f$ be a positive function with vertical asymptote $x = 5$. Then
  \[
    \lim_{x \to 5} f(x) = \infty.
  \]
\end{problem}

\section{Extra Problems for Personal Practice}
\label{section:extra-problems}
\begin{problem}
  \label{problem:finding-intervals-of-continuity}
  For the following function $g$ defined by
  \[
    g(t) =
    \begin{cases}
      5t + 7 & \mbox{if $t < -3$,}\\
      \displaystyle\frac{(t-1)(t+2)}{t+2} & \mbox{if $-3 \leq t < 1$ and $t \neq -2$, and}\\
      4 \ln(t) & \mbox{if $t \geq 1$,}
    \end{cases}
  \]
  find the intervals where $g$ is continuous.
  (\textbf{Important Note:} Write your answer as a list of intervals with each interval separated by a comma.)
\end{problem}

\begin{problem}
  \label{problem:makeing-a-function-cont}
  Find all real numbers $a$ and $b$ such that the function $h$ defined by
  \[
    h(z) =
    \begin{cases}
      az^2 + 38 &  \mbox{if $z < 3$,}\\
      a + b & \mbox{if $z = 3$, and }\\
      2bz - a &	\mbox{if $z > 3$,}
    \end{cases}
  \]
  is continuous for all real numbers $z$.
\end{problem}
\end{document} 
