\documentclass[nooutcomes]{ximera}

%% Options for ximera class: handout, space, newpage, numbers, nooutcomes

%\usepackage{graphicx}

\renewcommand{\outcome}[1]{\marginpar{\null\vspace{2ex}\scriptsize\framebox{\parbox{1in}{\begin{raggedright}\textbf{P\arabic{problem} Outcome:} #1\end{raggedright}}}}}

\renewenvironment{freeResponse}{
\ifhandout\setbox0\vbox\bgroup\else
\begin{trivlist}\item[\hskip \labelsep\bfseries Solution:\hspace{2ex}]
\fi}
{\ifhandout\egroup\else
\end{trivlist}
\fi}

\newcommand{\RR}{\mathbb R}
\renewcommand{\d}{\,d}
\newcommand{\dd}[2][]{\frac{d #1}{d #2}}
\renewcommand{\l}{\ell}
\newcommand{\ddx}{\frac{d}{dx}}
\everymath{\displaystyle}
\newcommand{\dfn}{\textbf}
\newcommand{\eval}[1]{\bigg[ #1 \bigg]}


\title{Breakout Session 1 Teaching Guide}  

\begin{document}
\begin{abstract}
  \textbf{Theme of calculus:} Calculus is the application of  \href{https://en.wikipedia.org/wiki/Derivative}{rates of change} and \href{https://en.wikipedia.org/wiki/Integral}{accumulation} to understand \href{https://en.wikipedia.org/wiki/Elementary_function}{famous functions} in their application to both real world and mathematical processes.

  \textbf{Goal of Math 1151 course:} Promote and cultivate an environment which improves students' ability to construct, organize, and demonstrate their knowledge of calculus.
\end{abstract}
\maketitle

\section{Teaching notes for problems}

\subsection*{Notes for problem 1}
The point of this problem is to give students practice in reading off from the graph the domain and range, evaluation of particular function values, and determining if a function is invertible.
Generally, students will have trouble, throughout the semester, with interpreting information from graphs.
(One (main) reason for students' difficulties with graphs is that interpreting information presented as a graph requires students to integrate a several component skills to be successfully at interpretation.
This is also one reason why producing a good visualization is hard!)

For part (a) students will have difficulties determining whether to include the point at $x = 0$ and the correct way of writing interval notation and union.
Break this problem into two steps: first write $(-\infty, 0) \cup \{0\} \cup (0, 1) \cup (1, \infty)$ by reading off the graph and then condense to $(-\infty, 1) \cup (1, \infty)$.
You should also quickly ask the students ``Why don't we include 1 in the domain?''.

Part (b) is similar to part (a): you should read the intervals off the graph and then condense and ask ``Why don't we include $-1$ or $0$ in the range?''.

For part (c) students will struggle with evaluating $f(0)$ because they don't know which value to pick.
From their point-of-view there may be three valid possibilities due to the circles.
Similarly, evaluating $f(1)$ will also be difficult: students may think there are two choices when really there are none.
Students may also be uncomfortable saying something is undefined.
It's important to note that $f(1)$ is undefined because 1 is not an element in the domain of $f$, as written in part (a).

For part (c) students will struggle with evaluating $f(0)$ because they don't know which value to pick.
From their point-of-view there may be three valid possibilities due to the circles.
Similarly, evaluating $f(1)$ will also be difficult: students may think there are two choices when really there are none.
Students may also be uncomfortable saying something is undefined.
It's important to note that $f(1)$ is undefined because 1 is not an element in the domain of $f$, as written in part (a).

For part (d), encourage students to try the horizontal line test.
Ask them to identify all those values of $y$ for which the horizontal line test fails.
Students may be tempted to say that the function fails the horizontal line test at $y = 0$.
This is incorrect and it should be explained why this choice is incorrect by relating it to the fact that 0 is not in the range of $f$.

Of course for part (e), there will be many valid answers.
The solutions only give two of them.

For part (f), remind the students that $f^{-1}(x) = y \iff x = f(y)$.
This relationship is important one for inverse function.
(In fact, I say stress this relationship, along with its graphically interpretation, every time you work through an inverse function problem.)

\subsection*{Notes for problem 2}
Simple true/false problem designed to test if students incorrectly assume $g^{-1} = 1/g$.
This is problem is easy for students to get wrong: the standard notation for an inverse function is suggestive but not consistent. 

\subsection*{Notes for problem 3}
Many students will be tempted to say this statement is ``true'', since $f^{-1}(f(x)) = x$.
It's important to emphasize that trigonometric functions are \emph{not} invertible, we must restrict their domains to make them one-to-one.
At this point it is probably a good idea to review the unit circle to help students evaluate $\cos(7\pi/6)$ and draw a picture of the restricted cosine function.
(If you review the unit circle, you should leave it on the board for the next problem.)

Also emphasize that $\mathrm{Domain}(f) = \mathrm{Range}(f^{-1})$ and $\mathrm{Range}(f) = \mathrm{Domain}(f^{-1})$.

\subsection*{Notes for problem 4}
Another problem to test if students are familiar with the domain and range of inverse trigonometric functions.
I recommend referring to the unit circle, from the previous problem, and draw the restricted sine function.

\subsection*{Notes for problem 5}
The main point of this problem is to review the fact that trigonometric functions are periodic and, using this fact, to solve trigonometric equations.
\begin{itemize}
  \item[(a)]
    Students will remember (or use the unit circle) to conclude $2\pi$ and 0 are both solutions, but they will forgot about multiplying by $n$.
    Review the graph of cosine along with the unit circle.
    It's important to mention that cosine gives the $x$-coordinate of the unit circle.
    (Some students may confuse which coordinate refers to which function.)

  \item[(b)]
    Again, students will forget about the period of sine and the need to consider integer multiples of $2\pi$.
    Also, they will struggle with the variable substitution of $x=3\theta$ and how this substitution effects the inequality $0 \le \theta \le 2\pi$.

    It may be better to start by first ignoring the inequality and solve the equation just using the periodicity of sine.
    
    Then use the inequality and your variable substitution $x=3\theta$ to find the solutions for this problem.
    Emphasize that half of the numbers in the collection $\{ \pi/9, 2\pi/9, 7\pi/9, 8\pi/9, 13\pi/9, 14\pi/9 \}$ comes from the equation $\theta = \pi/9 + 2\pi/3n$ while the other half comes from $\theta = 2\pi/9 + 2\pi/3n$.
\end{itemize}

\subsection*{Notes for problem 6}
The point of this problem is to review inverse trigonometric functions and function composition in a concrete case.
\begin{itemize}
  \item[(a)]
    This part should be straightforward.
    Recall the definition of a composition of functions (emphasizing we go from the inside function to the outside function) and the defining property of the inverse of a function.

    As a quick verbal quiz, you may ask students to evaluate $\cos^{-1}(\cos(2\pi)$.
    (The answer is $\cos^{-1}(\cos(2\pi) = 0$.
    This quick verbal quiz would also provide the opportunity to review the domain of $\cos^{-1}$ and make a connection with Problem~\ref{problem:warmup}.)

  \item[(b)]
    Here students may struggle with the geometric interpretation of the inverse sine as sides of a triangle.
    Again emphasize that in a composition we start with the inside function to the outside function.
\end{itemize}

\end{document} 
