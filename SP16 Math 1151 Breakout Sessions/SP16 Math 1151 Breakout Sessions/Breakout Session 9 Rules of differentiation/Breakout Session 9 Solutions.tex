\documentclass[nooutcomes]{ximera}
\usepackage{booktabs}
%% handout
%% space
%% newpage
%% numbers
%% nooutcomes

\renewcommand{\outcome}[1]{\marginpar{\null\vspace{2ex}\scriptsize\framebox{\parbox{0.75in}{\begin{raggedright}\textbf{P\arabic{problem} Outcome}: #1\end{raggedright}}}}}

\renewenvironment{freeResponse}{
\ifhandout\setbox0\vbox\bgroup\else
\begin{trivlist}\item[\hskip \labelsep\bfseries Solution:\hspace{2ex}]
\fi}
{\ifhandout\egroup\else
\end{trivlist}
\fi}

\newcommand{\RR}{\mathbb R}
\renewcommand{\d}{\,d}
\newcommand{\dd}[2][]{\frac{d #1}{d #2}}
\renewcommand{\l}{\ell}
\newcommand{\ddx}{\frac{d}{dx}}
\everymath{\displaystyle}
\newcommand{\dfn}{\textbf}
\newcommand{\eval}[1]{\bigg[ #1 \bigg]}


\title{Breakout Session 9 Solutions}  

\begin{document}
\begin{abstract}
  % \textbf{A look back:} In the previous (February 2, 2016) Breakout Session you were introduced to the concept of the derivative of a function and the connection between derivatives and slopes of tangent lines.

  % \textbf{Overview:} In today's (February 9, 2016) Breakout Session you'll practice the ``shortcut'' rules for computing derivatives.
  % Computing derivatives is the least important part of this course, but \emph{it is important} that you're able to quickly and accurately compute derivatives of various types of functions.

  % \textbf{A look ahead:} In the next (February 11, 2016) Breakout Session you will practice rules of differentiation.
  % Becoming comfortable and proficient with computing derivatives is an important skill to practice before we return our focus on the conceptual aspects (and interpretation) of derivatives. 
\end{abstract}
\maketitle

\section{Learning Outcomes}
% \label{section:learning-outcomes}
% The following outcomes are \emph{not an exhaustive} list of the skills you will need to develop and integrate for demonstration on quizzes and exams.
% This list is meant to be a starting point for conversation (with your Lecturer, Breakout Session Instructor, and fellow learners) for organizing your knowledge and monitoring the development of your skills.

% \subsection*{Basic rules of differentiation}
% \begin{itemize}
%   \item
%     Use ``shortcut'' rules to find and use derivatives.

%   \item 
%     Use the definition of the derivative to develop shortcut rules to find the derivatives of constants, constant multiples, powers of x, sums of functions, and the natural exponential function.

%   \item
%     Define higher order derivatives.

%   \item
%     Calculate higher order derivatives.
% \end{itemize}

% \subsection*{Product and quotient rules}
% \begin{itemize}
%   \item
%     Use the product rule to calculate derivatives. 

%   \item 
%     Use the quotient rule to calculate derivatives. 

%   \item 
%     Identify products of functions.

%   \item
%     Identify quotients of functions.
% \end{itemize}
% \newpage

\begin{problem}
  \outcome{Use ``shortcut'' rules to find and use derivatives.}
  \outcome{Use the quotient rule to calculate derivatives.}
  \outcome{Use the product rule to calculate derivatives.}
  \outcome{Identify quotients of functions.}
  Differentiate the function $f$ defined by $f(x) = 1/x^8$ in two different ways.
  \begin{freeResponse}
    Applying the quotient rule:
    \begin{align*}
      f'(x) &= \frac{(1)' \cdot x^8 - (1 \cdot (x^8)')}{(x^8)^2} \\
            &= \frac{(0 \cdot x^8) - (1 \cdot 8x^7)}{x^{16}} \\
            &= \frac{-8x^7}{x^{16}}\\
            &= \frac{-8}{x^9}. 
    \end{align*}

  Applying the power rule:
  \begin{align*}
    f(x) &= \frac{1}{x^8} = x^{-8}\\
    &\implies f'(x) = (x^{-8})'\\
    &= -8\cdot x^{-8 -1}\\
    &= -8x^{-9}
  \end{align*}
  \end{freeResponse}
\end{problem}

\begin{problem}
  \outcome{Use ``shortcut'' rules to find and use derivatives.}
  \outcome{Use the quotient rule to calculate derivatives.}
  \outcome{Use the product rule to calculate derivatives.}
  Use the given information to find the equation of the tangent line.
  \begin{itemize}
    \item[(a)]
      Given $g(x) = x^3 f(x)$, $f(2) = 4$, and $f'(2) = 7$, find the equation of the tangent line to the graph of $g$ at $x = 2$.
      \begin{freeResponse}
        Slope of tangent line at $x = 2$:
        \begin{align*}
          g'(x) &= 3x^2 f(x) + x^3 f'(x)\\
          &\implies g'(2) = 12(4) + 8(7) = 48 + 56 = 104.
        \end{align*}

        Point on tangent line: $(2, g(2)) = (2, 8f(2)) = (2, 32)$.

        Equation of tangent line:
        \begin{align*}
          y-32 &= 104(x-2)\\
          &\implies y = 104x - 176.
        \end{align*}
      \end{freeResponse}


    \item[(b)]
      Given $h(z) = \frac{z s(z)}{z-3}$, $s(2) = 4$, and $s'(2) = 7$, find the equation of the tangent line to the graph of $h$ at $z = 2$.
      \begin{freeResponse}
        Slope of tangent line at $z = 2$:
        \begin{align*}
          h'(z) &= \frac{ (z-3)(z s(z))' - z s(z)(z-3)'}{(z-3)^2}  \\
		&= \frac{(z-3)(s(z) + z s'(z)) - z s(z)(1)}{(z-3)^2} \\
                &\implies h'(2) = \frac{(2-3)(s(2) + 2 s'(2)) - 2f(2)}{(2-3)^2} = -26.
        \end{align*}

        Point on tangent line: $(2, h(2)) = (2, \frac{2s(2)}{2-3}) = (2, -8)$.

        Equation of tangent line:
        \begin{align*}
          y-(-8) &= -26(z-2) \\
          &\implies y = -26z + 44.
        \end{align*}
      \end{freeResponse}


    \item[(c)]
      Given
      \begin{center}
        \begin{tabular}{cccccc}
          \toprule
          $x$ & 1 & 2 & 3 & 4 & 5\\
          \midrule
          $f(x)$ & 5 & 3 & 0 &$-4$ & 3\\
          $f'(x)$ & $-3$ & $-5$ & $-2$ & 6& $-4$\\
          $g(x)$ & 6 &9&$-8$&13&15\\
          $g'(x)$ & 8 & 5 & $-10$ &7& 6\\
          \bottomrule
        \end{tabular}
      \end{center}
      find the equation of the tangent line of 
      \[
        \frac{f(x)}{e^xg(x)}
      \]
      at $x = 2$.
      \begin{freeResponse}
        Slope of tangent line:
        \begin{align*}
          \left( \frac{f(x)}{e^x g(x)} \right)'
          &= \frac{e^x g(x) f'(x) - f(x)(e^x g(x) + e^x g'(x))}{(e^x g(x))^2}.  \\
          &\implies \frac{d}{dx}\left(\frac{f(x)}{e^x g(x)}\right) \bigg|_{x=2}
          = \frac{e^2 g(2) f'(2) - f(2)(e^2 g(2) + e^2 g'(2))}{(e^2 g(2))^2}  \\
		&= \frac{e^2 (9)(-5) - (3)(9e^2 + 5e^2)}{(9e^2)^2}  \\
		&= \frac{-45e^2 -42e^2}{81e^4}  \\
		&= \frac{-87e^2}{81e^4}\\
                  &= \frac{-87}{81e^2}
        \end{align*}

        Point on tangent line: $(2, f(2)/(e^2g(2))) = (2, 3/(9e^2)) = (2, 1/(3e^2))$.

        Equation of tangent line:
        \begin{align*}
          y - \frac{1}{3e^2} &= \frac{-87}{81e^2}(x - 2)\\
                             &\implies y = \frac{-87}{81e^2}x  - \frac{49}{27e^2}.
        \end{align*}
      \end{freeResponse}
   \end{itemize}
\end{problem}

\begin{problem}
  \outcome{Use ``shortcut'' rules to find and use derivatives.}
  \outcome{Use the quotient rule to calculate derivatives.}
  \outcome{Use the product rule to calculate derivatives.}
  Suppose that
  \begin{center}
    \begin{tabular}{cccc}
      \toprule
      $f(5) = 7$ & $f'(5) = 8$ & $g(5) = 3$ & $g'(5) = -4$\\
      \bottomrule
    \end{tabular}
  \end{center}
  Use the product and quotient rules to find each of the following derivatives:
  \begin{itemize}
    \item[(a)]
      $(fg)'(5)$
      \begin{freeResponse}
        \begin{align*}
          (fg)'(5) &= (f'(5) \cdot g(5)) + (f(5) \cdot g'(5))  \\
                   &= (8)(3) + (7)(-4)  \\
                   &= 24 - 28 = -4.
	\end{align*}
      \end{freeResponse}


    \item[(b)]
      $\displaystyle \left(\frac{f}{g}\right)'(5)$
      \begin{freeResponse}
	\begin{align*}
		\left( \frac{f}{g} \right)' (5) &= \frac{(g(5) \cdot f'(5)) - (f(5) \cdot g'(5))}{(g(5))^2}  \\
		&= \frac{(3)(8) - (7)(-4)}{3^2}  \\
		&= \frac{24 + 28}{9} = \frac{52}{9}.
		\end{align*}
      \end{freeResponse}


    \item[(c)]
      $\displaystyle \left(\frac{g}{f}\right)'(5)$
      \begin{freeResponse}
                		\begin{align*}
		\left( \frac{g}{f} \right)' (5) &= \frac{(f(5) \cdot g'(5)) - (g(5) \cdot f'(5))}{(f(5))^2}  \\
		&= \frac{(7)(-4) - (3)(8)}{7^2}  \\
		&= \frac{-28 - 24}{49} = - \frac{52}{49}.
		\end{align*}
      \end{freeResponse}


    \item[(d)]
      $(gf)'(5)$
      \begin{freeResponse}
        \begin{align*}
          (gf)'(5) &= (g'(5) \cdot f(5)) + (g(5) \cdot f'(5))  \\
                   &= (-4)(7) + (3)(8)  \\
                   &= -28 + 24 = -4.
	\end{align*}
      \end{freeResponse}
  \end{itemize}
\end{problem}

\begin{problem}
  Differentiate each of the following functions:
  \begin{itemize}
    \item[(a)]
        \outcome{Use ``shortcut'' rules to find and use derivatives.}
  \outcome{Use the quotient rule to calculate derivatives.}
  \outcome{Use the product rule to calculate derivatives.}
      $f(x) = (x^2 + 4x - 7) e^{-5x}$
      \begin{freeResponse}
        \begin{align*}
          f'(x) &= \bigl( (x^2 + 4x - 7) e^{-5x}\bigr)'\\
                &= (x^2 + 4x - 7)' \cdot e^{-5x} + (x^2 + 4x - 7) \cdot (e^{-5x})'\\
                &= (2x + 4) \cdot  e^{-5x} + (x^2 + 4x - 7) \cdot -5e^{-5x} \\
                &= \bigl((2x + 4) + (x^2 + 4x -7)\cdot -5\bigr)\cdot e^{-5x} \\
                &= (2x + 4 - 5x^2 - 20x + 35)\cdot e^{-5x}\\
                &= (-5x^2 -18x + 39) e^{-5x}
        \end{align*}
      \end{freeResponse}


    \item[(b)]
      $\displaystyle g(t) = \frac{t^2 + 4t - 7}{e^{-3t}}$
      \begin{freeResponse}
    \begin{align*}
      g'(t) &= \frac{\left( e^{-3t} \cdot (t^2 + 4t - 7)' \right) - \left( (t^2 + 4t - 7) \cdot (e^{-3t})' \right)}{(e^{-3t})^2}  \\
			&= \frac{((e^{-3t}) \cdot (2t+4)) - ((t^2 + 4t - 7) \cdot (-3e^{-3t}))}{e^{-6t}} \\
			&= \frac{(2t + 4 + 3t^2 + 12t - 21)e^{-3t}}{e^{-6t}}  \\
            &= \frac{3t^2 + 14t - 17}{e^{-3t}}.
    \end{align*}        
      \end{freeResponse}

  \end{itemize}
\end{problem}

\section{Extra problems for personal practice}
\begin{problem}
  \outcome{Use ``shortcut'' rules to find and use derivatives.}
  \outcome{Use the quotient rule to calculate derivatives.}
  For each of the following functions use the ``short-cut derivative rules'' to compute their derivatives.
  \begin{itemize}
    \item[(a)]
      The function $f$, defined by $f(x) = \sqrt{x}$.
      \begin{freeResponse}
       \begin{align*}
          f'(x) &= (\sqrt{x})' = (x^{1/2})' \\
                &= \frac{1}{2} x^{\frac{1}{2} - 1} \\
                &= \frac{1}{2} x^{-\frac{1}{2}} \\
                &= \frac{1}{2\sqrt{x}}.
        \end{align*}
      \end{freeResponse}

    \item[(b)]
      The function $s$, defined by $\displaystyle s(u) = \frac{5}{u^2}$.
      \begin{freeResponse}
        \begin{align*}
          s'(x) &= \left(\frac{5}{u^2}\right)' = (5u^{-2})' \\
          &= -2 \cdot 5\cdot u^{-2 - 1} \\
          &= -10 u^{-3} \\
          &= \frac{-10}{u^3}.
        \end{align*}
      \end{freeResponse}


    \item[(c)]
      The function $p$, defined by $p(t) = t^5 + 4t^3 + e^\pi$.
      \begin{freeResponse}
                \begin{align*}
          p'(t) &= (t^5 + 4t^3 + e^\pi)'\\
          &= (t^5)' + (4t^3)' + (e^\pi)' \\
          &= 5t^{5-1} + (4t^3)' + (e^\pi)' \\
          &= 5t^{5-1} + 3\cdot 4t^{3-1} + (e^\pi)' \\
          &= 5t^{5-1} + 3\cdot 4t^{3-1} + 0 \\
          &= 5t^4 + 12t^2
        \end{align*}
      \end{freeResponse}
  \end{itemize}
\end{problem}

\begin{problem}
  Given the polynomial function $q$ defined by $q(v) = 2v^3 - 5v^2 + 7v - 9$ find:
  \begin{itemize}
  \outcome{Use ``shortcut'' rules to find and use derivatives.}
  \outcome{Use the quotient rule to calculate derivatives.}
  \outcome{Use the product rule to calculate derivatives.}
    \item[(a)]
      The slope of $q$ at $v = 3$ using the limit definition of a derivative.
      \begin{freeResponse}
        To compute $q'(3)$ from the limit definition we use either 
        \[
          \lim_{v \to 3} \frac{q(v)-q(3)}{v-3} \quad \text{or} \quad \lim_{h \to 0} \frac{q(3+h)-q(3)}{h}.
        \]
        (Recall from Breakout Session 7 Problem 2) that these two formulas mean the same thing.)
        
        Using the first formula:
        \begin{align*}
          q'(3) &= \lim_{v \to 3} \frac{q(v)-q(3)}{v-3} \\
          &= \lim_{v \to 3} \frac{(2v^3 - 5v^2 + 7v - 9) - (54 - 45 + 21 - 9)}{v-3}\\
          &= \lim_{v \to 3} \frac{(2v^3 - 5v^2 + 7v - 9) - 21}{v-3} \\
          &= \lim_{v \to 3} \frac{2v^3 - 5v^2 + 7v - 30}{v-3} \\
          &= \lim_{v \to 3} \frac{(v-3)(2v^2 + v + 10)}{v-3} \\
          &= \lim_{v \to 3} (2v^2 + v + 10)\\
          &= 2 \cdot 3^2 + 3 + 10 = 31.
        \end{align*}
      \end{freeResponse}


    \item[(b)]
      \outcome{Use ``shortcut'' rules to find and use derivatives.}
      The slope of $q$ at $v = 3$ using the ``short-cut derivative rules'' to find a formula for $q'$ and evaluating $q'(3)$.
      \begin{freeResponse}
        Applying the sum and product rules:
        \begin{align*}
          q'(v) &= (2v^3 - 5v^2 + 7v - 9)'\\
          &= (2v^3)' + (-5v^2)' + (7v)' + (-9)' \\
          &= 3\cdot 2 \cdot v^{3-1} + (-5v^2)' + (7v)' + (-9)' \\
          &= 3\cdot 2 \cdot v^{3-1} + 2\cdot -5\cdot v^{2-1} + (7v)' + (-9)'
\\
          &= 3\cdot 2 \cdot v^{3-1} + 2\cdot -5\cdot v^{2-1} + 1\cdot 7\cdot v^{1-1} + (-9)'
\\
          &= 3\cdot 2 \cdot v^{3-1} + 2\cdot -5\cdot v^{2-1} + 1\cdot 7\cdot v^{1-1} + 0 \\
          &= 6v^2 - 10v + 7\\
          &\implies q'(3) = 6\cdot3^2 - 10\cdot3 + 7 = 31.
        \end{align*}
      \end{freeResponse}


    \item[(c)]
      The equation of the tangent line of $q$ at $v = 3$.
      \begin{freeResponse}
        Slope of tangent line: $q'(3) = 6\cdot3^2 - 10\cdot3 + 7 = 31$.

        Point on tangent line: $(3, q(3)) = (3, 21)$.

        Equation of tangent line:
        \begin{align*}
          y-21&=31(x-3)\\
              &\implies y = 31x -72.
        \end{align*}
      \end{freeResponse}

  \end{itemize}
\end{problem}

\begin{problem}
  \outcome{Define higher order derivatives.}
  \outcome{Calculate higher order derivatives.}
   Find $s'$, $s''$, and $s'''$ of the function $s$ defined by $s(t) = 3t^2 + 5e^t - \frac{1}{t}$.
   \begin{freeResponse}
     $s'(t) = 6t + 5e^t + \frac{1}{t^2}$.

     $s''(t) = 6 + 5e^t - \frac{2}{t^3}$.

     $s'''(t) = 5e^t + \frac{6}{t^4}$.
   \end{freeResponse}

\end{problem}
\end{document} 
