\documentclass[handout,nooutcomes]{ximera}
\usepackage{booktabs}
%% handout
%% space
%% newpage
%% numbers
%% nooutcomes

\renewcommand{\outcome}[1]{\marginpar{\null\vspace{2ex}\scriptsize\framebox{\parbox{0.75in}{\begin{raggedright}P\arabic{problem} Outcome: #1\end{raggedright}}}}}

\renewenvironment{freeResponse}{
\ifhandout\setbox0\vbox\bgroup\else
\begin{trivlist}\item[\hskip \labelsep\bfseries Solution:\hspace{2ex}]
\fi}
{\ifhandout\egroup\else
\end{trivlist}
\fi}

\newcommand{\RR}{\mathbb R}
\renewcommand{\d}{\,d}
\newcommand{\dd}[2][]{\frac{d #1}{d #2}}
\renewcommand{\l}{\ell}
\newcommand{\ddx}{\frac{d}{dx}}
\everymath{\displaystyle}
\newcommand{\dfn}{\textbf}
\newcommand{\eval}[1]{\bigg[ #1 \bigg]}


\title{Breakout Session 9: Rules of differentiation}  

\begin{document}
\begin{abstract}
  \textbf{A look back:} In the previous (February 2, 2016) Breakout Session you were introduced to the concept of the derivative of a function and the connection between derivatives and slopes of tangent lines.

  \textbf{Overview:} In today's (February 9, 2016) Breakout Session you'll practice the ``shortcut'' rules for computing derivatives.
  Computing derivatives is the least important part of this course, but \emph{it is important} that you're able to quickly and accurately compute derivatives of various types of functions.

  \textbf{A look ahead:} In the next (February 11, 2016) Breakout Session you will practice rules of differentiation.
  Becoming comfortable and proficient with computing derivatives is an important skill to practice before we return our focus on the conceptual aspects (and interpretation) of derivatives. 
\end{abstract}
\maketitle

\section{Learning Outcomes}
\label{section:learning-outcomes}
The following outcomes are \emph{not an exhaustive} list of the skills you will need to develop and integrate for demonstration on quizzes and exams.
This list is meant to be a starting point for conversation (with your Lecturer, Breakout Session Instructor, and fellow learners) for organizing your knowledge and monitoring the development of your skills.

\subsection*{Basic rules of differentiation}
\begin{itemize}
  \item
    Use ``shortcut'' rules to find and use derivatives.

  \item 
    Use the definition of the derivative to develop shortcut rules to find the derivatives of constants, constant multiples, powers of x, sums of functions, and the natural exponential function.

  \item
    Define higher order derivatives.

  \item
    Calculate higher order derivatives.
\end{itemize}

\subsection*{Product and quotient rules}
\begin{itemize}
  \item
    Use the product rule to calculate derivatives. 

  \item 
    Use the quotient rule to calculate derivatives. 

  \item 
    Identify products of functions.

  \item
    Identify quotients of functions.
\end{itemize}
\newpage

\begin{problem}
  Differentiate the function $f$ defined by $f(x) = 1/x^8$ in two different ways.
\end{problem}

\begin{problem}
    Use the given information to find the equation of the tangent line.
  \begin{itemize}
    \item[(a)]
      Given $g(x) = x^3 f(x)$, $f(2) = 4$, and $f'(2) = 7$, find the equation of the tangent line to the graph of $g$ at $x = 2$.

    \item[(b)]
      Given $h(z) = \frac{z s(z)}{z-3}$, $s(2) = 4$, and $s'(2) = 7$, find the equation of the tangent line to the graph of $h$ at $z = 2$.

    \item[(c)]
      Given
      \begin{center}
        \begin{tabular}{cccccc}
          \toprule
          $x$ & 1 & 2 & 3 & 4 & 5\\
          \midrule
          $f(x)$ & 5 & 3 & 0 &$-4$ & 3\\
          $f'(x)$ & $-3$ & $-5$ & $-2$ & 6& $-4$\\
          $g(x)$ & 6 &9&$-8$&13&15\\
          $g'(x)$ & 8 & 5 & $-10$ &7& 6\\
          \bottomrule
        \end{tabular}
      \end{center}
      find the equation of the tangent line of 
      \[
        \frac{f(x)}{e^xg(x)}
      \]
      at $x = 2$.
  \end{itemize}
\end{problem}

\begin{problem}
    Suppose that
  \begin{center}
    \begin{tabular}{cccc}
      \toprule
      $f(5) = 7$ & $f'(5) = 8$ & $g(5) = 3$ & $g'(5) = -4$\\
      \bottomrule
    \end{tabular}
  \end{center}
  Use the product and quotient rules to find each of the following derivatives:
  \begin{itemize}
    \item[(a)]
      $(fg)'(5)$

    \item[(b)]
      $\displaystyle \left(\frac{f}{g}\right)'(5)$

    \item[(c)]
      $\displaystyle \left(\frac{g}{f}\right)'(5)$

    \item[(d)]
      $(gf)'(5)$
  \end{itemize}
\end{problem}

\begin{problem}
  Differentiate each of the following functions:
  \begin{itemize}
    \item[(a)]
      $f(x) = (x^2 + 4x - 7) e^{-5x}$

    \item[(b)]
      $\displaystyle g(t) = \frac{t^2 + 4t - 7}{e^{-3t}}$
  \end{itemize}
\end{problem}

\section{Extra problems for personal practice}
\begin{problem}
  For each of the following functions use the ``short-cut derivative rules'' to compute their derivatives.
  \begin{itemize}
    \item[(a)]
      The function $f$, defined by $f(x) = \sqrt{x}$.

    \item[(b)]
      The function $s$, defined by $\displaystyle s(u) = \frac{5}{u^2}$.

    \item[(c)]
      The function $p$, defined by $p(t) = t^5 + 4t^3 + e^\pi$.
  \end{itemize}
\end{problem}

\begin{problem}
  Given the polynomial function $q$ defined by $q(v) = 2v^3 - 5v^2 + 7v - 9$ find:
  \begin{itemize}
    \item[(a)]
      The slope of $q$ at $v = 3$ using the limit definition of a derivative.

    \item[(b)]
      The slope of $q$ at $v = 3$ using the ``short-cut derivative rules'' to find a formula for $q'$ and evaluating $q'(3)$.

    \item[(c)]
      The equation of the tangent line of $q$ at $v = 3$.
  \end{itemize}
\end{problem}

\begin{problem}
   Find $s'$, $s''$, and $s'''$ of the function $s$ defined by $s(t) = 3t^2 + 5e^t - \frac{1}{t}$.
\end{problem}
\end{document} 
