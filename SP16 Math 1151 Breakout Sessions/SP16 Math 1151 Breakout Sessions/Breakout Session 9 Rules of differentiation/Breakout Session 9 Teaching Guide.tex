\documentclass[handout,nooutcomes]{ximera}
\usepackage{booktabs}
%% handout
%% space
%% newpage
%% numbers
%% nooutcomes

\renewcommand{\outcome}[1]{\marginpar{\null\vspace{2ex}\scriptsize\framebox{\parbox{0.75in}{\begin{raggedright}P\arabic{problem} Outcome: #1\end{raggedright}}}}}

\renewenvironment{freeResponse}{
\ifhandout\setbox0\vbox\bgroup\else
\begin{trivlist}\item[\hskip \labelsep\bfseries Solution:\hspace{2ex}]
\fi}
{\ifhandout\egroup\else
\end{trivlist}
\fi}

\newcommand{\RR}{\mathbb R}
\renewcommand{\d}{\,d}
\newcommand{\dd}[2][]{\frac{d #1}{d #2}}
\renewcommand{\l}{\ell}
\newcommand{\ddx}{\frac{d}{dx}}
\everymath{\displaystyle}
\newcommand{\dfn}{\textbf}
\newcommand{\eval}[1]{\bigg[ #1 \bigg]}


\title{Breakout Session 9 Teaching Guide}  

\begin{document}
\begin{abstract}
 \textbf{Theme of calculus:} Calculus is the application of  \href{https://en.wikipedia.org/wiki/Derivative}{rates of change} and \href{https://en.wikipedia.org/wiki/Integral}{accumulation} to understand \href{https://en.wikipedia.org/wiki/Elementary_function}{famous functions} in their application to both real world and mathematical processes.

  \textbf{Goal of Math 1151 course:} Promote and cultivate an environment which improves students' ability to construct, organize, and demonstrate their knowledge of calculus.

\end{abstract}
\maketitle

\section{Notes for Problem 1}
You should give students a hint on which two differentiation rules they should try.
The two natural ones are the quotient rule and exteneded power rule.

\textbf{Dangerous and not recommended:} You can also show the students how you can find $f'$ using the product rule.
(By say, rewriting $x^{-8}$ as $x^{-3} \cdot x^{-5}$.)

\section{Notes for Problem 2}
 Pick and choose!
 You probably won't have enough time to solve all of these subproblems.

    This will be another tough problem for the students: this problem combines formulas, differentiation rules, and particular functional values.

    Similar to Problem 4, you should start each solution by writing out the differentiation rule that will be applied.

    Students may be mentally (and physically!) exhausted after computing the value of the derivative.
    So, it's important to motivate and remind the students that they must complete the problem by finding the equation of the tangent line.

\section{Notes for Problem 3}
For part (a), the main point of this problem is to practice with the product rule.

You should point out to your students that you can also solve this problem using the quotient rule---but that involves a bit of algebraic manipulation before differentiating.
Some students may ask, ``how do you pick which rule to use?''
A possible answer, ``you pick the rule that requires the less amount of upfront algebra.''
(This is only a heuristic!)
               
Make sure you carefully work through the derivation of $f'$ and explicitly mention which rules you are applying.
While most of your students have previously taken a course called ``Calculus'', there are still many for whom this is the first time they are working with differentiation rules.

\textbf{Important Note:}Also, when presenting the solution it's important to emphasize, generally, students \emph{do not} have to algebraically simplify their final answers.
You can warn students that the more they write increases the probability that they will make a typo---which increases the probability of point deduction.

For part (b), the main point of this problem is to practice the quotient rule.
The notes above on which rule to pick also applies here.

Again, carefully work through the derivation of $g'$ and explicitly mention the rules you are using.


\section{Notes for Problem 4}
    This problem stresses students' understanding of the product and quotient rules without a given explicit formula for the functions.
    Since the power rule doesn't apply, this problem will be tough for the students.

    Before starting the solutions for each problem you should write the differentiation formulas each time.

\section{Notes for Problem 5}

For part (a), the power rule given in Briggs doesn't cover differentiating this function, but you should tell students that it can be extended to cover this function also.
You should probably rewrite this as $f(x) = x^{1/2}$, before applying the power rule.

For part (b), the power rule given in Briggs doesn’t cover differentiating this function, but you should tell students that it can be extended to cover this function also.

Again, you should probably rewrite this as $s(x) = 5u^{−2}$.

For part (c), this should be a simple application of the sum rule, the power rule, and constant rule.
Some students may be initially confused about $e^\pi$ but remind them that this is just a fancy name for a constant number.

\section{Notes for Problem 6}
The point of parts (a) and (b) is to demonstrate to the students that the ``short-cut rules'' give the correct answer (if correctly applied).

\end{document} 
