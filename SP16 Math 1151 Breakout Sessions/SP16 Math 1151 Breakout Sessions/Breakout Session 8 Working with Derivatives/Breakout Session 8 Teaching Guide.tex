\documentclass[handout,nooutcomes]{ximera}
\usepackage{booktabs}
%% handout
%% space
%% newpage
%% numbers
%% nooutcomes

\renewcommand{\outcome}[1]{\marginpar{\null\vspace{2ex}\scriptsize\framebox{\parbox{0.75in}{\begin{raggedright}P\arabic{problem} Outcome: #1\end{raggedright}}}}}

\renewenvironment{freeResponse}{
\ifhandout\setbox0\vbox\bgroup\else
\begin{trivlist}\item[\hskip \labelsep\bfseries Solution:\hspace{2ex}]
\fi}
{\ifhandout\egroup\else
\end{trivlist}
\fi}

\newcommand{\RR}{\mathbb R}
\renewcommand{\d}{\,d}
\newcommand{\dd}[2][]{\frac{d #1}{d #2}}
\renewcommand{\l}{\ell}
\newcommand{\ddx}{\frac{d}{dx}}
\everymath{\displaystyle}
\newcommand{\dfn}{\textbf}
\newcommand{\eval}[1]{\bigg[ #1 \bigg]}


\title{Breakout Session 8 Teaching Guide}  

\begin{document}
\begin{abstract}
 \textbf{Theme of calculus:} Calculus is the application of  \href{https://en.wikipedia.org/wiki/Derivative}{rates of change} and \href{https://en.wikipedia.org/wiki/Integral}{accumulation} to understand \href{https://en.wikipedia.org/wiki/Elementary_function}{famous functions} in their application to both real world and mathematical processes.

  \textbf{Goal of Math 1151 course:} Promote and cultivate an environment which improves students' ability to construct, organize, and demonstrate their knowledge of calculus.
\end{abstract}
\maketitle

\section{Notes for Problem 1}
For part (I), this problem focuses on students' conceptual understanding of ``differentiable at a point implies continuous at a point''.
(This is Question 7 from Cornell Good Questions project.)

In addition to identifying the correct answer, it may be useful to state why each of the incorrect answers are incorrect.
Don't try to prove things formally!
Just provide a brief explanation along with a picture showing why statements (a), (c), and (d) are incorrect.

Part (a) is incorrect because ``differentiable at a point implies continuous at a point implies limit exists at a point and we know what the limit is equal to''.
For instance, (c) is incorrect because there are many functions where $f'(2)$ exists but $f'(2) \ne \lim_{x \to 2} f(x)$.
(A suitable linear function should do.)
Part (d) is incorrect because ``differentiable at a point implies continuous at a point implies limit exists at a point''.

For part (II), this problem focuses on students' conceptual interpretation of a limit and is tricky.
(This problem is also from the Cornell Good Questions project.)
The tricky part involves interpreting a tangent line in two different ways: part (b) indicates that this limit gives the slope of the tangent line at $x = 0$ while part (e) indicates that tangent lines provides good approximations.

Some students may be tempted to select (a) or (d).
It's important to emphasize that $0/0$ is an indeterminate form and so, by itself, can never be equal to 1.
Selecting part (d) is a less serious error but is still technically wrong.
(Arguing against part (d) will be tricky: in some courses writing $\sin(x) = x$ means $\sin(x) \approx x$.)

Finally, some students may be tempted to select part (c).
It's important to remind them and emphasize that $\sin$ is a function and so it ``captures'' the $x$ in $\sin(x)$.
(Unfortunately, this error is caused by our inconsistent mathematical notation: `$fx$' can stand for ``multiplying $f$ by $x$'' or ``applying the function $f$ to $x$'' or many other things.)

\section{Notes for Problem 2}
When working this problem, it is important to emphasize that the derivative $f'$ helps determine the shape of the graph of $f$.
After solving this problem it's helpful to summarize by drawing the graph of the original function $f$, and under this graph, draw the graph of $f'$.


\section{Notes for Problem 3}
The point of this problem is to test students's understanding of contrapositive form of ``differentiable at a point implies continuous at a point''.
We exclude the endpoints $0$ and $4$ since we \emph{won't} be discussing one-sided derivatives.

\section{Notes for Problem 4}
This will be a difficult problem for the students, especially when they attempt to draw the derivative near the vertical asymptotes.
(It's difficult to see but this graph has a vertical asymptote at $x = -5$, a corner at $x = 10$ and is smooth everywhere else.)

Again, make sure you draw the graph of the original function $h$ and nearby draw the graph of $h'$.

\section{Notes for Problem 5}
This is a problem from the AU15 Exam 2.
Work carefully through the algebra!

\section{Notes for Problem 6}
For part (a), it may be helpful to start students off by substituting `$x$' for `$3$' in the first line.
The algebra will be difficult for some students, so it's important not to skip any steps while computing $f'(x)$.

The point of parts (b), (c), and (d) is to emphasize that $f'$ is a function itself.
All functions have a domain, a range, and (as far as the students know) a graph.
\end{document} 
