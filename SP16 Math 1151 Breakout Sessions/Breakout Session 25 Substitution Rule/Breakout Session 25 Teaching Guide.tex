\documentclass[handout, nooutcomes]{ximera}
\usepackage{booktabs}
%% handout
%% space
%% newpage
%% numbers
%% nooutcomes

\renewcommand{\outcome}[1]{\marginpar{\null\vspace{2ex}\scriptsize\framebox{\parbox{0.75in}{\begin{raggedright}P\arabic{problem} Outcome: #1\end{raggedright}}}}}

\renewenvironment{freeResponse}{
\ifhandout\setbox0\vbox\bgroup\else
\begin{trivlist}\item[\hskip \labelsep\bfseries Solution:\hspace{2ex}]
\fi}
{\ifhandout\egroup\else
\end{trivlist}
\fi}

\newcommand{\RR}{\mathbb R}
\renewcommand{\d}{\,d}
\newcommand{\dd}[2][]{\frac{d #1}{d #2}}
\renewcommand{\l}{\ell}
\newcommand{\ddx}{\frac{d}{dx}}
\everymath{\displaystyle}
\newcommand{\dfn}{\textbf}
\newcommand{\eval}[1]{\bigg[ #1 \bigg]}


\title{Breakout Session 25: Teaching Guide}

\begin{document}
\begin{abstract}

\end{abstract}
\maketitle

\section{Overall notes}
It's important to cover substitution for both indefinite and definite integrals.

I suggest you only work problems 1, 3, and 6(a) and (c). 

\section{Notes for problem 1}
This problem forces students to start thinking about how they will perform a particular substitution.
It’s important to stress that students should practice and be comfortable with both substitutions.

\section{Notes for problem 3}
This problem is from a previous exam.
Please go through the algebra and substitution slowly: students will have problems with the algebra and determining which expression should be substituted.


Before starting on the first couple of problems, I suggest you review the definition, appropriate graphs, and a few examples of even and odd functions.

Part (3) (of problem 1) will be tricky for students.
The point of this problem is to encourage students to understand symmetry by looking at the appropriate graph before integrating.
(Cosine is an even function, but on the given non-symmetric interval, it has rotational symmetry.)

\section{Notes for problem 6}
For at least one part, I suggest you draw the graph of the integrand over the appropriate interval and the graph of the $y = f(u)$ function over the transformed interval.
This way students can see how substitution makes integration easier.
\end{document} 
