\documentclass[nooutcomes]{ximera}
%% handout
%% space
%% newpage
%% numbers
%% nooutcomes


\newcommand{\RR}{\mathbb R}
\renewcommand{\d}{\,d}
\newcommand{\dd}[2][]{\frac{d #1}{d #2}}
\renewcommand{\l}{\ell}
\newcommand{\ddx}{\frac{d}{dx}}
\newcommand{\dfn}{\textbf}
\newcommand{\eval}[1]{\bigg[ #1 \bigg]}

\renewenvironment{freeResponse}{
\ifhandout\setbox0\vbox\bgroup\else
\begin{trivlist}\item[\hskip \labelsep\bfseries Solution:\hspace{2ex}]
\fi}
{\ifhandout\egroup\else
\end{trivlist}
\fi} %% we can turn off input when making a master document

\title{Recitation \#10 - 3.5 Derivatives of Trig Functions (Solutions)}  

\begin{document}
\begin{abstract}		\end{abstract}
\maketitle

\section*{Warm up:} 
	
	\begin{freeResponse}
	\end{freeResponse}	
	
	
	
	
	

\section*{Group work:}

%problem 1
\begin{problem}
Find the following limits:

	\begin{enumerate}
	
	%part a
	\item  $\lim_{x \to 0} \frac{\sin(8x)}{x}$
			\begin{freeResponse}
			
			\end{freeResponse}
			
			
			
	%part b
	\item  $\lim_{x \to 0} \frac{\cos^2(x) - 1}{4x}$
			\begin{freeResponse}
			
			\end{freeResponse}
			
			
			
	%part c
	\item  $\lim_{x \to 0} \frac{x}{\tan(5x)}$
			\begin{freeResponse}
			
			\end{freeResponse}
			
			
			
	\end{enumerate}
\end{problem}
	
	
	
	
			
			

%problem 2			
\begin{problem}
Find the derivative of the following functions:

	\begin{enumerate}
	
	%part a
	\item  $f(x) = \frac{x+5}{7x^6 + \cot(x)}$
			\begin{freeResponse}
			$f'(x) = \frac{(7x^6 + \cot(x))(1) - (x+5)(42x^5 - \csc^2(x))}{(7x^6 + \cot(x))^2}
			= f'(x) = \frac{7x^6 + \cot(x) - (x+5)(42x^5 - \csc^2(x))}{(7x^6 + \cot(x))^2}$.
			\end{freeResponse}
			
			
			
	%part b
	\item  $f(x) = \sin(x) \cos(x)$
			\begin{freeResponse}
			$f'(x) = (\cos(x))(\cos(x)) + (\sin(x))(-\sin(x)) = \cos^2(x) - \sin^2(x)$.
			\end{freeResponse}
			
			
			
	%part c
	\item  $f(x) = \frac{e^x \tan(x)}{\sec(x) + 2}$
			\begin{freeResponse}
			$f'(x) = \frac{(\sec(x)+2)(e^x \tan(x) + e^x \sec^2(x)) - e^x \tan(x) (\sec(x) \tan(x))}{(\sec(x) + 2)^2}
			= \frac{e^x[(\sec(x) + 2)(\tan(x) + \sec^2(x)) - \sec(x) \tan^2(x)]}{(\sec(x) + 2)^2}$.
			\end{freeResponse}
			
			
			
	%part d
	\item  $f(x) = \sin(x) \cos(x) e^{3x}$
			\begin{freeResponse}
			$f'(x) = \ddx(\sin(x) \cos(x)) e^{3x} + (\sin(x) \cos(x)) \ddx(e^{3x})$
			
			$= (\cos^2(x) - \sin^2(x))e^{3x} + 3e^{3x} \sin(x) \cos(x)$
			
			$= e^{3x}(\cos^2(x) + 3\sin(x) \cos(x) - \sin^2(x))$.
			\end{freeResponse}
			
			
			
	\end{enumerate}
		
\end{problem}









%problem 3			
\begin{problem}

		\begin{freeResponse}
		
		\end{freeResponse}
		
\end{problem}







%problem 4
\begin{problem}

		\begin{freeResponse}
		\end{freeResponse}
		



\end{problem}
	
	
	
	
	
	
	
	
	

	










								
				
				
	














\end{document} 


















