\documentclass[handout,nooutcomes]{ximera}
%% handout
%% space
%% newpage
%% numbers
%% nooutcomes


\newcommand{\RR}{\mathbb R}
\renewcommand{\d}{\,d}
\newcommand{\dd}[2][]{\frac{d #1}{d #2}}
\renewcommand{\l}{\ell}
\newcommand{\ddx}{\frac{d}{dx}}
\newcommand{\dfn}{\textbf}
\newcommand{\eval}[1]{\bigg[ #1 \bigg]}

\usepackage{multicol}

\renewenvironment{freeResponse}{
\ifhandout\setbox0\vbox\bgroup\else
\begin{trivlist}\item[\hskip \labelsep\bfseries Solution:\hspace{2ex}]
\fi}
{\ifhandout\egroup\else
\end{trivlist}
\fi} %% we can turn off input when making a master document

\title{Recitation \#5 - 2.4 Infinite Limits and 2.5 Limits at Infinity (Teaching Guide)}  

\begin{document}
\begin{abstract}		\end{abstract}
\maketitle

Here is a suggested structure for this recitation

\section*{Warm up:} 

	\begin{itemize}
	
	\item  \emph{5 minutes}:  Let students work on the problems in groups.
	
	\item  \emph{5 minutes}:   Bring the class together to discuss solutions.
	
	\end{itemize}


\section*{Problem 1:}

	\begin{itemize}
	
	\item  \emph{5 minutes}:  Allow students to work on problem 1 in groups.  Assign different groups to start with (a) and (b).
	
	\item  \emph{5 minutes}:  Let students present their solutions to the two parts.  
		
	\end{itemize}
	
	
	
\section*{Problem 2:}

	\begin{itemize}
	
	\item  \emph{10 minutes}:  Have each group work on the problem
	
	\item  \emph{5 minutes}:  Let a group present their solution, or discuss as a class if there seems to be a lot of confusion on what to do.
	
	\end{itemize}
	
	
	
\section*{Problem 3:}

	\begin{itemize}
	
	\item  \emph{10 minutes}:  Allow students to work on problem 3 in groups.  Assign different groups to start with (a) and (b).
	
	\item  \emph{10 minutes}:  Discuss problem 1 as a class.  Ask for input from the students in the groups who started with that problem.  Be sure to explain why the limit is negative as the limit approaches negative infinity.  It may help to give a concrete example like $\sqrt{(-2)^2} = 2$.  
		
	\end{itemize}
	
	
	

	
	
	
















\end{document}