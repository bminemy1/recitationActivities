\documentclass[handout,nooutcomes]{ximera}
%% handout
%% space
%% newpage
%% numbers
%% nooutcomes


\newcommand{\RR}{\mathbb R}
\renewcommand{\d}{\,d}
\newcommand{\dd}[2][]{\frac{d #1}{d #2}}
\renewcommand{\l}{\ell}
\newcommand{\ddx}{\frac{d}{dx}}
\newcommand{\dfn}{\textbf}
\newcommand{\eval}[1]{\bigg[ #1 \bigg]}

\usepackage{multicol}

\renewenvironment{freeResponse}{
\ifhandout\setbox0\vbox\bgroup\else
\begin{trivlist}\item[\hskip \labelsep\bfseries Solution:\hspace{2ex}]
\fi}
{\ifhandout\egroup\else
\end{trivlist}
\fi} %% we can turn off input when making a master document

\title{Recitation \#18 - 4.4 Optimization Problems (Teaching Guide)}  

\begin{document}
\begin{abstract}		\end{abstract}
\maketitle


\section*{Warm up:} 
	
	\begin{itemize}
	
	\item  \emph{5 minutes}:  Discuss the warm-up.  Then review the methods for finding absolute maximums or minimums and how you know which one may apply.  (Extreme Value Theorem verse if there is only one CP on an open interval so you can use 1st or 2nd Derivative test).
	
	
	
	\end{itemize}


\section*{Problem 1:}

	\begin{itemize}
	
	\item  \emph{10 minutes}:  Have the students do \#1 parts $a-d$ in groups.  Walk around and help them through the steps, pulling them together for a minute to talk about a step if multiple groups seem to be having trouble.  
	
	\item  \emph{5 minutes}:  Review problem \#1 parts $a-d$ as a group.
	
	\item  \emph{10 minutes}:  Have the students finish problem \#1 in groups.  Walk around and help them through the steps, pulling them together for a minute to talk about a step if multiple groups seem to be having trouble.  
	
	\item  \emph{5 minutes}:  Review problem \#1 as a group
	
	\end{itemize}



\section*{Problem 2:}

	\begin{itemize}
	
	\item  \emph{10 minutes}:  Have the students work on problem \#2
		
	\item  \emph{10 minutes}:  Let a group present if there is time.
			
	\end{itemize}
	
For both, you may want to do the geometry/set-up as a whole class and then ask them to finish the problem in their groups.
	
	
	

	
	
	

	
	

	
	
	

	
	
	
















\end{document}