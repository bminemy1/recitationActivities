\documentclass[handout]{ximera}
%handout:  for handout version with no solutions or instructor notes
%handout,instructornotes:  for instructor version with just problems and notes, no solutions
%noinstructornotes:  shows only problem and solutions

%% handout
%% space
%% newpage
%% numbers
%% nooutcomes

%I added the commands here so that I would't have to keep looking them up
%\newcommand{\RR}{\mathbb R}
%\renewcommand{\d}{\,d}
%\newcommand{\dd}[2][]{\frac{d #1}{d #2}}
%\renewcommand{\l}{\ell}
%\newcommand{\ddx}{\frac{d}{dx}}
%\everymath{\displaystyle}
%\newcommand{\dfn}{\textbf}
%\newcommand{\eval}[1]{\bigg[ #1 \bigg]}

%\begin{image}
%\includegraphics[trim= 170 420 250 180]{Figure1.pdf}
%\end{image}

%add a ``.'' below when used in a specific directory.
\newcommand{\RR}{\mathbb R}
\renewcommand{\d}{\,d}
\newcommand{\dd}[2][]{\frac{d #1}{d #2}}
\renewcommand{\l}{\ell}
\newcommand{\ddx}{\frac{d}{dx}}
\newcommand{\dfn}{\textbf}
\newcommand{\eval}[1]{\bigg[ #1 \bigg]}

\renewenvironment{freeResponse}{
\ifhandout\setbox0\vbox\bgroup\else
\begin{trivlist}\item[\hskip \labelsep\bfseries Solution:\hspace{2ex}]
\fi}
{\ifhandout\egroup\else
\end{trivlist}
\fi} %% we can turn off input when making a master document

\usepackage{fullpage}

\title{Recitation \#1: Review of Functions}  

\begin{document}
\begin{abstract}		\end{abstract}
\maketitle












\section{Group work:}



%problem 1
\begin{problem}
Define $f(x)=$
\begin{enumerate}	
	\item  Find an expression for $a$ such that $f(x)$ will be even.
	
	\item  Find an expression for $a$ such that $f(x)$ will be odd.
	
	\end{enumerate}
	
	\begin{freeResponse}
		\begin{enumerate}
		
		\item 
		
		\item  
		
		\end{enumerate}
	\end{freeResponse}
	
\end{problem}

\begin{instructorNotes}

\end{instructorNotes}



%problem 2
\begin{problem}
The graph of $g(x)=e^x$ is given below.
\begin{enumerate}	
	\item  Find the domain and range of $g(x)$.
	
	\item  Find the values of $g(1), g(0), g(-1)$.
	
	\end{enumerate}
	
	\begin{freeResponse}
		\begin{enumerate}
		
		\item 
		
		\item  
		
		\end{enumerate}
	\end{freeResponse}
	
\end{problem}

\begin{instructorNotes}

\end{instructorNotes}









\end{document} 


















