\documentclass[handout,nooutcomes]{ximera}
%% handout
%% space
%% newpage
%% numbers
%% nooutcomes


\newcommand{\RR}{\mathbb R}
\renewcommand{\d}{\,d}
\newcommand{\dd}[2][]{\frac{d #1}{d #2}}
\renewcommand{\l}{\ell}
\newcommand{\ddx}{\frac{d}{dx}}
\newcommand{\dfn}{\textbf}
\newcommand{\eval}[1]{\bigg[ #1 \bigg]}

\renewenvironment{freeResponse}{
\ifhandout\setbox0\vbox\bgroup\else
\begin{trivlist}\item[\hskip \labelsep\bfseries Solution:\hspace{2ex}]
\fi}
{\ifhandout\egroup\else
\end{trivlist}
\fi} %% we can turn off input when making a master document

\title{Recitation \#2 - 2.1:  The Idea of Limits (Teaching Guide)}  

\begin{document}
\begin{abstract}		\end{abstract}
\maketitle

Here is a suggested structure for this recitation

\section*{Warm up:} 

	\begin{itemize}
	
	\item  \emph{5 minutes}:  Let students think about the question and then bring together for a whole class discussion.  Discuss the differences between a secant and tangent line.  Foor a non-linear function, are they the same line?
	
	\end{itemize}


\section*{Problem 1:}

	\begin{itemize}
	
	\item  \emph{10 minutes}:  Give students a minute to think about \#1 parts (a), (b),and (c), and then discuss as a whole class.
	
	\item  \emph{15 minutes}:  Let students work in groups on \#1d.  Then, discuss \#1d as a class.
	
	\item  \emph{10 minutes}:  Let students work in groups on \#1e and \#1f.  Then, discuss these problems as a class.
		
	\end{itemize}
	
	
	
\section*{Problem 2:}

	\begin{itemize}
	
	\item  \emph{15 minutes}:  Let students work in groups on \#2.  Then, discuss this problem as a class.  If there is time, these would be good problems to have students come to the board and present the solutions.  Emphasize that our answer of ?6? is an approximation.  The real answer is still unknown and isn?t necessarily a nice whole number.	
	
	\end{itemize}
	
	
	
















\end{document}