\documentclass[handout,nooutcomes]{ximera}
%% handout
%% space
%% newpage
%% numbers
%% nooutcomes


\newcommand{\RR}{\mathbb R}
\renewcommand{\d}{\,d}
\newcommand{\dd}[2][]{\frac{d #1}{d #2}}
\renewcommand{\l}{\ell}
\newcommand{\ddx}{\frac{d}{dx}}
\newcommand{\dfn}{\textbf}
\newcommand{\eval}[1]{\bigg[ #1 \bigg]}

\usepackage{multicol}

\renewenvironment{freeResponse}{
\ifhandout\setbox0\vbox\bgroup\else
\begin{trivlist}\item[\hskip \labelsep\bfseries Solution:\hspace{2ex}]
\fi}
{\ifhandout\egroup\else
\end{trivlist}
\fi} %% we can turn off input when making a master document

\title{Recitation \#3 - 2.2:  Definition of Limits (Teaching Guide)}  

\begin{document}
\begin{abstract}		\end{abstract}
\maketitle

Here is a suggested structure for this recitation

\section*{Warm up:} 

	\begin{itemize}
	
	\item  \emph{5 minutes}: Ask students to think about the Warm-up as they are waiting for class to begin.  Then discuss the Warm-Up as a class when class begins. 
	
	\end{itemize}


\section*{Problem 1:}

	\begin{itemize}
	
	\item  \emph{10 minutes}:  Allow students to work on problem 1 in groups.  Identify students to present if students’ seem to understand the idea.  You may have students write up their work on the board whiles other groups are still finishing.
	
	\item  \emph{10 minutes}:  Discuss problem 1 as a class.  Allow students to present.
		
	\end{itemize}
	
	
	
\section*{Problem 2:}

	\begin{itemize}
	
	\item  \emph{10 minutes}:  Allow students to work on \#2 in groups.
	
	\item  \emph{5 minutes}:  Talk about \#2 as a class.  Allow a student to present.
	
	\end{itemize}
	
	
	
\section*{Problem 3:}

	\begin{itemize}
	
	\item  \emph{5 minutes}:  Allow students to talk about \#3 in groups
	
	\item  \emph{10 minutes}:  Talk about \#3 as a class.  Take a quick poll of whether the students think each part is true or false before discussing it.  Solicit explanations and counterexamples from the students.
	
	\end{itemize}
	
	
	
















\end{document}