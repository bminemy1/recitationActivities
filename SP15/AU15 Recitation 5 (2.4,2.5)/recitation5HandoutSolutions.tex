\documentclass[nooutcomes]{ximera}
%% handout
%% space
%% newpage
%% numbers
%% nooutcomes


\newcommand{\RR}{\mathbb R}
\renewcommand{\d}{\,d}
\newcommand{\dd}[2][]{\frac{d #1}{d #2}}
\renewcommand{\l}{\ell}
\newcommand{\ddx}{\frac{d}{dx}}
\newcommand{\dfn}{\textbf}
\newcommand{\eval}[1]{\bigg[ #1 \bigg]}

\renewenvironment{freeResponse}{
\ifhandout\setbox0\vbox\bgroup\else
\begin{trivlist}\item[\hskip \labelsep\bfseries Solution:\hspace{2ex}]
\fi}
{\ifhandout\egroup\else
\end{trivlist}
\fi} %% we can turn off input when making a master document

\title{Recitation \#5 - 2.4 Infinite Limits and 2.5 Limits at Infinity (Solutions)}  

\begin{document}
\begin{abstract}		\end{abstract}
\maketitle

\section*{Warm up:} 
Come up with your own example of a limit of a function $f(x)$, as $x$ approaches 4, that will be of the following form:

	\begin{enumerate}[label=(\alph*)]
	
	\item  Limit is of the form $\frac{0}{0}$ and the limit exists as a finite number.
		\begin{freeResponse}
		Let $f(x) = \frac{4-x}{x(4-x)}$.  Then $\lim_{x \to 4} f(x)$ is of the form $\frac{0}{0}$ and $\lim_{x \to 4} \frac{4-x}{x(4-x)} = \lim_{x \to 4} \frac{1}{x} = \frac{1}{4} $.  
		\end{freeResponse}
	
	
	
	\item  Limit is of the form $\frac{0}{0}$ and the limit is infinite.
		\begin{freeResponse}
		Let $f(x) = \frac{4-x}{(4-x)^3}$.  Notice that, for $x \neq 4$, $f(x) = \frac{1}{(x-4)^2}$.  $\lim_{x \to 4} f(x)$ is of the form $\frac{0}{0}$, and $\lim_{x \to 4} \frac{1}{(x-4)^2} $ is of the form $\frac{1}{0}$.  Thus the limit is infinite, and since both $1$ and $(x-4)^2$ are always positive (for $x \neq 4$) we have that $\lim_{x \to 4} f(x) = \infty $.  		
		\end{freeResponse}
	
	\item  Limit is infinite and approaches positive infinity from both sides of 4.
		\begin{freeResponse}
		 The same function as part (b) works.  
		 \end{freeResponse}
	
	
	
	\item  Limit does not exist (DNE), but approaches positive infinity from the left side of 4 and negative infinity from the right side of 4.
		\begin{freeResponse}  Let $f(x) = \frac{1}{4-x}$.  Since $\lim_{x \to 4} f(x)$ is of the form $\frac{1}{0}$, the limit is either $\infty, \, -\infty, $ or DNE.  Since $1 > 0$, $4-x > 0$ as $x \to 4$ from the left, and $4-x < 0$ as $x \to 4$ from the right, we have that $ \lim_{x \to 4^-} f(x) = \infty  $ and $ \lim_{x \to 4^+} f(x) = - \infty  $.
		\end{freeResponse}
	
	\end{enumerate}

	
	

\section*{Group work:}

%problem 1
\begin{problem}

Determine the following limits:
	\begin{enumerate}
	
	%part a		
	\item  $\lim_{x \to 3} \frac{x^2 - 3}{x^2 - x - 6}   $
		\begin{freeResponse}
		Notice that as $x \to 3$, $(x^2 - 3) \to 6$ and $(x^2 - x - 6) \to 0$.  Thus, this limit is of the form $\frac{\neq 0}{0}$ and therefore the answer is one of $\infty$, $-\infty$, or the limit does not exist (DNE).  What we need to do is check the limits as $x$ approaches $3$ from both the left and right hand sides.  Notice that, in either sided limit, the numerator approaches $9 > 0$.  Also, the denominator factors as $(x-3)(x+2)$.  As $x \to 3^-$, $(x-3)$ is negative making the denominator and thus the entire fraction negative (note that $x+2$ approaches $5$ which is positive).  As $x \to 3^+$, $(x-3)$ is positive making the entire fraction positive.  So we can conclude: 
		
		$ \lim_{x \to 3^-} \frac{x^2 - 3}{x^2 - x - 6} = -\infty$
		
		$ \lim_{x \to 3^+} \frac{x^2 - 3}{x^2 - x - 6} = \infty$
		
		Therefore \, $ \lim_{x \to 3} \frac{x^2 - 3}{x^2 - x - 6}$ \, DNE
		\end{freeResponse}
			
			
	
	%part b		
	\item  $\lim_{x \to 1} \frac{4-x}{x^2 - 2x + 1}   $
		\begin{freeResponse}
		This is exactly like parts (a) and (b) above in that the limit is of the form $\frac{\neq 0}{0}$, and so we need to consider the sided limits.  The function can be rewritten as $\frac{4-x}{(x-1)^2}$.  As $x \to 1$ (from either side), $(4-x) \to 3$ and 3 is positive.  But now notice that, for $x \neq 1$, the denominator $(x-1)^2$ is always positive.  So as $x$ approaches 1 from \textbf{both} the left and right hand sides, the entire fraction is positive.  Thus:
		
		$ \lim_{x \to 1^-} \frac{4-x}{x^2 - 2x + 1} = \infty$
		
		$ \lim_{x \to 1^+} \frac{4-x}{x^2 - 2x + 1} = \infty$
		
		Therefore \, $ \lim_{x \to 1} \frac{4-x}{x^2 - 2x + 1} = \infty$  
		\end{freeResponse}
		
			
	\end{enumerate}

\end{problem}
	
	
	
	
			
			
%problem 2			
\begin{problem}

Use the Squeeze Theorem to determine the value of $ \lim_{x \to 0} x \cos \left( \frac{1}{x} \right)  $
		\begin{freeResponse}
		For all $x \neq 0$ we have that
		
		\begin{equation}
		-1 \leq \cos \left( \frac{1}{x} \right) \leq 1.
		\end{equation}
		
		We now need to split the problem up into two cases, as $x$ approaches $0$ from the right and as $x$ approaches $0$ from the left.
		
		(Case 1:  $x \to 0^+$)
		
		In this case $x > 0$, and so multiplying equation (1) by $x$ yields
		
		$$ -x \leq x \cos \left( \frac{1}{x} \right) \leq x. $$
		
		But we know that $\lim_{x \to 0^+} (-x) = 0 = \lim_{x \to 0^+} x$, and thus by the Squeeze Theorem we can conclude that $\lim_{x \to 0^+} x \cos \left( \frac{1}{x} \right) = 0$.
		
		(Case 2:  $x \to 0^-$)
		
		In this case $x < 0$, and so multiplying equation (1) by $x$ yields
		
		$$ -x \geq x \cos \left( \frac{1}{x} \right) \geq x.$$
		
		But just like in Case 1 we know that $\lim_{x \to 0^-} x = 0 = \lim_{x \to 0^-} (-x)$, and thus by the Squeeze Theorem we can conclude that $\lim_{x \to 0^-} x \cos \left( \frac{1}{x} \right) = 0$.
		
		Therefore, since $\lim_{x \to 0^-} x \cos \left( \frac{1}{x} \right) = 0 = \lim_{x \to 0^+} x \cos \left( \frac{1}{x} \right) = 0$, we can conclude that $\lim_{x \to 0} x \cos \left( \frac{1}{x} \right) = 0$.
		\end{freeResponse}

\end{problem}
	
	
	
	
	
	
	
	
	
	
\begin{problem}

Find any vertical or horizontal asymptotes for the given function.  Be sure to tell where the function crosses its horizontal or vertical asymptote.  Also, when finding vertical asymptotes, be sure to say how the function approaches the asymptote on each side:

	\begin{enumerate}
	
	%part a
	\item  $f(x) = \frac{\sqrt{2x^2 + 1}}{3x-5}$
		
		\begin{freeResponse}
		To find the vertical asymptote, solve:
		$$3x-5=0 \qquad \Longrightarrow \qquad 3x = 5 \qquad \Longrightarrow \qquad x = \frac{5}{3}.$$
		
		$\lim_{x \to \frac{5}{3}^+} \frac{\sqrt{2x^2 + 1}}{3x-5} = \infty$ and $\lim_{x \to \frac{5}{3}^-} \frac{\sqrt{2x^2 + 1}}{3x-5} = - \infty$ since in both limits the numerator approaches $\frac{\sqrt{59}}{3}$, but the denominator of the first limit approaches $0$ from the right whereas the denominator of the second limit approaches $0$ from the left.
		
		To find the horizontal asymptotes:
		\begin{align*}
		(\text{as} x \to \infty)  \lim_{x \to \infty} \frac{\sqrt{2x^2 + 1}}{3x-5} &= \lim_{x \to \infty} \frac{\sqrt{2x^2 + 1}}{3x-5} \cdot \frac{\frac{1}{x}}{\frac{1}{x}} \\
		&= \lim_{x \to \infty} \frac{\frac{\sqrt{2x^2 + 1}}{x}}{\frac{3x-5}{x}} \\
		&= \lim_{x \to \infty}  \frac{\frac{\sqrt{2x^2 + 1}}{\sqrt{x^2}}}{3 - \frac{5}{x}} \\
		&= \lim_{x \to \infty}  \frac{\sqrt{\frac{2x^2 + 1}{x^2}}}{3 - \frac{5}{x}} \\
		&= \lim_{x \to \infty}  \frac{\sqrt{2 + \frac{1}{x^2}}}{3 - \frac{5}{x}} \\
		&= \frac{\sqrt{2+0}}{3-0} = \frac{\sqrt{2}}{3} 
		\end{align*}
		\begin{align*}
		(\text{as} x \to -\infty)  \lim_{x \to -\infty} \frac{\sqrt{2x^2 + 1}}{3x-5} &= \lim_{x \to -\infty} \frac{\sqrt{2x^2 + 1}}{3x-5} \cdot \frac{\frac{1}{x}}{\frac{1}{x}} \\
		&= \lim_{x \to -\infty} \frac{\frac{\sqrt{2x^2 + 1}}{x}}{\frac{3x-5}{x}} \\
		&= \lim_{x \to -\infty}  \frac{\frac{\sqrt{2x^2 + 1}}{-\sqrt{x^2}}}{3 - \frac{5}{x}} \\
		&= \lim_{x \to -\infty}  -\frac{\sqrt{\frac{2x^2 + 1}{x^2}}}{3 - \frac{5}{x}} \\
		&= \lim_{x \to -\infty}  -\frac{\sqrt{2 + \frac{1}{x^2}}}{3 - \frac{5}{x}} \\
		&= -\frac{\sqrt{2+0}}{3-0} = -\frac{\sqrt{2}}{3} 
		\end{align*}
		
		So there are horizontal asymptotes at $y = \pm \frac{\sqrt{2}}{3}$.  
		
		In order for the graph of $f(x)$ to cross $y = \frac{\sqrt{2}}{3}$, we must have that 
		\begin{align*}
		\frac{\sqrt{2x^2 + 1}}{3x-5} = \frac{\sqrt{2}}{3} &\qquad \Longrightarrow \qquad 3 \sqrt{2x^2+1} = \sqrt{2}(3x-5) \\
		&\qquad \Longrightarrow \qquad 9 (2x^2 + 1) = 2(9x^2 - 30x + 25)  \\
		&\qquad \Longrightarrow \qquad 18x^2 + 9 = 18x^2 - 60x + 50  \\
		&\qquad \Longrightarrow \qquad 60x = 41  \\
		&\qquad \Longrightarrow \qquad x = \frac{41}{60}.
		\end{align*}
		
		We need to check this:  $f \left( \frac{41}{60} \right) = -\frac{\sqrt{2}}{3}$.  So the graph of $f(x)$ never crosses the line $y= \frac{\sqrt{2}}{3}$, but it crosses $y= -\frac{\sqrt{2}}{3}$ at the point $\left( \frac{41}{60}, - \frac{\sqrt{2}}{3} \right)$.  
		\end{freeResponse}
	
	
	
	%part b
	\item  $f(x) = \frac{x^2 + 7x + 11}{x-3}$
	
		\begin{freeResponse}
		
		To find the vertical asymptote, we solve $x - 3 = 0$, and so $x = 3$.  
		
		$\lim_{x \to 3^-}  \frac{x^2 + 7x + 11}{x-3} = -\infty$ and $\lim_{x \to 3^+}  \frac{x^2 + 7x + 11}{x-3} = \infty$ since, in both cases the numerator approaches $41$, but in the first limit the denominator approaches $0$ from the left whereas in the second limit the denominator approaches $0$ from the right.
		
		\dfn{Because the degree of the numerator is one higher than the degree of the denominator, this function has a slant asymptote.}  Performing long division, we see that $\frac{x^2 + 7x + 11}{x-3} = x + 10 - \frac{41}{x-3}$.  Thus, $y = x+10$ is a slant asymptote for $f(x)$.  $f(x)$ has no horizontal asymptotes.
		
		\end{freeResponse}
	
	\end{enumerate}

\end{problem}
	










								
				
				
	














\end{document} 


















