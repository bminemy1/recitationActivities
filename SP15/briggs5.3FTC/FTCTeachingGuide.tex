\documentclass[handout,nooutcomes]{ximera}
%% handout
%% space
%% newpage
%% numbers
%% nooutcomes


\newcommand{\RR}{\mathbb R}
\renewcommand{\d}{\,d}
\newcommand{\dd}[2][]{\frac{d #1}{d #2}}
\renewcommand{\l}{\ell}
\newcommand{\ddx}{\frac{d}{dx}}
\newcommand{\dfn}{\textbf}
\newcommand{\eval}[1]{\bigg[ #1 \bigg]}

\usepackage{multicol}

\renewenvironment{freeResponse}{
\ifhandout\setbox0\vbox\bgroup\else
\begin{trivlist}\item[\hskip \labelsep\bfseries Solution:\hspace{2ex}]
\fi}
{\ifhandout\egroup\else
\end{trivlist}
\fi} %% we can turn off input when making a master document

\title{Recitation \#24 - Fundamental Theorem of Calculus Part I (Teaching Guide)}  

\begin{document}
\begin{abstract}		\end{abstract}
\maketitle


\section*{Warm up:} 
	
	\begin{itemize}
	
	\item  \emph{5 minutes}:  Give students a minute to read/think about the warm up, take a show of hands, and then discuss it.
	
	
	
	\end{itemize}


\section*{Problem 1:}

	\begin{itemize}
	
	\item  \emph{5 minutes}:  Solve 1(a) as a class, reminding the students of the FTC and how to combine it with the chain rule.  
	
	\item  \emph{5 minutes}:  Let students work on 1(b) in groups.  
	
	\item  \emph{5 minutes}:  Have a group present their work for 1(b).  Note that this could also be done finding an anti-derivative, plugging in, and then differentiating the result. Show that this will result in the same answer.
	
	\end{itemize}



\section*{Problem 2:}

	\begin{itemize}
	
	\item  \emph{10 minutes}:  Let students work on \#2 in groups.
		
	\item  \emph{10 minutes}:  Discuss \#2 as a class.  You can use the interactive figure with the graph of the accumulation function at 
	
	\url{https://people.math.osu.edu/miller.4962/152applets/Applets/Accumulation_Function.nbp}
	
	to help explain to the students what is happening.
			
	\end{itemize}
	
	
	
\section*{Problem 3:}

	\begin{itemize}
	
	\item  \emph{10 minutes}:  Let students work on \#3 (a) and (b) for a couple minutes and then go over as a class.  Let groups present if there is time.
		
	\item  \emph{5 minutes}:  Go over \#3(c) as a class.
	
	\end{itemize}
	


	
	
	

	
	

	
	
	

	
	
	
















\end{document}