\documentclass[nooutcomes]{ximera}
%% handout
%% space
%% newpage
%% numbers
%% nooutcomes

%\ddx
%\dfn


\newcommand{\RR}{\mathbb R}
\renewcommand{\d}{\,d}
\newcommand{\dd}[2][]{\frac{d #1}{d #2}}
\renewcommand{\l}{\ell}
\newcommand{\ddx}{\frac{d}{dx}}
\newcommand{\dfn}{\textbf}
\newcommand{\eval}[1]{\bigg[ #1 \bigg]}

\renewenvironment{freeResponse}{
\ifhandout\setbox0\vbox\bgroup\else
\begin{trivlist}\item[\hskip \labelsep\bfseries Solution:\hspace{2ex}]
\fi}
{\ifhandout\egroup\else
\end{trivlist}
\fi} %% we can turn off input when making a master document

\title{3.3 Rules of Differentiation (Solutions)}  

\begin{document}
\begin{abstract}		\end{abstract}
\maketitle

\section*{Warm up:} 
	
	\begin{freeResponse}
	\end{freeResponse}	
	
	
	
	
	

\section*{Group work:}

%problem 1
\begin{problem}
Use the ``short-cut derivative rules" to compute the derivatives of the following functions:
	
	\begin{enumerate}
	
	%part a
	\item $f(x) = \sqrt{x}$
	
		\begin{freeResponse}
		$f(x) = \sqrt{x} = x^{\frac{1}{2}}.$  So $f'(x) = \frac{1}{2} x^{\frac{1}{2} - 1} = \frac{1}{2} x^{-\frac{1}{2}} = \frac{1}{2\sqrt{x}}$.
		\end{freeResponse}
			
			
	
	%part b
	\item $f(x) = \frac{5}{x^2}$
	
		\begin{freeResponse}
		$f(x) = \frac{5}{x^2} = 5x^{-2}.$  So $f'(x) = 5(-2) x^{-2-1} = -10x^{-3} = \frac{-10}{x^3}$.  
		\end{freeResponse}
			
			
	
	%part c
	\item $f(x) = x^5 + 4x^3 + \pi $
	
		\begin{freeResponse}
		$f'(x) = 5x^{5-1} + 4(3)x^{3-1} + 0 = 5x^4 + 12x^2$.  Note that $\ddx (\pi) = 0$ because $\pi$ is a constant.	
		\end{freeResponse}
			
			
	
	\end{enumerate}
\end{problem}
	
	
	
	
			
			

%problem 2			
\begin{problem}

		\begin{freeResponse}

		\end{freeResponse}
		
\end{problem}









%problem 3			
\begin{problem}

		\begin{freeResponse}
		
		\end{freeResponse}
		
\end{problem}







%problem 4
\begin{problem}

		\begin{freeResponse}
		\end{freeResponse}
		



\end{problem}
	
	
	
	
	
	
	
	
	

	










								
				
				
	














\end{document} 


















