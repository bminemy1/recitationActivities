\documentclass[nooutcomes]{ximera}
%% handout
%% space
%% newpage
%% numbers
%% nooutcomes

%\ddx
%\dfn


\newcommand{\RR}{\mathbb R}
\renewcommand{\d}{\,d}
\newcommand{\dd}[2][]{\frac{d #1}{d #2}}
\renewcommand{\l}{\ell}
\newcommand{\ddx}{\frac{d}{dx}}
\newcommand{\dfn}{\textbf}
\newcommand{\eval}[1]{\bigg[ #1 \bigg]}

\usepackage{multicol}

\renewenvironment{freeResponse}{
\ifhandout\setbox0\vbox\bgroup\else
\begin{trivlist}\item[\hskip \labelsep\bfseries Solution:\hspace{2ex}]
\fi}
{\ifhandout\egroup\else
\end{trivlist}
\fi} %% we can turn off input when making a master document

\title{3.3 Rules of Differentiation (Solutions)}  

\begin{document}
\begin{abstract}		\end{abstract}
\maketitle

\section*{Warm up:} 
	
	\begin{freeResponse}
	\end{freeResponse}	
	
	
	
	
	

\section*{Group work:}

%problem 1
\begin{problem}
Use the ``short-cut derivative rules" to compute the derivatives of the following functions:
	
	\begin{enumerate}
	
	%part a
	\item $f(x) = \sqrt{x}$
	
		\begin{freeResponse}
		$f(x) = \sqrt{x} = x^{\frac{1}{2}}.$  So $f'(x) = \frac{1}{2} x^{\frac{1}{2} - 1} = \frac{1}{2} x^{-\frac{1}{2}} = \frac{1}{2\sqrt{x}}$.
		\end{freeResponse}
			
			
	
	%part b
	\item $f(x) = \frac{5}{x^2}$
	
		\begin{freeResponse}
		$f(x) = \frac{5}{x^2} = 5x^{-2}.$  So $f'(x) = 5(-2) x^{-2-1} = -10x^{-3} = \frac{-10}{x^3}$.  
		\end{freeResponse}
			
			
	
	%part c
	\item $f(x) = x^5 + 4x^3 + \pi $
	
		\begin{freeResponse}
		$f'(x) = 5x^{5-1} + 4(3)x^{3-1} + 0 = 5x^4 + 12x^2$.  Note that $\ddx (\pi) = 0$ because $\pi$ is a constant.	
		\end{freeResponse}
			
			
	
	\end{enumerate}
\end{problem}
	
	
	
	
			
			

%problem 2			
\begin{problem}
Find the slope of the function $f(x) = 2x^3 - 5x^2 + 7x - 9$ at $x=3$ two ways.  First, by finding it directly by the limit definition and, secondly, by using the ``derivative short-cut" to find $f'(x)$ and then evaluating it at $x=3$.  Then, find the equation of the tangent line to the graph of $f(x)$ at $x=3$.
		\begin{freeResponse}
		$f'(3) = \lim_{x \to 3} \frac{f(x) - f(3)}{x-3}
		= \lim_{x \to 3} \frac{(2x^3 - 5x^2 + 7x - 9) - (54 - 45 + 21 - 9)}{x-3}
		= \lim_{x \to 3} \frac{(2x^3 - 5x^2 + 7x - 9) - 21}{x-3}
		= \lim_{x \to 3} \frac{2x^3 - 5x^2 + 7x - 30}{x-3}
		= \lim_{x \to 3} \frac{(x-3)(2x^2 + x + 10)}{x-3}
		= \lim_{x \to 3} (2x^2 + x + 10)
		= 18 + 3 + 10 = 31.$
		
		Using the ``short-cut rules", $f'(x) = 6x^2 - 10x + 7$ and so $f'(3) = 6(9) - 10(3) + 7 = 54 - 30 + 7 = 31$.
		
		The tangent line to the graph of $f(x)$ at $x=3$ has slope $m=31$ and goes through the point $(3,21)$.  Thus, the equation of the tangent line is $y-21=31(x-3)$, or $y = 31x -72$. 
		\end{freeResponse}
		
\end{problem}









	
	
	
	
	
	
	
	
	

	










								
				
				
	














\end{document} 


















