\documentclass[handout,nooutcomes]{ximera}
%% handout
%% space
%% newpage
%% numbers
%% nooutcomes

%I added the commands here so that I would't have to keep looking them up
%\newcommand{\RR}{\mathbb R}
%\renewcommand{\d}{\,d}
%\newcommand{\dd}[2][]{\frac{d #1}{d #2}}
%\renewcommand{\l}{\ell}
%\newcommand{\ddx}{\frac{d}{dx}}
%\everymath{\displaystyle}
%\newcommand{\dfn}{\textbf}
%\newcommand{\eval}[1]{\bigg[ #1 \bigg]}

%\begin{image}
%\includegraphics[trim= 170 420 250 180]{Figure1.pdf}
%\end{image}


\newcommand{\RR}{\mathbb R}
\renewcommand{\d}{\,d}
\newcommand{\dd}[2][]{\frac{d #1}{d #2}}
\renewcommand{\l}{\ell}
\newcommand{\ddx}{\frac{d}{dx}}
\newcommand{\dfn}{\textbf}
\newcommand{\eval}[1]{\bigg[ #1 \bigg]}

\renewenvironment{freeResponse}{
\ifhandout\setbox0\vbox\bgroup\else
\begin{trivlist}\item[\hskip \labelsep\bfseries Solution:\hspace{2ex}]
\fi}
{\ifhandout\egroup\else
\end{trivlist}
\fi} %% we can turn off input when making a master document

\title{Recitation \#23 - 5.2 Definite Integrals}  

\begin{document}
\begin{abstract}		\end{abstract}
\maketitle

\section*{Warm up:} 
Evaluate the following sums:
	\begin{enumerate}
	
	%part a 
	\item $\sum_{i=1}^{4} i^5 $
		\begin{freeResponse}
		$\sum_{i=1}^{4} i^5 = 1^5 + 2^5 + 3^5 + 4^5 = 1 + 32 + 243 + 1024 = 1300$.
		\end{freeResponse}	
		
		
		
	%part b
	\item $\sum_{i=1}^{400} (5(i+1)^2 + 3) $
		\begin{freeResponse}
			\begin{align*}
			\sum_{i=1}^{400} (5(i+1)^2 + 3) &= \sum_{i=1}^{400} (5(i^2 + 2i + 1) + 3) \\
			&= \sum_{i=1}^{400} (5i^2 + 10i + 8) \\
			&= \sum_{i=1}^{400} 5i^2 + \sum_{i=1}^{400} 10i + \sum_{i=1}^{400} 8  \\
			&= 5\sum_{i=1}^{400} i^2 + 10 \sum_{i=1}^{400} i + 8(400)  \\
			&= 5 \left( \frac{400(400+1)(2(400) + 1)}{6} \right) + 10 \left( \frac{400(400+1)}{2} \right) + 3,200  \\
			&= 5 (200)(401)(267) + 10(200)(401) + 3,200  \\
			&= 107,067,000 + 802,000 + 3,200 = 107,872,200
			\end{align*}
		\end{freeResponse}	
		
		
		
	\end{enumerate}
		
		
		

	
	
	
	
	

\section*{Group work:}



%problem 1
\begin{problem}
Snow is starting to fall with a rate at any time $t$ after the start being 
$$ f'(t) = \frac{3}{2} t - \frac{1}{4} t^2 + \frac{3}{10} $$
inches per hour for $t$ in $[0,4]$ (i.e., the snow falls for 4 hours - from noon until 4pm).  
There were already $5$ inches of snow on the ground when the storm started.  
	\begin{enumerate}
	
	%part a 
	\item  Use the formula for a right Riemann sum to estimate how much snow fell during the storm using $n$ rectangles.
		\begin{freeResponse}
		$\Delta x = \frac{b-a}{n} = \frac{4-0}{n} = \frac{4}{n}$.
		
		$x_i = a + i \Delta x = 0 + i \frac{4}{n} = \frac{4i}{n}$.
			\begin{align*}
			f'(x_i) = f' \left( \frac{4i}{n} \right) &= \frac{3}{10} + \frac{3}{2} \left( \frac{4i}{n} \right) - \frac{1}{4} \left( \frac{4i}{n} \right)^2  \\
			&= \frac{3}{10} + \frac{6i}{n} - \frac{4i^2}{n^2}
			\end{align*}
			
		So our approximate area is:
			\begin{align*}
			\sum_{i=1}^n f'(x_i) \Delta x &= \sum_{i=1}^n \left[ \left( \frac{3}{10} + \frac{6i}{n} - \frac{4i^2}{n^2} \right) \left( \frac{4}{n} \right) \right]  \\
			&= \frac{4}{n} \sum_{i=1}^n \left( \frac{3}{10} + \frac{6i}{n} - \frac{4i^2}{n^2} \right)  \\
			&= \frac{4}{n} \sum_{i=1}^n \left( \frac{3}{10} \right) + \frac{4}{n} \sum_{i=1}^n \left( \frac{6i}{n} \right) - \frac{4}{n} \sum_{i=1}^n \left( \frac{4i^2}{n^2} \right)  \\
			&= \frac{6}{5n} \sum_{i=1}^n 1 + \frac{24}{n^2} \sum_{i=1}^n i - \frac{16}{n^3} \sum_{i=1}^n i^2  \\
			&= \frac{6}{5n} (n) + \frac{24}{n^2} \left( \frac{n(n+1)}{2} \right) - \frac{16}{n^3} \left( \frac{n(n+1)(2n+1)}{6} \right)  \\
			&= \frac{6}{5} + \frac{12n(n+1)}{n^2} - \frac{8n(n+1)(2n+1)}{3n^3}.
			\end{align*}
		\end{freeResponse}
		
		
		
	%part b
	\item  Take the limit as $n$ goes to infinity to find the exact amount of snow that fell.
		\begin{freeResponse}
			\begin{align*}
			&  \lim_{n \to \infty} \left( \frac{6}{5} + \frac{12n(n+1)}{n^2} - \frac{8n(n+1)(2n+1)}{3n^3} \right)  \\
			&= \lim_{n \to \infty} \left( \frac{6}{5} + \frac{12n^2(1+\frac{1}{n})}{n^2} - \frac{8n^3(1+\frac{1}{n})(2+\frac{1}{n})}{3n^3} \right)  \\
			&= \lim_{n \to \infty} \left( \frac{6}{5} + 12 \left( 1 + \frac{1}{n} \right) - \frac{8(1 + \frac{1}{n})(2 + \frac{1}{n})}{3} \right)  \\
			&= \frac{6}{5} + 12(1 + 0) - \frac{8(1+0)(2+0)}{3}  \\
			&= \frac{6}{5} + 12 - \frac{16}{3} = \frac{18 + 180 - 80}{15} = \frac{118}{15}.
			\end{align*}
		\end{freeResponse}
		
		
		
	\end{enumerate}
		
		
\end{problem}
















%problem 2
\begin{problem}
The \dfn{velocity} function for a man walking along a straight road which runs east and west is given by $v(t) = -t^2 + 4t - 3$ ft/min.
	\begin{enumerate}
	
	%part a 
	\item  Set up a definite integral for the man's \dfn{displacement} during the time interval from $2$ minutes to $6$ minutes after he began running.
		\begin{freeResponse}
			\begin{align*}
			\int_2^6 v(t) \d t &= \lim_{n \to \infty} \sum_{i=1}^n v(x_i) \Delta x
			\end{align*}
		Where:  \\	
		$\Delta x = \frac{b-a}{n} = \frac{6-2}{n} = \frac{4}{n}$.
		
		$x_i = a + i \Delta x = 2 + i \frac{4}{n} = 2 + \frac{4i}{n}$.
		\end{freeResponse}
		
		
		
	%part b
	\item  \dfn{At home:}  Solve the definite integral using the limit of a right Riemann sum.
		\begin{freeResponse}
			\begin{align*}
			v(x_i) &= -\left(2 + \frac{4i}{n} \right)^2 + 4 \left( 2 + \frac{4i}{n} \right) - 3  \\
			&= - \left( 4 + \frac{16i}{n} + \frac{16i^2}{n^2} \right) + 8 + \frac{16i}{n} - 3  \\
			&= 1 - \frac{16i^2}{n^2}
			\end{align*}
			
		So we compute:
			\begin{align*}
			\int_2^6 v(t) \d t &= \lim_{n \to \infty} \sum_{i=1}^n \left[ \left( 1 - \frac{16i^2}{n^2} \right) \left( \frac{4}{n} \right) \right]  \\
			&= \lim_{n \to \infty} \sum_{i=1}^n \left( \frac{4}{n} - \frac{64 i^2}{n^3} \right)  \\
			&= \lim_{n \to \infty} \left[ \frac{4}{n} \sum_{i=1}^n 1 - \frac{64}{n^3} \sum_{i=1}^n i^2 \right]  \\
			&= \lim_{n \to \infty} \left[ \frac{4}{n} (n) - \frac{64}{n^3} \left( \frac{n(n+1)(2n+1)}{6} \right) \right]  \\
			&= 4 - \frac{64}{3} = \frac{12-64}{3} = - \frac{52}{3}.
			\end{align*}
		\end{freeResponse}
		
		
		
	%part c
	\item  Is this the same as the total \dfn{distance} the man walked from $2$ minutes to $6$ minutes?  Why or why not?
		\begin{freeResponse}
		This number is not the same as the total distance.  The man starts his walk by going east (the positive direction) but eventually ends his walk west of where he started.  
		The total distance that the man walks would be measured by computing 
		$$\int_2^6 \left| v(t) \right| \d t$$  
		\end{freeResponse}
		
		
		
	\end{enumerate}
		
		
		

		
		
		

\end{problem}
	
	
	
	
	
	
	
	
			
			

%problem 3
\begin{problem}
Let $f(x)$ and $g(x)$ be functions for which we only know the following:
$$ \int_1^4 f(x)\d x = 7	\qquad	\int_2^4 f(x)\d x = 5	\qquad	\int_1^4 g(x)\d x = 2 $$
Compute the following integrals, if possible.  If it is not possible, give examples explaining why not.
	\begin{enumerate}
	
	%part a 
	\item  $\int_1^4 (8f(x) - 7g(x))\d x $
		\begin{freeResponse}
			\begin{align*}
			\int_1^4 (8f(x) - 7g(x))\d x &= 8 \int_1^4 f(x) \d x - 7 \int_1^4 g(x) \d x  \\
			&= 8(7) - 7(2) \\
			&= 56 - 14 = 42
			\end{align*}
		\end{freeResponse}
		
		
		
	%part b
	\item  $\int_1^2 (-f(x)) \d x $
		\begin{freeResponse}
		First notice that
			\begin{equation*}
			\int_1^4 f(x) \d x - \int_2^4 f(x) \d x = \int_1^2 f(x) \d x.
			\end{equation*}
		So
			\begin{align*}
			\int_1^2 (-f(x)) \d x &= - \int_1^2 f(x) \d x  \\
			&= - \left( \int_1^4 f(x)\d x - \int_2^4 f(x)\d x \right)  \\
			&= - (7 - 5) = -2.
			\end{align*}
			
			%\begin{align*}
			%\int_1^2 (-f(x)) \d x &= - \int_1^2 f(x) \d x  \\
			%&= - \left( \int_1^4 f(x)\d x + \int_4^2 f(x)\d x \right)  \\
			%&= - \left( \int_1^4 f(x)\d x - \int_2^4 f(x)\d x \right)  \\
			%&= - (7 - 5) = -2
			%\end{align*}
		\end{freeResponse}
		
		
		
	%part c
	\item  $\int_1^4 \left| f(x) \right| \d x$
		\begin{freeResponse}
		We are not given enough information to solve this integral because we do not know the regions where $f$ is positive or negative.
		Consider the following two functions $f_1(x)$ and $f_2(x)$:  
		
		$f_1(x) =   \left\{ \begin{array}{cl}
	2		 	&	\qquad \text{if } \quad 1 \leq x < 2					\\
	\frac{5}{2}		&	\qquad \text{if } \quad 2 \leq x \leq 4		\end{array} \right.  $
	
	$f_2(x) =   \left\{ \begin{array}{cl}
	6		 	&	\qquad \text{if } \quad 1 \leq x < 1.5					\\
	-2		 	&	\qquad \text{if } \quad 1.5 \leq x < 2					\\
	\frac{5}{2}		&	\qquad \text{if } \quad 2 \leq x \leq 4		\end{array} \right.  $
	
	Just using geometry, one can check that
	$$ \int_1^4 f_1(x)\d x = 7	\qquad	\int_2^4 f_1(x)\d x = 5	\qquad	\int_1^4 f_2(x)\d x = 7	\qquad	\int_2^4 f_2(x)\d x = 5 $$
	and so both $f_1$ and $f_2$ satisfy the assumptions of $f$.  But notice that
	$$\int_1^4 \left| f_1(x) \right| \d x = 7	\qquad	\text{and}		\qquad	\int_1^4 \left| f_2(x) \right| \d x = 9  $$
	These two examples demonstrate that we were not given enough information to solve this problem.
		
		\end{freeResponse}
		
	%part d
	\item  $\int_1^4 \left( 2 - x + f(x) \right) \d x$
		\begin{freeResponse}
		First notice that since the integral is linear over addition:
			\begin{equation}\label{3d}
			\int_1^4 \left( 2 - x + f(x) \right) \d x = \int_1^4 2 \d x - \int_1^4 x \d x + \int_1^4 f(x) \d x = \int_1^4 2 \d x - \int_1^4 x \d x + 7.
			\end{equation}
		By using geometry, we can see that
			\begin{equation*}
			\int_1^4 2 \d x = 2(4-1) = 6
			\end{equation*}
			\begin{equation*}
			\int_1^4 x \d x = 1(4-1) + \frac{1}{2} (4-1)(4-1) = 3 + \frac{9}{2} = \frac{15}{2}.
			\end{equation*}
		Then substituting into equation \eqref{3d} gives:
			\begin{equation*}
			\int_1^4 \left( 2 - x + f(x) \right) \d x = 6 - \frac{15}{2} + 7 = \frac{11}{2}.
			\end{equation*}
		\end{freeResponse}
		
		
		
	\end{enumerate}
		
			
			
		
\end{problem}


















	
	
	
	
	
	
	
	
	

	










								
				
				
	














\end{document} 


















