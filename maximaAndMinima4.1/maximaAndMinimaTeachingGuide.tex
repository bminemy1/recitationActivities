\documentclass[handout,nooutcomes]{ximera}
%% handout
%% space
%% newpage
%% numbers
%% nooutcomes


\newcommand{\RR}{\mathbb R}
\renewcommand{\d}{\,d}
\newcommand{\dd}[2][]{\frac{d #1}{d #2}}
\renewcommand{\l}{\ell}
\newcommand{\ddx}{\frac{d}{dx}}
\newcommand{\dfn}{\textbf}
\newcommand{\eval}[1]{\bigg[ #1 \bigg]}

\usepackage{multicol}

\renewenvironment{freeResponse}{
\ifhandout\setbox0\vbox\bgroup\else
\begin{trivlist}\item[\hskip \labelsep\bfseries Solution:\hspace{2ex}]
\fi}
{\ifhandout\egroup\else
\end{trivlist}
\fi} %% we can turn off input when making a master document

\title{Recitation \#15 - 4.1 Maxima and Minima (Teaching Guide)}  

\begin{document}
\begin{abstract}		\end{abstract}
\maketitle


\section*{Warm up:} 
	
	\begin{itemize}
	
	\item  \emph{10 minutes}:  Ask students to think about the Warm-up as they are waiting for class to begin.  Then discuss the Warm-Up as a class when class begins.  Do as a whole class discussion, having the students vote on whether they think it is true or false.  Ask a student to explain why for each question.
	
	
	
	\end{itemize}


\section*{Problem 1:}

	\begin{itemize}
	
	\item  \emph{5 minutes}:  Give students a few minutes to think about \#1.  Have a group present their answers.
	
	\end{itemize}



\section*{Problem 2:}

	\begin{itemize}
	
	\item  \emph{10 minutes}:  Split the parts of \#2 between the groups and give them time to work.  Circle around and help if they get stuck, especially on the algebra.
		
	\item  \emph{15 minutes}:  Have the groups present their solutions to \#2, approximately $5$ minutes each.  
			
	\end{itemize}
	
	
	
\section*{Problem 3:}

	\begin{itemize}
	
	\item  \emph{5 minutes}:  Give the students some time to get started on \#3 in their groups.
	
	\item  \emph{10 minutes}:  Have a student present \#3  if there is time, or just go over it as a class.  On the side, you might want to show students where the formula for $P(x)$ comes from if time allows.  Showing a graphical interpretation of the problem might be helpful to students.
	
	\end{itemize}
	
	
	
	

	
	

	
	
	

	
	
	
















\end{document}