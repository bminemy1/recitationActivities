\documentclass[handout,nooutcomes]{ximera}
%% handout
%% space
%% newpage
%% numbers
%% nooutcomes


\newcommand{\RR}{\mathbb R}
\renewcommand{\d}{\,d}
\newcommand{\dd}[2][]{\frac{d #1}{d #2}}
\renewcommand{\l}{\ell}
\newcommand{\ddx}{\frac{d}{dx}}
\newcommand{\dfn}{\textbf}
\newcommand{\eval}[1]{\bigg[ #1 \bigg]}

\usepackage{multicol}

\renewenvironment{freeResponse}{
\ifhandout\setbox0\vbox\bgroup\else
\begin{trivlist}\item[\hskip \labelsep\bfseries Solution:\hspace{2ex}]
\fi}
{\ifhandout\egroup\else
\end{trivlist}
\fi} %% we can turn off input when making a master document

\title{Recitation \#17 - 4.3 Graphing Functions (Teaching Guide)}  

\begin{document}
\begin{abstract}		\end{abstract}
\maketitle


\section*{Warm up:} 
	
	\begin{itemize}
	
	\item  \emph{5 minutes}:  Allow students to think about the warm-up as class starts and then discuss the questions as a class. Take a poll on each part to generate discussion and ask those with differing opinions to explain their reasoning. (For 1 a and c, discuss why these must always be false)
	
	\end{itemize}


\section*{Problem 1:}

	\begin{itemize}
	
	\item  \emph{10 minutes}:  Allow students to work on \#1 in groups. Suggest that they might split the work among their team to make it faster.  
	
	\item  \emph{10 minutes}:  Allow students to present their solutions to \#1.  Let different groups present different parts of the problem if possible.
	
	\end{itemize}
	
	
	
\section*{Problem 2:}

	\begin{itemize}
	
	\item  \emph{20 minutes}:  Start \#2 together as a class, showing students how to carefully organize their work.  This is usually the biggest problem students have with sketching the graph.  It is a good idea to get them to draw the $f'$ and $f''$ number lines directly below their graph, lining up the x-values on all three.  Stop before actually drawing the graph and ask the students to sketch the graph themselves.  Go around and look at all of their graphs.  Point out to them any places they are incorrect or sloppy, especially where their local extreme values do not exactly line up with the $x$-value or where the concavity is not clear.  If you want, tell them you want to make sure their work is clear to graders.  
	
	\end{itemize}
	
	- You can use any extra time to review for the upcoming exam.




	
	
	

	
	

	
	
	

	
	
	
















\end{document}