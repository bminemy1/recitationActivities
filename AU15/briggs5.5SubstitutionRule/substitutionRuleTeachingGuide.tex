\documentclass[handout,nooutcomes]{ximera}
%% handout
%% space
%% newpage
%% numbers
%% nooutcomes


\newcommand{\RR}{\mathbb R}
\renewcommand{\d}{\,d}
\newcommand{\dd}[2][]{\frac{d #1}{d #2}}
\renewcommand{\l}{\ell}
\newcommand{\ddx}{\frac{d}{dx}}
\newcommand{\dfn}{\textbf}
\newcommand{\eval}[1]{\bigg[ #1 \bigg]}

\usepackage{multicol}

\renewenvironment{freeResponse}{
\ifhandout\setbox0\vbox\bgroup\else
\begin{trivlist}\item[\hskip \labelsep\bfseries Solution:\hspace{2ex}]
\fi}
{\ifhandout\egroup\else
\end{trivlist}
\fi} %% we can turn off input when making a master document

\title{Recitation \#27 - 5.5 Substitution Rule (Teaching Guide)}  

\begin{document}
\begin{abstract}		\end{abstract}
\maketitle


\section*{Warm up:} 
	
	\begin{itemize}
	
	\item  \emph{10 minutes}:  Allow students to think about the warm up and have two different students present their work for the two ways to evaluate the integral.
	
	
	
	\end{itemize}


\section*{Problem 1:}

	\begin{itemize}
	
	\item  \emph{5 minutes}:  Split \#1 between the groups.
	
	\item  \emph{10 minutes}:  Let groups present their solutions.  Be sure to emphasize the correct way of showing work for a substitution.  Some students may say that they can just do it in their heads, so it may be worth pointing out that the problems get much more difficult.
	
	\end{itemize}



\section*{Problem 2:}

	\begin{itemize}
	
	\item  \emph{5 minutes}:  Split \#2 between the groups.
		
	\item  \emph{10 minutes}:  Let groups present their solutions.
	
	{\bf Note:} The online lesson teach both ways of doing a substitution with definite integrals:  finding the indefinite integral and then using the antiderivative or changing the limits of integration.  The online lessons conclude that changing the limits of integration is usually quicker/more straightforward.  And I always like to point out that changing the bounds will help them out a lot when they are taking Calculus II.
			
	\end{itemize}
	
	
	
\section*{Problem 3:}

	\begin{itemize}
	
	\item  \emph{5 minutes}:  Split \#3 between the groups.
	
	\item  \emph{10 minutes}:  Let groups present their solutions.
	
	\end{itemize}
	



	
	
	

	
	

	
	
	

	
	
	
















\end{document}