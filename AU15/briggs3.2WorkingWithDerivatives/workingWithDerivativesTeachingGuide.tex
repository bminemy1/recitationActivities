\documentclass[handout,nooutcomes]{ximera}
%% handout
%% space
%% newpage
%% numbers
%% nooutcomes


\newcommand{\RR}{\mathbb R}
\renewcommand{\d}{\,d}
\newcommand{\dd}[2][]{\frac{d #1}{d #2}}
\renewcommand{\l}{\ell}
\newcommand{\ddx}{\frac{d}{dx}}
\newcommand{\dfn}{\textbf}
\newcommand{\eval}[1]{\bigg[ #1 \bigg]}

\usepackage{multicol}

\renewenvironment{freeResponse}{
\ifhandout\setbox0\vbox\bgroup\else
\begin{trivlist}\item[\hskip \labelsep\bfseries Solution:\hspace{2ex}]
\fi}
{\ifhandout\egroup\else
\end{trivlist}
\fi} %% we can turn off input when making a master document

\title{Recitation \#8 - 3.2 Working with Derivatives (Teaching Guide)}  

\begin{document}
\begin{abstract}		\end{abstract}
\maketitle


\section*{Warm up:} 
	
	\begin{itemize}
	
	\item  \emph{5 minutes}:  Discuss warm up \#1 as a class.  Before explaining the answer, be sure to take a poll to see how many students think each part is true or not.  If the class is split, try to get a discussion going.
	
	\item  \emph{5 minutes}:  Discuss warm up \#2 as a class.  Before explaining the answer, be sure to take a poll to see how many students think each part is true or not.  If the class is split, try to get a discussion going.
	
	\end{itemize}


\section*{Problem 1:}

	\begin{itemize}
	
	\item  \emph{5 minutes}:  Discuss as a class.  The point here is that students just need to replace the $3$'s with $x$'s.
			
	\end{itemize}
	
	
	
\section*{Problem 2:}

	\begin{itemize}
	
	\item  \emph{20 minutes}:  \#2 should be done together as a whole class.  This is to set up what is important in finding the graphical representation in \#2

First, where the derivative is 0 (gives $x$-intercepts of $f'$:  The gatekeeper between positive and negative!).  

Second, where $f$ is increasing and decreasing and $f'$ is positive and negative

Third, they should also pay attention to where the graph of $f$ is the steepest before going shallow (or vice versa)- the time at which we ``hit the brakes" or ``step on the gas".  Help them realize that these will be the peaks and valleys for the graph of $f'$.

Fourth, where the values of $f$ are zero, negative, or positive have nothing to do with $f'$ (if we shifted the graph of $f$ up or down 10000 units it wouldn't change the graph of $f'$ at all)
	
	\end{itemize}
	


\section*{Problem 3:}

	\begin{itemize}
	
	\item \emph{5 minutes}:  Allows students to work on \#3 in groups. (Have a student begin writing up their solution to \#3 for presenting) 
	
	\item \emph{5 minutes}:  Have someone come up while the other groups finish up and draw their graph on the board.  You explain their graph or discuss any problem spots.  Be sure to point out that on \#2 and \#3, we're accomplishing precisely the same goal as we did in finding the general derivative - only we're doing so graphically rather than algebraically (two different kinds of representation of the same thing - each with its own advantages and disadvantages).	
	\end{itemize}	
	
	
	
\section*{Problem 4:}

	\begin{itemize}
	
	\item \emph{5 minutes}:  Allows students to work on \#4 in groups. 
	
	\item \emph{5 minutes}:  Discuss the solution to problem 4 as a class, emphasizing the difference between continuity and differentiability.  If you have extra time, have them draw the graph of $g'(x)$ for this problem as well.	
	\end{itemize}	
	
	

	
	
	

	
	
	
















\end{document}