\documentclass[handout,nooutcomes]{ximera}
%% handout
%% space
%% newpage
%% numbers
%% nooutcomes


\newcommand{\RR}{\mathbb R}
\renewcommand{\d}{\,d}
\newcommand{\dd}[2][]{\frac{d #1}{d #2}}
\renewcommand{\l}{\ell}
\newcommand{\ddx}{\frac{d}{dx}}
\newcommand{\dfn}{\textbf}
\newcommand{\eval}[1]{\bigg[ #1 \bigg]}

\renewenvironment{freeResponse}{
\ifhandout\setbox0\vbox\bgroup\else
\begin{trivlist}\item[\hskip \labelsep\bfseries Solution:\hspace{2ex}]
\fi}
{\ifhandout\egroup\else
\end{trivlist}
\fi} %% we can turn off input when making a master document

\title{Recitation \#20 - 4.6 Mean Value Theorem (Teaching Guide)}  

\begin{document}
\begin{abstract}		\end{abstract}
\maketitle


\section*{Warm up:} 
	
	\begin{itemize}
	
	\item  \emph{5 minutes}:  Give the students a couple of minutes to read the warm up and think about it before discussing the problem as a class.
	
	
	
	\end{itemize}


\section*{Problem 1:}

	\begin{itemize}
	
	\item  \emph{5 minutes}:  Allow students to think about problem \#1 and then discuss it as a class.
	
	\end{itemize}



\section*{Problem 2:}

	\begin{itemize}
	
	\item  \emph{10 minutes}:  Let the students work on problem \#2 in groups.
		
	\item  \emph{10 minutes}:  Discuss \#2 as a whole class.  Make sure to make this explanation as rigorous as possible, and be sure to put the statement of the Mean Value Theorem on the board.  Students often do not understand what it means to solve for $c$, they frequently say that $\frac{f(b) - f(a)}{b-a}$ is the answer.
			
	\end{itemize}
	
	
	
\section*{Problem 3:}

	\begin{itemize}
	
	\item  \emph{10 minutes}:  Let students work on \#3 in groups.
	
	\item  \emph{5 minutes}:  Have a group present \#3 if possible.  If not, go over it as a group.
	
	\end{itemize}
	


\section*{Problem 4:}

	\begin{itemize}
	
	\item  \emph{5 minutes}:  Let students try \#4 in groups.  This is often a difficult problem for students because they have trouble comprehending what $f(t)-g(t)$ represents.  Therefore they do not understand what $f'(t)-g'(t)$ represents.
		
	\item  \emph{5 minutes}:  Have a group present \#4 if possible.  If not, go over it as a group.
	\end{itemize}	
	
	
	

	
	

	
	
	

	
	
	
















\end{document}