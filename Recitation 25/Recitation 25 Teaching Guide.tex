\documentclass[handout,nooutcomes]{ximera}
%% handout
%% space
%% newpage
%% numbers
%% nooutcomes


\newcommand{\RR}{\mathbb R}
\renewcommand{\d}{\,d}
\newcommand{\dd}[2][]{\frac{d #1}{d #2}}
\renewcommand{\l}{\ell}
\newcommand{\ddx}{\frac{d}{dx}}
\newcommand{\dfn}{\textbf}
\newcommand{\eval}[1]{\bigg[ #1 \bigg]}

\usepackage{multicol}

\renewenvironment{freeResponse}{
\ifhandout\setbox0\vbox\bgroup\else
\begin{trivlist}\item[\hskip \labelsep\bfseries Solution:\hspace{2ex}]
\fi}
{\ifhandout\egroup\else
\end{trivlist}
\fi} %% we can turn off input when making a master document

\title{Recitation \#25 - 5.3 Fundamental Theorem of Calculus Part II (Teaching Guide)}  

\begin{document}
\begin{abstract}		\end{abstract}
\maketitle


\section*{Warm up:} 
	
	\begin{itemize}
	
	\item  \emph{5 minutes}:  %Ask students to think about the Warm-Up as they are waiting for class to begin.  Then discuss the warm-up as a class when class begins.
	
	
	
	\end{itemize}


\section*{Problem 1:}

	\begin{itemize}
	
	\item  \emph{10 minutes}:  Divide the problems in \#1 between groups and let students work in groups to solve the problems.
	
	\item  \emph{15 minutes}:  Have students present the solutions to the problems in \#1 
	
	\end{itemize}



\section*{Problem 2:}

	\begin{itemize}
	
	\item  \emph{5 minutes}:  Allow students to work on \#2 in groups.  Split the two parts between the groups.
		
	\item  \emph{5 minutes}:  Let groups present their solutions to each part.
			
	\end{itemize}
	
	
	
\section*{Problem 3:}

	\begin{itemize}
	
	\item  \emph{10 minutes}:  Allow students to work on \#3 in groups.  
	
	\item  \emph{5 minutes}:  Allow a group to present their solutions to \#3.
	
	\end{itemize}
	


\section*{Problem 4:}

	\begin{itemize}
	
	\item  \emph{5 minutes}:  Allow students to work on \#4 in groups.  
		
	\item  \emph{5 minutes}:  Discuss the various parts to problem 4 as a class.  The solutions contain a cute story which makes the problem a little more tangible.  I read it to my class last semester and I think they both enjoyed it (in a humerous way) and it helped them understand what is going on as well.  But feel free to skip the story if you do not like it.	
	\end{itemize}	
	
	
	

	
	

	
	
	

	
	
	
















\end{document}