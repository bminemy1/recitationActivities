\documentclass[handout,nooutcomes]{ximera}
%% handout
%% space
%% newpage
%% numbers
%% nooutcomes


\newcommand{\RR}{\mathbb R}
\renewcommand{\d}{\,d}
\newcommand{\dd}[2][]{\frac{d #1}{d #2}}
\renewcommand{\l}{\ell}
\newcommand{\ddx}{\frac{d}{dx}}
\newcommand{\dfn}{\textbf}
\newcommand{\eval}[1]{\bigg[ #1 \bigg]}

\usepackage{multicol}

\renewenvironment{freeResponse}{
\ifhandout\setbox0\vbox\bgroup\else
\begin{trivlist}\item[\hskip \labelsep\bfseries Solution:\hspace{2ex}]
\fi}
{\ifhandout\egroup\else
\end{trivlist}
\fi} %% we can turn off input when making a master document

\title{4.7 L'Hospitals Rule (Teaching Guide)}  

\begin{document}
\begin{abstract}		\end{abstract}
\maketitle


\section*{Warm up:} 
	
	\begin{itemize}
	
	\item  \emph{5 minutes}:  Ask students to think about the warm-up as they are waiting for class to begin.  Then discuss the warm-up as a class when class begins.  Review the indeterminant forms that may require L'Hospital's Rule, and use the warm-up to review the statement of L'Hospital's Rule and when it can be used.
	
	
	
	\end{itemize}


\section*{Problem 1:}

	\begin{itemize}
	
	\item  \emph{5 minutes}:  Use 1(a) as a demonstration to model to the students what they are supposed to be doing.  Then give the students some time to try to determine the form for each limit.
	
	\item  \emph{5 minutes}:  Go over the form for each limit, explaining the reasoning to get the answer. Discuss the use of arrow or limit notation instead of using an equals sign inappropriately.  It would be nice to have students present here, but there likely won’t be enough time.
	
	\end{itemize}



\section*{Problem 2:}

	\begin{itemize}
	
	\item  \emph{10 minutes}:  Split the problems up between the groups for \#2 and give them time to work.  If they finish, they should work on the other parts.  Be prepared for students to be confused on why L'Hospital's Rule is not working for them on problem 2(a).
		
	\item  \emph{10 minutes}:  Have students present their solutions from \#2
			
	\end{itemize}
	
	
	

	
	
	

	
	

	
	
	

	
	
	
















\end{document}