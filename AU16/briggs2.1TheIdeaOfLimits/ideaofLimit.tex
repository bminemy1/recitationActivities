\documentclass[nooutcomes]{ximera}
%% handout
%% space
%% newpage
%% numbers
%% nooutcomes


\newcommand{\RR}{\mathbb R}
\renewcommand{\d}{\,d}
\newcommand{\dd}[2][]{\frac{d #1}{d #2}}
\renewcommand{\l}{\ell}
\newcommand{\ddx}{\frac{d}{dx}}
\newcommand{\dfn}{\textbf}
\newcommand{\eval}[1]{\bigg[ #1 \bigg]}

\usepackage{multicol}

\renewenvironment{freeResponse}{
\ifhandout\setbox0\vbox\bgroup\else
\begin{trivlist}\item[\hskip \labelsep\bfseries Solution:\hspace{2ex}]
\fi}
{\ifhandout\egroup\else
\end{trivlist}
\fi} %% we can turn off input when making a master document

\usepackage{fullpage}


\title{Recitation \ 2.1:  The Idea of Limits}  

\begin{document}
\begin{abstract}		\end{abstract}
\maketitle


%problem 1
 \begin{problem} \hfil

	\begin{enumerate}

		\item What does a secant line to a linear function look like?  What does a tangent line to a linear function look like? 

 \begin{freeResponse}		 
	
	A (non-vertical) line can be represented by a function $L$ defined by $L(x) = mx + b$, where $m$ is the slope of the line and $b$ is the $y$-intercept.
      
        For instance, here is the graph of a particular linear function
        \begin{image}
          \includegraphics[scale = 0.7]{"Graph of linear function".png}
        \end{image}

        If we draw a secant line between two points of this graph we have
        \begin{image}
          \includegraphics[scale = 0.7]{"Graph of linear function with secant line".png}
        \end{image}
        So the secant line is identical to the line itself.

        If we draw a tangent line at one point on this graph we have
        \begin{image}
          \includegraphics[scale = 0.7]{"Graph of linear function with tangent line".png}
        \end{image}
        So the tangent line is identical to the line itself.
\end{freeResponse}


		\item  What might a secant line and tangent line of the function $f$, defined by $f(x) = x^2$, look like?

\begin{freeResponse} \hfil
	\begin{image}
           \includegraphics[scale = 0.8]{"Graph of quadratic function with negative slope secant line".png}
         \end{image}
         \begin{image}
           \includegraphics[scale = 0.8]{"Graph of quadratic function with zero slope secant line".png}
         \end{image}
         \begin{image}
           \includegraphics[scale = 0.8]{"Graph of quadratic function with positive slope secant line".png}
         \end{image}
          \begin{image}
            \includegraphics[scale = 0.8]{"Graph of quadratic function with negative slope tangent line".png}
          \end{image}
          \begin{image}
            \includegraphics[scale = 0.8]{"Graph of quadratic function with zero slope tangent line".png}
          \end{image}
          \begin{image}
            \includegraphics[scale = 0.8]{"Graph of quadratic function with positive slope tangent line".png}
          \end{image}
          There is an important difference between secant lines and tangent lines!
          When we zoom in enough, \emph{at an appropriate point}, the tangent line looks \emph{nearly} indistinguishable from the graph itself:
          \begin{image}
            \includegraphics[scale = 0.3]{"Graph of zoomed in quadratic function".png}
          \end{image}
          Secant lines usually don't have this property.
	\end{freeResponse}

		\item In the graph below, is the given line a secant line or a tangent line?
		\begin{image}
		 \includegraphics[scale = 0.5]{"A secant or tangent line".png}
		\end{image}

	\begin{freeResponse}
	This is a trick question!

        The given line is a tangent line---when we zoom in enough the graph is nearly indistinguishable from its tangent line at that point.
        But, it can also be considered a secant line---it intersects the graph at two points.

        By convention however, since we have drawn the graph by emphasizing only one point of intersection we usually interpert such a line as a tangent line.
	      \end{freeResponse}
	\end{enumerate}

\end{problem}

\begin{problem}
 Part of the given graph can be used to model to ``position-time''graph of a ball thrown straight up into the air.
  Use this graph, and the given function, to answer the following questions.
  \begin{image}
    \includegraphics[scale = 0.3]{"Graph of polynomial function".png}
  \end{image}
	
	
		\begin{enumerate}
			
		 \item  What are the units on the $t$ axis?  What are the units on the $y$ axis?
		 \begin{freeResponse}		 
	The units on the $t$ axis are ``seconds'' (for time), while the units on the $f(t)$ axis are ``feet'' (for height).        
		\end{freeResponse}
			
		\item  If you were watching a movie of the ball being thrown, is the graph a picture of the path that the ball follows?  Why or why not?
		\begin{freeResponse}		 
	 No, the position-time graph is \emph{not} the path the ball follows.
        The graph shows the height of the ball at a given \emph{time}.
        The ball is thrown straight up and has no horizontal movement.
		\end{freeResponse}
	

		\item  What is the domain of the position-time graph of the ball?

		 \begin{freeResponse}
     	   The domain of $f$ is the interval $[0, 9]$.
     	   With this domain the position-time graph of the ball is given by
       		 \begin{center}
       	  	 \includegraphics[scale = 0.4]{"Position time graph of ball".png}
        		\end{center}
     	 \end{freeResponse}

		\item  When will the ball hit the ground?
		\begin{freeResponse}		 
		The ball will hit the ground when the height $f(t)$ equals zero.
			\begin{align*}
			0&=-16t^2+128t+144 \\
			0&=-16(t^2-8t-9) \\
 			0&=-16(t+1)(t-9) \\
 			t&=-1,9 
 			\end{align*}
		The ball will hit the ground at $t=9$ or 9 seconds after the ball is thrown into the air.  We do not use $t=-1$ since we can't have negative time. (In part c, we just excluded $t=-1$ from our domain).
		\end{freeResponse}
		
		\item Use the table of values to find the average velocity of the ball between $t=8.9$ and $t=9$ seconds.
	\begin{center}
	\begin{tabular}{|l|l|}
			\hline
			\text{$t$} & \text{$\approx f(t)$}  \\
			\hline
			$8.9$ & $15.84$  \\
			\hline
			$8.99$ & $1.6$  \\
			\hline
			$8.999$ & $0.159984$  \\
			\hline
			$8.9999$ &  $0.015998$  \\
			\hline
			$9$ &  $0$  \\
			\hline
			\end{tabular}
		\end{center}
  \begin{freeResponse}
    The average velocity of the ball between $t = 8.9$ seconds and $t = 9$ seconds is
    \[
       \frac{f(9) - f(8.9)}{9- 8.9} = \frac{0- 15.84}{0.1} = -158.4
    \]
  \end{freeResponse}


		\item  Use the table of average velocities to approximate the instantaneous velocity of the ball when it hits the ground.
			\begin{center}	 
			\begin{tabular}{|l|l|}
			\hline
			\text{Time Interval} & \text{Average Velocity}  \\
			\hline
			[8.9, 9] & $\frac{f(9)-f(8.9)}{.1}=\frac{0-15.84}{.1}=-158.4$  \\
			\hline
			[8.99,9] & $\frac{f(9)-f(8.99)}{.01}=\frac{0-1.5984}{.01}=-159.84$  \\
			\hline
			[8.999, 9] & $\frac{f(9)-f(8.999)}{.001}=\frac{0-.159984}{.001}=-159.984$  \\
			\hline
			[8.9999, 9] &  $\frac{f(9)-f(8.999)}{.0001}=\frac{0-.0159998}{.0001}=-159.998$  \\
			\hline
			\end{tabular}
			\end{center}

		\begin{freeResponse}
		 The instantaneous velocity of the ball hitting the ground appears to be $-160$ ft/sec.
		\end{freeResponse}
		
		
			
		\item    Use the  graph to determine if the ball has instantaneous velocity equal to 0.  Why or why not?

		\begin{freeResponse}		 
		 The ball has zero instantaneous velocity when it has a tangent line with zero slope:
        \begin{image}
          \includegraphics[scale = 0.5]{"Position time graph with zero velocity".png}
        \end{image}
		\end{freeResponse}
		
		
		
		\item  For which times is the instantaneous velocity of the ball negative?
      What happens to the height of the ball when its velocity is negative?
      \begin{freeResponse}
        The instantaneous velocity of the ball is negative for $t > 4$:
        \begin{image}
          \includegraphics[scale = 0.5]{"Position time graph with negative velocity".png}
        \end{image}
        The height of the ball is decreasing at those times.
      \end{freeResponse}
			
		\end{enumerate}
			
\end{problem}
			
\begin{problem} This was an exam question in Autumn 2015.
			
 The position, $s(t)$, of an object moving along a horizontal line is given by $s(t) = t^2 - 4$.
  \begin{enumerate}
    \item
      Mark the position of the object on the line at time $t = 1$:
\begin{image}
        \includegraphics[scale = .5]{"Blank number line".png}
      \end{image}      

	\begin{freeResponse}
	$s(1)=1^2-4=-3$
	
	\begin{image}
        \includegraphics[scale = 1]{"Position of object at time 1".png}
      \end{image}
	\end{freeResponse}


    \item
      Find the average velocity, $v_{\mathrm{AV}}$, of the object during the time interval $[1, 3]$.
      \begin{freeResponse}
        The average velocity over $[1, 3]$ is
        \[
          \frac{s(3) - s(1)}{3-1}  = \frac{5 - (-3)}{2} = \frac{8}{2} = 4.
        \]
      \end{freeResponse}


    \item
      Compute the average velocity, $v_{\mathrm{AV}}(t)$, of the object during the time interval
      \begin{enumerate}
        \item
          $[1, t]$, for $t > 1$;
          \begin{freeResponse}
            The average velocity over $[1, t]$ is
            \begin{align*}
              \frac{s(t) - s(1)}{t-1}  &= \frac{(t^2-4) - (-3)}{t-1}\\
              &= \frac{t^2-1}{t-1} = t+1.
            \end{align*}
          \end{freeResponse}

        \item
          $[t, 1]$, for $0 < t < 1$.
          \begin{freeResponse}
            The average velocity over $[t, 1]$ is
            \begin{align*}
              \frac{s(1) - s(t)}{1-t}  &= \frac{(-3) - (t^2-4)}{1-t}\\
              &= \frac{1-t^2}{1-t} = 1+t.
            \end{align*}
          \end{freeResponse}
      \end{enumerate}

    \item 
      Find the instantaneous velocity, $v_{\mathrm{inst}}$, of the object at $t = 1$.
      Justify your answer.
      \begin{freeResponse}
        The instantaneous velocity of the object at $t = 1$ is
        \begin{align*}
          v_{\mathrm{inst}} &= \lim_{t \to 1} \frac{s(t) - s(1)}{t-1} \\
          &= \lim_{t \to 1} t+1 = 2.
        \end{align*}
      \end{freeResponse}


    \item
      The position-time graph of the function $s$ is given in the figure below.
      \begin{image}
        \includegraphics[scale = 1]{"Position time graph of object".png}
      \end{image}
      \begin{enumerate}
        \item
          Assume $P$ is a point on the graph of $s$.
          Fill in the blank.
          \[
            P = (1, \mbox{\underline{\hspace{2em}}}).
          \]
          \begin{freeResponse}
            $P = (1, \mbox{\underline{$-3$}})$
          \end{freeResponse}


        \item
          Plot the point $P$ and draw the tangent line at this point in the figure above.
          \begin{freeResponse}  \hfil
            \begin{image}
              \includegraphics[scale = 1]{"Tangent at time 1".png}
            \end{image}
          \end{freeResponse}


        \item
          Find the slope, $m_{\mathrm{tan}}$, of the tangent line in part (b).
          Explain.
          \begin{freeResponse}
            The slope of the tangent line at $t = 1$ is the same as the instantaneous velocity at $t = 1$.
            Therefore $m_{\mathrm{tan}} = v_{\mathrm{inst}} = 2$.
          \end{freeResponse}
      \end{enumerate}


  \end{enumerate}
\end{problem}			










								
				
\end{document} 


















