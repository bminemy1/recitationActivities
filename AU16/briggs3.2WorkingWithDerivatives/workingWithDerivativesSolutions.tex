\documentclass[nooutcomes]{ximera}
%% handout
%% space
%% newpage
%% numbers
%% nooutcomes

\usepackage{fullpage}
\newcommand{\RR}{\mathbb R}
\renewcommand{\d}{\,d}
\newcommand{\dd}[2][]{\frac{d #1}{d #2}}
\renewcommand{\l}{\ell}
\newcommand{\ddx}{\frac{d}{dx}}
\newcommand{\dfn}{\textbf}
\newcommand{\eval}[1]{\bigg[ #1 \bigg]}

\renewenvironment{freeResponse}{
\ifhandout\setbox0\vbox\bgroup\else
\begin{trivlist}\item[\hskip \labelsep\bfseries Solution:\hspace{2ex}]
\fi}
{\ifhandout\egroup\else
\end{trivlist}
\fi} %% we can turn off input when making a master document

\title{Section 3.2 Working with Derivatives}  

\begin{document}
\begin{abstract}		\end{abstract}
\maketitle

%problem1
\begin{problem}
	
	\begin{enumerate}

    \item
   
      If $f'(2)$ exists, then
      \begin{enumerate}
        \item
          $\lim_{x \to 2} f(x)$ must exist, but more information is needed if we want to find it.

        \item
          $\lim_{x \to 2} f(x) = f(2)$.

        \item
          $\lim_{x \to 2} f(x) = f'(2)$

        \item
          $\lim_{x \to 2} f(x)$ need not exist.
      \end{enumerate}
      \begin{freeResponse}
        The correct answer is (ii):
        \begin{align*}
          \mbox{$f'(2)$ exists} &\iff \mbox{$f$ differentiable at $x = 2$}\\
                                &\implies \mbox{$f$ continuous at $x = 2$}\\
                                &\iff \lim_{x \to 2} f(x) = f(2)
        \end{align*}
      \end{freeResponse}


    \item

      Assuming that $\lim_{x \to 0} \frac{\sin x}{x} = 1$, we can conclude
      \begin{enumerate}
        \item
          $\frac{0}{0} = 1$
        \item
          the tangent line to $y = \sin x$ at $(0,0)$ has slope $1$.

        \item
          you can cancel the $x$'s.

        \item
          for all $x$ near $0$, $\sin x = x$.

        \item
          for all $x$ near $0$, $\sin x \approx x$.
      \end{enumerate}
      \begin{freeResponse}
        This problem has two correct answers: (ii) and (v).

        Statement (ii) is true:
        \begin{align*}
          f'(0) &= \lim_{h \to 0} \frac{\sin (h + 0) - \sin (0)}{h} \\
                &= \lim_{h \to 0} \frac{\sin (h)}{h} \\
                &= 1\\
          &\implies \mbox{slope of tangent line at $(0,0)$ is $1$}
        \end{align*}

        Statement (v) is true:
        When $x$ is near $0$ the tangent line $y = x$ is a good approximation to $f$.
      \end{freeResponse}

  \end{enumerate}

	
\end{problem}	
	
	
%problem 2			
\begin{problem}
  Suppose we are given the graph of a function $f$:
  \begin{image}
    \includegraphics[scale=.5]{Figure1.png}
  \end{image}
  \begin{enumerate}
    \item
      Using this graph find the following.  Assume all values will be integers or $+\infty$ or $-\infty$
      \begin{enumerate}
        \item 
          all $x$ where $f(x) = 0$,
          \begin{freeResponse}
            $f(x)$ is zero when the function crosses the $x$-axis.
            Therefore $f(x) = 0$ when $x = -5, x=-1$, and $x = 5$.
          \end{freeResponse}

        \item 
          all $x$ where $f(x) > 0$, 
          \begin{freeResponse}
            $f(x)$ is positive when the function is above the $x$-axis.
            Therefore $f(x) > 0$ on $(-5,-1) \cup (5,\infty)$. 
          \end{freeResponse}

        \item
          all $x$ where $f(x) < 0$, and
          \begin{freeResponse}
            $f(x)$ is negative when the function is below the $x$-axis.
            Therefore $f(x) < 0$ on $(-\infty ,-5) \cup (-1,5)$.
          \end{freeResponse}

        \item
          all $x$ where $f(x)$attains a local maximum and all $x$ where $f(x)$ attains a local minimum.
          \begin{freeResponse}
            $f(x)$ has a local maximum at $x=-3$.
            $f(x)$ has a local minimum at $x=2$.
          \end{freeResponse}
      \end{enumerate}

    \item

      Without sketching the graph of $f'$ find
      \begin{enumerate}
        \item 
          all $x$ where $f'(x) = 0$,
          \begin{freeResponse}
            $f'(x)$ is zero when the tangent line has a slope of zero, which is approximately at $x=-3$ and $x=2$.
            Note, for this question, these are the same answers as the (local) highest and lowest point for the graph of $f$.   
          \end{freeResponse}

        \item
          all $x$ where $f'(x) > 0$,
          \begin{freeResponse}
            ${f}'(x)$ is positive when the slope of the tangent line is positive.
            Observe that the graph of $f$ is increasing on $(-\infty ,-3), (2,\infty)$ and this same union of intervals is where the tangent lines have positive slope.
            Therefore $f'(x) > 0$ on $(-\infty ,-3), (2,\infty)$.
          \end{freeResponse}
        
        \item
          all $x$ where $f'(x) < 0$, and
          \begin{freeResponse}
            ${f}'(x)$ is negative when the slope of the tangent line is negative.
            Observe that the graph of $f$ is decreasing on $(-3,2)$ and this interval is where the tangent lines have negative slope.
            Therefore $f'(x) < 0$ on $(-3,2)$.
          \end{freeResponse}

        \item
          On the following intervals, is $f'(x)$ increasing or decreasing?
          \begin{freeResponse}
            \begin{enumerate}
		\item $(-\infty,0)$
		\item $(0,\infty)$
	\end{enumerate}
          \end{freeResponse}
      \end{enumerate}

    \item
      Determine where $f$ is the steepest.
      (What does this mean in terms of $f'$?)
      \begin{freeResponse}
        $f(x)$ is steepest at approximately $x=2$ and $x = 8.5$.
        As we observed in part (b), this means that the slope of the tangent line at these points are either a small negative number or large positive number.
      \end{freeResponse}

    \item
      Sketch a graph of $f'$.
      \begin{freeResponse}
        The graph of $f'$ is approximately
        \begin{image}
          \includegraphics[scale = 0.5]{Figure2.png}
        \end{image}
      \end{freeResponse}
  \end{enumerate}
\end{problem}





%problem 3
\begin{problem}
  Use the graph of $g$
			\begin{image}
	 		\includegraphics{Figure3.png}
			\end{image}
			

	\begin{enumerate}
	
	\item Find the values of $t$ in $(0,4)$ at which $g$ is not continuous.
		\begin{freeResponse}
		$g$ is not continuous at $t=1$.
		\end{freeResponse}
		
		
	
	\item Find the values of $t$ in $(0,4)$ at which $g$ is not differentiable.
		\begin{freeResponse}
		$g$ is not differentiable at $t=1$ and $t=2$.
		\end{freeResponse}
		
		\end{enumerate}



\end{problem}
	

%problem4
\begin{problem}

	 Given the following graph of a function $h$ sketch a graph of the derivative $h'$.
  \begin{image}
    \includegraphics[scale=.7]{Figure4.png}
  \end{image}
  \begin{freeResponse}
    The graph of the derivative is in red.
    \textbf{Important Note:} Despite being drawn on the same graph, the ``units'' for $f$ and $f'$ are \emph{not} the same!
    \begin{image}
      \includegraphics[scale = 0.7]{Figure5.png}
    \end{image}
  \end{freeResponse}
\end{problem}

%problem5
\begin{problem}
  \mbox{}
  \begin{enumerate}
    \item


      Fill in the blanks
      \[
        f'(x) = \lim_{\text{???}} \frac{\text{????}}{h}
      \]
      if the limit exists.
	\begin{freeResponse}
      \[
        f'(x) = \lim_{h \to 0} \frac{f(x + h) - f(x)}{h}
      \]
      if the limit exists.
	\end{freeResponse}
    \item
      Let 
      \[
        f(x) = \frac{1}{x + 4}.
      \]
      Use the (limit) \emph{definition} of derivative in (a) to find $f'(x)$.
      \begin{freeResponse}
        \begin{align*}
          f'(x) &= \lim_{h \to 0} \frac{f(x + h) - f(x)}{h} = \lim_{h \to 0} \frac{\frac{1}{x+h +4}-\frac{1}{x+4}}{h} \\
          &= \lim_{h \to 0} \frac{\frac{x+4 -(x+h +4)}{(x+h +4)(x+4)}}{h}\\
          &= \lim_{h \to 0} \frac{1}{h} \cdot \frac{-h}{(x+h +4)(x+4)} \\
          &= \lim_{h \to 0} \frac{-1}{(x+h +4)(x+4)} = \frac{-1}{(x+4)^2}
        \end{align*}
      \end{freeResponse}
  \end{enumerate}
\end{problem}

	
	
	
	


\end{document} 


















