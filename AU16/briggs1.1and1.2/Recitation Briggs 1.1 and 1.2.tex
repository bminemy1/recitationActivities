\documentclass[handout]{ximera}
%handout:  for handout version with no solutions or instructor notes
%handout,instructornotes:  for instructor version with just problems and notes, no solutions
%noinstructornotes:  shows only problem and solutions

%% handout
%% space
%% newpage
%% numbers
%% nooutcomes



%\begin{image}
%\includegraphics{Figure1.pdf}
%\end{image}

%add a ``.'' below when used in a specific directory.
\newcommand{\RR}{\mathbb R}
\renewcommand{\d}{\,d}
\newcommand{\dd}[2][]{\frac{d #1}{d #2}}
\renewcommand{\l}{\ell}
\newcommand{\ddx}{\frac{d}{dx}}
\newcommand{\dfn}{\textbf}
\newcommand{\eval}[1]{\bigg[ #1 \bigg]}

\renewenvironment{freeResponse}{
\ifhandout\setbox0\vbox\bgroup\else
\begin{trivlist}\item[\hskip \labelsep\bfseries Solution:\hspace{2ex}]
\fi}
{\ifhandout\egroup\else
\end{trivlist}
\fi} %% we can turn off input when making a master document

\usepackage{fullpage}

\title{1.1 and 1.2:  Review of Functions}  

\begin{document}
\begin{abstract}		\end{abstract}
\maketitle






%problem 1
\begin{problem}
Define 
	$f(x) =   \left\{ \begin{array}{cl}
	x^2-1		 	&	\qquad \text{if } x > 0					\\
	\text{? }			&	\qquad \text{if }  x < 0 	\\		\end{array} \right.  $
\begin{enumerate}	
	\item  Find an expression for "?" such that $f$ will be even.
	
	\begin{freeResponse} 
			If $f$ is even then $f(x)=f(-x)$.  $f(-x)$ represents the part of the graph less than zero (which is the part we are trying to find).  Lets call $f(-x)=g(x)$ where $g(x)$ is 			the blank we are trying to fill in.  Then we need $f(x)=g(x)$.  So ? must equal $x^2-1$
	\end{freeResponse}
	
	\item  Find an expression for "?" such that $f$ will be odd.
	
	\begin{freeResponse}
	   If $f$ is odd then $f(-x)=-f(x)$.  Using the same arguement as in part a, we'll allow $g(x)=f(-x)$.  So we need $g(x)=-f(x)$.  So ? must equal $-(x^2-1) =-x^2+1$.  Graphically, this 		can be seen in the figure below.

	\begin{image}		
	\includegraphics{Figure3.jpg}
	\end{image}
		

	\end{freeResponse}


	\end{enumerate}
	
	
	
	
\end{problem}

\begin{instructorNotes}

\end{instructorNotes}




%problem 2
\begin{problem}
The graph of $g(x)=e^x$ is given below.

	\begin{image}		
	\includegraphics{Figure1.jpg}
	\end{image}

\begin{enumerate}	
	\item  Find the domain and range of $g$.
		\begin{freeResponse}
		Domain: $(-\infty,\infty)$, Range: $(0,\infty)$
		\end{freeResponse}


	
	\item  Find the values of $g(1), g(0), g(-1)$ and plot them on the graph below.
		\begin{freeResponse}
	
			$$g(1)=e^1=e$$
			$$ g(0)=e^0=1$$ 
			$$g(-1)=e^{-1}=\frac{1}{e}$$
			 These are values of $g(x)=e^x$ that you should have memorized

		\begin{image}		
		\includegraphics{Figure5.png}
		\end{image}


		\end{freeResponse}
	\item Graph $h(x)=ln(x)$ on the same axis below.
		\begin{freeResponse}
		Recall:  $ln(x)$ is the inverse of $e^x$.  To find the graph of $ln(x)$ we reflect the graph of $e^x$ over the line $y=x$

		\begin{image}		
		\includegraphics{Figure4.png}
		\end{image}
		\end{freeResponse}

	\item Find the domain and range of $h$.
		\begin{freeResponse}
		The domain of $h(x)=ln(x)$ is $(0,\infty)$.  The range is $(-\infty,\infty)$.
		\end{freeResponse}

	\item Find the values of $h(1), h(0), h(-1)$, or say $x$ not in the domain.
			\begin{freeResponse}
			 $$h(1)=ln(1)=e$$
			$$ h(0)=ln(0)=1$$
			$$ h(-1) \text{is not in the domain}$$
			$$g(-1)=e^{-1}=\frac{1}{e}$$
			 These are values of $h(x)=ln(x)$ that you should have memorized

		\end{freeResponse}
	\end{enumerate}
	
 		
		
	
\end{problem}

\begin{instructorNotes}

\end{instructorNotes}



%problem 3
\begin{problem}
Given $y(t)=t- \frac{\pi}{3}$ and $w(t)=\sin(t)$.  Find:
\begin{enumerate}	
	\item  $y(w(t))$
		\begin{freeResponse}
			$y(w(t))=y\left( \sin(t) \right)=\sin(t)-\frac{\pi}{3}$
		\end{freeResponse}	


	\item  $w(y(t))$
		\begin{freeResponse}
		$w(y(t))=w\left( t-\frac{\pi}{3}\right)$
		\end{freeResponse}	


	\item  $w \left(y \left(\frac{4\pi}{3} \right)\right)$
		\begin{freeResponse}
		$w \left(y \left(\frac{4\pi}{3} \right)\right)=\sin \left(\frac{4\pi}{3}-\frac{\pi}{3}\right)=sin(\pi)=0$
		\end{freeResponse}	


	\item  $y(w(\frac{4\pi}{3}))$
		\begin{freeResponse}
		$w \left(y \left(\frac{4\pi}{3} \right)\right)=\sin \left(\frac{4\pi}{3}\right)-\frac{\pi}{3}=\frac{\sqrt{3}}{2}-\frac{\pi}{3}$

		You should know values of $sin(x)$ and $cos(x)$ for all values found on the unit circle.
		\end{freeResponse}	
	
	\end{enumerate}
	
	
\end{problem}

\begin{instructorNotes}

\end{instructorNotes}

%problem 4
\begin{problem}
Define 
	$g(x) =   \left\{ \begin{array}{cl}
	|x-2|		 	&	\qquad \text{if } x < 0					\\
	\cos(x)			&	\qquad \text{if }  x \geq 0  	\\		\end{array} \right.  $
\begin{enumerate}	
	\item  Sketch a graph of $g$
		\begin{freeResponse} \hfil
			\begin{image}			
			\includegraphics{Figure6.png}
			\end{image}

		\end{freeResponse}

	\item  Find the domain and range of $g$
		\begin{freeResponse}	
			Domain: $(-\infty,\infty)$, Range: $[-1,1]\cup[2,\infty)$
		\end{freeResponse}		
	
	\item  Find the values of $g(\pi)$ and $g(-\pi)$
		\begin{freeResponse}	
			$g(\pi)=\cos (\pi)=1$ and $g(-\pi)=|-\pi-2|=\pi+2$
		\end{freeResponse}
	
	\end{enumerate}

\end{problem}

\begin{instructorNotes}

\end{instructorNotes}

%problem 5
\begin{problem}
The entire graph of $f(x)$ is given below.

	\begin{image}
	\includegraphics{Figure2.png}
	\end{image}

\begin{enumerate}	
	\item  Find the domain and range of $f$
		\begin{freeResponse}
			Domain: $[-4,-3)\cup{-2}\cup(-1,4]$
			Range: $[-4,-3)\cup{-2}\cup[-1,3]$
		\end{freeResponse}	

	\item  Find the values of $f(-3),f(-2), f(-1),f(2)$
		\begin{freeResponse}
		$f(-3)=0, f(-2)=-2, f(-1) \text{Does not exist, not in the domain}, f(2)=-1$
		\end{freeResponse}	

	\item  Find the intervals on which $f(x)$ is postive.  Find the intervals on which $f(x)$ is negative.
		\begin{freeResponse}
		 $f(x)$ is postive on $(1,2)$. $f(x)$ is negative on $[-4,-3),{-2},(-1,1),[2,4)$
		\end{freeResponse}
	
	\item Find the intervals on which $f$ is increasing.  Find the intervals on which $f$ is decreasing.
		\begin{freeResponse}
		$f(x)$ is increasing on $(-4,-3)\cup(0,2)\cup(2,4)$.  $f(x)$ is decreasing on $(-1,0)$
		\end{freeResponse}
	
	\item True or False: $f(1.5) < f(2)$
		\begin{freeResponse}
		False, $f(2) < f(1.5)$
		\end{freeResponse}	
	
	\end{enumerate}

	
\end{problem}

\begin{instructorNotes}

\end{instructorNotes}

%problem 6
\begin{problem}
Determine if the function is even, odd, or neither.


\begin{enumerate}	
	\item  $h(x)=\cos(x)$
		\begin{freeResponse}

		A function is even if $f(x)=f(-x)$ which means it is symmetric about the y-axis.  A function is odd if $f(-x)=-f(x)$ which means it is symmetic about the origin. 
		 $$h(x)=cos(x)$$
			$$h(-x)=cos(-x)$$
			It is a property of $cos(x)$ that $cos(x)=cos(-x)$.  Hence $h(x)=cos(x)$ is even.  This can be verified by graphing $h(x)=cos(x)$ and seeing that it is symmetric about the y-axis.
		\end{freeResponse}

	\item  $s(t)=t^2-t$
		\begin{freeResponse}
			$$s(t)=t^2-t$$
			$$s(-t)=(-t)^2-(-t)=t^2+t$$ This does not equal $s(t)$ so $s(t)$ is not even.
			$$-s(t)=-(t^2-t)=-t^2-t$$  This does not equal $s(-t)$ so $s(t)$ is not odd.  Hence, $s(t)$ is neither even, nor odd.
		\end{freeResponse}
	\end{enumerate}
	
	
\end{problem}

\begin{instructorNotes}

\end{instructorNotes}


%problem 7
\begin{problem}
For any function $f$ defined on $(-\infty,\infty)$, can define $f_e$ and $f_o$ as follows: $f_e=\frac{f(x)+f(-x)}{2}$ and $f_o=\frac{f(x)-f(-x)}{2}$


\begin{enumerate}	
	\item  Show  $f_e=\frac{f(x)+f(-x)}{2}$  is even
		\begin{freeResponse}	
			 $$f_e=\frac{f(x)+f(-x)}{2}$$
			We want to show this is an even function, so we need to show that when we substitute in $-x$ for $x$ we get back $f_e$
			$$f_e(-x)=\frac{f(-x)+f(-(-x))}{2}$$ 
			$$=\frac{f(-x)+f(x)}{2}$$
			$$=f_e$$
		\end{freeResponse}

	\item  Show $f_o=\frac{f(x)-f(-x)}{2}$ is odd
		\begin{freeResponse}	
			 $$f_o=\frac{f(x)-f(-x)}{2}$$
			We want to show this is an odd function, so we need to show that when we substitute in $-x$ for $x$ we get back $-f_o$
			First, we'll find $-f_o$
			$$-f_o=-\frac{f(x)-f(-x)}{2}$$
			$$-f_o=\frac{-f(x)+f(-x)}{2}$$

			Now we'll find $f_o(-x)$
			$$\frac{f(-x)-f(-(-x))}{2}$$ 
			$$=\frac{f(-x)-f(x)}{2}$$
			$$=f_o$$
		\end{freeResponse}
	\item Show that $f(x)=f_e(x)+f_o(x)$, for all $x$
		\begin{freeResponse}
		
		We want to show that $f(x)=f_e(x)+f_o(x)$, for all $x$.  From parts a and b, we know $f_e=\frac{f(x)+f(-x)}{2}$ and $f_o=\frac{f(x)-f(-x)}{2}$.  Therefore,
			$$f(x)=f_e(x)+f_o(x)=\frac{f(x)+f(-x)}{2}+\frac{f(x)-f(-x)}{2}$$
			$$f(x)=\frac{f(x)+f(-x)+f(x)-f(-x)}{2}$$
			$$f(x)=\frac{2f(x)}{2}$$
			$$f(x)=f(x)$$
			$$\text{We have just proved that every function can be written as the sum of an odd function and an even function.}$$
	
		\end{freeResponse}	
	\end{enumerate}
	

\end{problem}

\begin{instructorNotes}

\end{instructorNotes}




\end{document} 


















