\documentclass[nooutcomes]{ximera}
%% handout
%% space
%% newpage
%% numbers
%% nooutcomes


\newcommand{\RR}{\mathbb R}
\renewcommand{\d}{\,d}
\newcommand{\dd}[2][]{\frac{d #1}{d #2}}
\renewcommand{\l}{\ell}
\newcommand{\ddx}{\frac{d}{dx}}
\newcommand{\dfn}{\textbf}
\newcommand{\eval}[1]{\bigg[ #1 \bigg]}

\usepackage{multicol}

\renewenvironment{freeResponse}{
\ifhandout\setbox0\vbox\bgroup\else
\begin{trivlist}\item[\hskip \labelsep\bfseries Solution:\hspace{2ex}]
\fi}
{\ifhandout\egroup\else
\end{trivlist}
\fi} %% we can turn off input when making a master document
\usepackage{fullpage}

\title{Recitation \#3 - 2.2:  Definition of Limits (Solutions)}  

\begin{document}
\begin{abstract}		\end{abstract}
\maketitle

%Problem `
\begin{problem} \hfil
	\begin{enumerate}

	\item  True or False: To find $\lim_{x \to 2} f(x)$, it's enough to know the values of $f(2.1)$, $f(2.01)$, $f(2.001)$, and so on.
	\begin{freeResponse}
	 False.  These values will only help us make a guess at $\lim_{x \to 2^+} f(x)$, the right hand limit of $f$ as $x$ approaches $2$.  To determine $\lim_{x \to 2} f(x)$, we also need to know $\lim_{x \to 2^-} f(x)$ which we cannot determine from the above values.  For example, consider the function
	 	        \[
          f(x) =
        \begin{cases}
         -1 & \mbox{if $x < 2$,}\\
          0 & \mbox{if $x = 2$, and}\\
          1 & \mbox{if $2 < x$.}
        \end{cases}
        \]

	Looking at the graph of this function below, we can see that $\lim_{x \to 2^+} f(x) = 1$ and $\lim_{x \to 2^-} f(x) = -1$.  Thus, since $\lim_{x \to 2^+} f(x) \neq \lim_{x \to 2^-} f(x)$, $\lim_{x \to 2} f(x)$ does not exist.
	
		\begin{image}
          \includegraphics[scale = 0.6]{"Graph of piecewise defined function".png}
		\end{image}
	\end{freeResponse}
	
	
	
	\item  True or False: If we know teh value of $f(2)$, then we know the value of $\lim_{x \to 2} f(x)$.
		\begin{freeResponse}
		False.  In the above example, we have that $f(2) = 0$, while $\lim_{x \to 2} f(x)$ does not exist.
		\end{freeResponse}
	\end{enumerate}


\end{problem}

%Problem 2
\begin{problem}
Use the graphs and the given definitions of the following two functiosn to answer the questions below.
	
	\begin{image}

    \includegraphics[scale = 0.7]{"Graph of rational function".png}
  \end{image}
\begin{image}
        \includegraphics[scale = 0.7]{"Graph of linear function".png}
  \end{image}

	\begin{enumerate}
	
	  \item Find the domain of $f$ and the domain of $g$.
      \begin{freeResponse}
        The domain of $f$ is $(-\infty, 1) \cup (1, \infty)$ (all real numbers except $1$).
        The domain of $g$ is $(-\infty, \infty)$ (all real numbers).
      \end{freeResponse}
	
  	\item Is $f = g$?
      (Why or why not?)
      \begin{freeResponse}
        No, these two functions are not equal.
        Two functions are equal if and only if they have identical domains and their values agree on points in the domain.

        Since $f$ and $g$ have different domains, by part (a), they cannot be equal functions.
	\end{freeResponse}
	
	 \item  Looking at the graphs, find $\lim_{x \to 1} f(x)$ and $\lim_{x \to 1} g(x)$.
      \begin{freeResponse}
        From the graphs we have that $\lim_{x \to 1} f(x) = 2$ and $\lim_{x \to 1} g(x) = 2$.
      \end{freeResponse}

	\end{enumerate}
\end{problem}
			
%Problem 3 
\begin{problem} This problem was on the AU15 Midterm. \hfil
	 The graph of a function $f$ is given below.  Use this graph to answer the following questions.
  \begin{image}
    \includegraphics[scale = 0.3]{"Piecewise defined function".png}
  \end{image}

 \begin{enumerate}
    \item
        Find the domain of $f$.
        \begin{freeResponse}
          The domain of $f$ is $(-\infty, -2) \cup (-2, \infty)$.
        \end{freeResponse}


    \item
        Find the range of $f$.
        \begin{freeResponse}
          The range of $f$ is $(-\infty, \infty)$.
        \end{freeResponse}

    \item
      Find the following values.
      \begin{itemize}
        \item
          $\displaystyle \lim_{x \to -2} f(x) = $
          \begin{freeResponse}
            $\displaystyle \lim_{x \to -2} f(x) = 0$
          \end{freeResponse}

        \item
          $f(-2) = $
          \begin{freeResponse}
            $f(-2)$ is undefined
          \end{freeResponse}

        \item
          $f(-5) = $
          \begin{freeResponse}
            $f(-5) = -2$
          \end{freeResponse}

        \item
          $\displaystyle \lim_{x \to 0^+} f(x) = $
          \begin{freeResponse}
            $\displaystyle \lim_{x \to 0^+} f(x) = 0$
          \end{freeResponse}

        \item
          $\displaystyle \lim_{x \to 0} f(x) = $
          \begin{freeResponse}
            $\displaystyle \lim_{x \to 0} f(x)$ is undefined
          \end{freeResponse}
      \end{itemize}
  \end{enumerate}

\end{problem}

					

%Problem 4
\begin{problem}
Sketch the graph of a function that satisfies all of the given properties.
  (You \emph{do not} need to find a formula for the function.)
	$$ f(3) = -2 , f(5) = 6 , \lim_{x \to 5^-} f(x) = -1 ,   \lim_{x \to 5^+} f(x) = 4 ,  \lim_{x \to 3} f(x) = 7 $$
	$$  \lim_{x \to -2^-} f(x) = 3 ,  \lim_{x \to -2^+} f(x) = 0 ,  \lim_{x \to 1^+} f(x) = 5  $$
	\begin{freeResponse} \hfil
	    \begin{image}
      \includegraphics[scale = 0.5]{"Construction step 1".png}
    \end{image}
    \begin{image}
      \includegraphics[scale = 0.4]{"Construction step 2".png}
    \end{image}
    \begin{image}
      \includegraphics[scale = 0.4]{"Construction step 3".png}
    \end{image}
    \begin{image}
      \includegraphics[scale = 0.4]{"Construction step 4".png}
    \end{image}
    \begin{image}
      \includegraphics[scale = 0.4]{"Construction step 5".png}
    \end{image}
	\end{freeResponse}
\end{problem}
	
	
	
	

%Problem 5
\begin{problem}
True/False:  Give an explanation or counterexample.  Assume $a$ and $L$ are finite numbers.
	
			\begin{enumerate}
			
			%part a
			\item  If $ \lim_{x \to a} f(x) = L$, then $f(a) = L$.
			\begin{freeResponse}
			False.  In the graph below $ \lim_{x \to 1} f(x) = 2 $, but $f(1)$ does not exist.
			
				\begin{image}
			
				\end{image}
			\end{freeResponse}
			
			
			
			%part b
			\item  If $  \lim_{x \to a^-} f(x) = L$, then $  \lim_{x \to a^+} f(x) = L $.
			\begin{freeResponse}
			False.  In the graph below $ \lim_{x \to 1^-} f(x) = 5$ but $ \lim_{x \to 1^+} f(x) = 6$.
			
				\begin{image}
				
				\end{image}
			\end{freeResponse}
			
			
			
			%part c
			\item  If $ \lim_{x \to a} f(x) = L $ and $  \lim_{x \to a} g(x) = L $, then $f(a) = g(a)$.
			\begin{freeResponse}
			 False.  If we let 
			 
			 	\begin{image}
			 
			 	\end{image}
			 	
			and
			
				\begin{image}
		
				\end{image}
				
			 we see that $ \lim_{x \to 1} f(x) = \lim_{x \to 1} g(x) = 2$, but $f(1) = 3$ and $g(1) = 1$.
			\end{freeResponse}
			
			
			
			%part d
			\item  $ \lim_{x \to a} \frac{f(x)}{g(x)} $ does not exist if $g(a) = 0$.
			\begin{freeResponse}
			False.  If $f(x) = x^3$ and $g(x) = x^2$, then $g(0) = 0$ but 
			$$ \lim_{x \to 0} \frac{f(x)}{g(x)} = \lim_{x \to 0} \frac{x^3}{x^2} = \lim_{x \to 0} x = 0.$$
			\end{freeResponse}
			
			
			
			\end{enumerate}
\end{problem}
	

\end{document} 


















