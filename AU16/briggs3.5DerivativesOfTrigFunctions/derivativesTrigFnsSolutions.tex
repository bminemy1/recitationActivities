\documentclass[nooutcomes]{ximera}
%% handout
%% space
%% newpage
%% numbers
%% nooutcomes


\newcommand{\RR}{\mathbb R}
\renewcommand{\d}{\,d}
\newcommand{\dd}[2][]{\frac{d #1}{d #2}}
\renewcommand{\l}{\ell}
\newcommand{\ddx}{\frac{d}{dx}}
\newcommand{\dfn}{\textbf}
\newcommand{\eval}[1]{\bigg[ #1 \bigg]}

\renewenvironment{freeResponse}{
\ifhandout\setbox0\vbox\bgroup\else
\begin{trivlist}\item[\hskip \labelsep\bfseries Solution:\hspace{2ex}]
\fi}
{\ifhandout\egroup\else
\end{trivlist}
\fi} %% we can turn off input when making a master document
\usepackage{fullpage}

\title{3.5 Derivatives of Trig Functions}  

\begin{document}
\begin{abstract}		\end{abstract}
\maketitle


%problem1
\begin{problem} \hfil
	\begin{enumerate}
	\item Suppose we're given the right triangle below.  Express $\sin(\theta)$ and $\cos(\theta)$ in terms of the sides of the triangle.

	\begin{image}
	\includegraphics[scale=.3]{figure1.png}
	\end{image}
	\begin{freeResponse}
	$\sin(\theta)=\frac{B}{C}=B$ and $\cos(\theta)=\frac{A}{C}=A$ 
	\end{freeResponse}

	\item Suppose we are given the triangle below.  
		\begin{image}
		\includegraphics[scale=.3]{figure2.png}
		\end{image}

		\begin{enumerate}
	\item Find the length of the sides A and B.
	\begin{freeResponse}
	This is an isosceles triangle. $A=B$  Using the Pythagorean Theorem:
	\begin{align*}
	A^2+B^2&=1^2\\
	2A^2&=1 \\ 	
	A^2&=\frac{1}{2}\\
	A&=\sqrt{\frac{1}{2}}=\frac{\sqrt{2}}{2}\\
	B&=\sqrt{\frac{1}{2}}=\frac{\sqrt{2}}{2}
	\end{align*}
	\end{freeResponse}

	\item Express $\sin\left(\frac{\pi}{4}\right)$ and $\cos\left(\frac{\pi}{4}\right)$ in terms of the sides of the triangle.

	\begin{freeResponse}
	$\sin\left(\frac{\pi}{4}\right)=\frac{B}{C}=\frac{\sqrt{2}}{2}$ and $\cos\left(\frac{\pi}{4}\right)=\frac{A}{C}=\frac{\sqrt{2}}{2}$

	\end{freeResponse}
	\end{enumerate}

	\item Suppose we are given the triangle below.  
		\begin{image}
		\includegraphics[scale=.5]{figure3.png}
		\end{image}

		\begin{enumerate}
	\item Find the length of the sides A and B.
	\begin{freeResponse}
	You might remember this as a 30/60/90 triangle.  To find the lengths of the sides of the triangle, we create a triangle as in the figure below.  All the angles in this triangle are of 60 degrees, therefore, this is an equilateral triangle 
		\begin{image}
		\includegraphics[scale=.5]{figure4.png}
		\end{image}
	Now that we have an equilateral triangle, we have $C=C=2A$.  Thus, $1=2A \implies A=\frac{1}{2}$ \\
	To find B, we use the Pythagorean Theorem.
	\begin{align*}
	\left(\frac{1}{2}\right)^2+B^2&=1^2\\
	\frac{1}{4}+B^2&=1 \\ 	
	B^2&=\frac{3}{4}\\
	B&=\sqrt{\frac{3}{4}}=\frac{\sqrt{3}}{2}
	\end{align*}
	\end{freeResponse}

	\item Express $\sin\left(\frac{\pi}{3}\right)$ and $\cos\left(\frac{\pi}{3}\right)$ in terms of the sides of the triangle.

	\begin{freeResponse}
	$\sin\left(\frac{\pi}{3}\right)=\frac{B}{C}=\frac{\sqrt{3}}{2}$ and $\cos\left(\frac{\pi}{3}\right)=\frac{A}{C}=\frac{1}{2}$

	\end{freeResponse}
	\end{enumerate}

\item Suppose we are given the triangle below.  
		\begin{image}
		\includegraphics[scale=.5]{figure5.png}
		\end{image}

		\begin{enumerate}
	\item Find the length of the sides A and B.
	\begin{freeResponse}
	We've actually already found the lengths of the sides for this type of triangle in part c.  $B=\sqrt{\frac{3}{4}}=\frac{\sqrt{3}}{2}$ and $A=\frac{1}{2}$ 
	\end{freeResponse}

	\item Write $\sin\left(\frac{\pi}{6}\right)$ and $\cos\left(\frac{\pi}{6}\right)$ in terms of the sides of the triangle.

	\begin{freeResponse}
	$\sin\left(\frac{\pi}{6}\right)=\frac{A}{C}=\frac{1}{2}$ and $\cos\left(\frac{\pi}{6}\right)=\frac{B}{C}=\frac{\sqrt{3}}{2}$

	\end{freeResponse}
	\end{enumerate}



	\item 
For any point P(x,y) on the unit circle, we can express its coordinates in terms of $\sin(\theta)$ and $\cos(\theta)$.
Here $\theta$ is the radian measure of the angle in standard position whose terminal side is the line through the origin and the point $P(x,y)$.
\begin{image}
		\includegraphics[scale=.8]{figure13.png}
		\end{image}
		\begin{freeResponse}
	$(x,y)=(\cos(\theta),\sin(\theta))$
	\end{freeResponse}

	\item  Use all of the above information to label the given points on the unit circle.  That is, for each point on the unit circle, provide the angle measure in radians and degrees, and give the (x,y) coordinate for the point.
		\begin{image}
		\includegraphics{figure6.png}
		\end{image}

		\begin{freeResponse} \hfil
		\begin{image}
		\includegraphics[scale=.7]{figure7.png}
		\end{image}
		\end{freeResponse}
	\end{enumerate}

\end{problem}


%problem2
\begin{problem}

  Find all real numbers which satisfy each of the equations.  In the previous problem, we used $\theta$ to denote the radian measure of the angle.  However, we can use any variable to represent the angle measure.  For example, in part a, $x$ is the variable representing the radian measure of the angle.
  \begin{enumerate}
    \item
      $\cos(x) = 1$
      \begin{freeResponse}
        This is asking for the collection of all angles such that cosine of that angle equals 1.

        The unit circle shows that one such angle is $0$ (since $\cos(0) = 1$).
        There is a slight trick here: since cosine has period $2\pi$ we actually have $\cos(0 + 2\pi n) = 1$ for every integer $n$.
        In summary, $x = 2\pi n$, where $n$ is any integer, gives all the solutions to this equation.
      \end{freeResponse}

    \item
      $\sin(3 \theta) = \sqrt{3}/2$ for $0 \leq \theta \leq 2\pi$
      \begin{freeResponse}
        Finding all numbers $\theta$ with $0 \leq \theta \leq 2\pi$ that satisfy $\sin(3 \theta) = \sqrt{3}/2$ is a bit tricky.
        We first perform a useful trick from algebra~---~variable substitution.

        Let $x = 3\theta$.
        So, we are trying to find all numbers $x$ such that $\sin(x) = \sqrt{3}/2$ for $0 \leq x/3 \leq 2\pi$.
        Then $ x= \frac{\pi}{3} + 2 \pi n$ or $ x = \frac{2 \pi }{3} + 2 \pi n $ for $n$ any integer as long as $0 \leq \theta \leq 2\pi$
        Since $x = 3 \theta$, we can solve for $\theta$ to obtain $\theta = \pi/9 + (2 \pi n)/3$ or $\theta = (2\pi)/9 + (2\pi n)/3$, where $n$ is again any integer as long as $0 \leq \theta \leq 2\pi$.
        We are only looking for solutions of $\theta$ in $[0, 2\pi ]$, and so our solutions are
        \[
        \theta = \frac{\pi}{9}, \frac{2\pi}{9}, \frac{7\pi}{9}, \frac{8\pi}{9}, \frac{13\pi}{9}, \frac{14\pi}{9}. 
        \]
      \end{freeResponse}
  \end{enumerate}
\end{problem}


%problem3
\begin{problem} \hfil

\begin{enumerate}
	\item Graph $f(\theta)=\sin(\theta)$ and $g(\theta)=\cos(\theta)$ from $[-\frac{\pi}{8},2\pi+\frac{\pi}{8}]$
		\begin{freeResponse} \hfil
		\begin{image}
		\includegraphics{figure8.png}
		\end{image}
		\begin{image}
		\includegraphics{figure9.png}
		\end{image}
		\end{freeResponse}

	\item For $0 \leq \theta \leq 2\pi$, find the following values or intervals.
		\begin{enumerate}
			\item Where does $f'(\theta)=0$?
		\begin{freeResponse}
			$f'(\theta)=0$ at $\theta=\frac{\pi}{2},\frac{3\pi}{2}$
		\end{freeResponse}
			\item Where is $f'$ negative?  Where is it positive?
		\begin{freeResponse}
			$f'$ is negative on $\left( \frac{\pi}{2},\frac{3 \pi}{2}\right)$.  $f'$ is positive on $ \left( 0,\frac{\pi}{2} \right)$ , $ \left( \frac{3 \pi}{2},2 \pi \right)$
		\end{freeResponse}
			\item Where is $f'$:
			\begin{enumerate}
			\item negative and increasing?
				\begin{freeResponse}
					$(\pi,\frac{3\pi}{2})$
				\end{freeResponse}
			\item negative and decreasing?
				\begin{freeResponse}
					$(\frac{\pi}{2},\pi)$
				\end{freeResponse}
			\item positive and increasing?
				\begin{freeResponse}
					$(\frac{3\pi}{2},2\pi)$
				\end{freeResponse}
			\item positive and decreasing?
				\begin{freeResponse}
					$(0,\frac{\pi}{2})$
				\end{freeResponse}
			\end{enumerate}
			\item Estimate the slope of $f$ at the points $x=0$ (i.e. $f'(0)$), $x=\pi$ (i.e. $f'(\pi)$), $x=2\pi$ (i.e. $f'(2\pi)$).
				\begin{freeResponse}
					The slope of $f$ at $x=0$ and $x=2\pi$ is 1.  	The slope of $f$ at $x=\pi$ is -1
				\end{freeResponse}
			\item Where does $g'(\theta)=0$?
		\begin{freeResponse}
			$g'(\theta=0$ at $\theta=0,\pi,2\pi$
		\end{freeResponse}
			\item Where is $g'$ negative?  Where is it positive?
		\begin{freeResponse}
			$g'$ is negative on $(0,\pi)$ and positive on $(\pi,2\pi)$
		\end{freeResponse}
			\item Where is $g'$:
			\begin{enumerate}
			\item negative and increasing?
				\begin{freeResponse}
					$(\frac{\pi}{2},\pi)$
				\end{freeResponse}
			\item negative and decreasing?
				\begin{freeResponse}
					$(0,\frac{\pi}{2})$
				\end{freeResponse}
			\item positive and increasing?
				\begin{freeResponse}
					$(\pi,\frac{3\pi}{2})$
				\end{freeResponse}
			\item positive and decreasing?
				\begin{freeResponse}
					$(\frac{3\pi}{2},2\pi)$
				\end{freeResponse}
			\end{enumerate}
			\item Estimate the slope of $g$ at the points $x=\frac{\pi}{2}$ (i.e. $g'(\frac{\pi}{2})$), $x=\frac{3\pi}{2}$ (i.e. $g'(\frac{3\pi}{2}i)$).
				\begin{freeResponse}
					The slope of $g$ at  $x=\frac{\pi}{2}$  is -1.  The slope of $g$ at $x=\frac{3\pi}{2}$ is 1.
				\end{freeResponse}
		\end{enumerate}			

		\item Use the above information to sketch a graph of $f'$ and $g'$
		\begin{freeResponse} \hfil
		\begin{image}
		\includegraphics{figure10.png}
		\end{image}
		\begin{image}
		\includegraphics{figure11.png}
		\end{image}
		\end{freeResponse}

	\item Based on the graphs of $f'$ and $g'$, what might the equation for $f'$ and $g'$ be?
		\begin{freeResponse}
			It appears $f'(\theta)=\cos(\theta)$ and $g'(\theta)=-\sin(\theta)$
		\end{freeResponse}


\end{enumerate}
\end{problem}


%problem 4
\begin{problem}
Find the following limits:

	\begin{enumerate}
	
	\item  $\lim_{x \to 0} \frac{\sin(8x)}{x}$
			\begin{freeResponse}
			\begin{align*}
			\lim_{x \to 0} \frac{\sin(8x)}{x} &= \lim_{x \to 0} \frac{\sin(8x)}{x} \cdot \frac{8}{8}  \\
			&= 8 \lim_{x \to 0} \frac{\sin(8x)}{8x}  \\
			&= 8 \lim_{u \to 0} \frac{\sin(u)}{u}  \\
			&= 8 \cdot 1 = 8
			\end{align*}
			where $u = 8x$.  
			\end{freeResponse}
			
	
	\item  $\lim_{x \to 0} \frac{x}{\tan(5x)}$
			\begin{freeResponse}
			\begin{align*}
			\lim_{x \to 0} \frac{x}{\tan(5x)} &= \lim_{x \to 0} \frac{x}{\frac{\sin(5x)}{\cos(5x)}}  \\
			&= \lim_{x \to 0} \left( \frac{x}{1} \cdot \frac{\cos(5x)}{\sin(5x)} \right)  \\
			&= \lim_{x \to 0} \left( \cos(5x) \frac{x}{\sin(5x)} \cdot \frac{5}{5} \right)  \\
			&= \frac{1}{5} \left( \lim_{x \to 0} \cos(5x) \right) \left( \lim_{x \to 0} \frac{5x}{\sin(5x)} \right)  \\
			&= \frac{1}{5} (1) (1) = \frac{1}{5}
			\end{align*}
			\end{freeResponse}
			
			
			
	\end{enumerate}
\end{problem}
	
%problem 5		
\begin{problem}
Find the derivative of the following functions:

	\begin{enumerate}
	
	%part a
	\item  $f(x) = \frac{x+5}{7x^6 + \cot(x)}$
			\begin{freeResponse}
			\begin{align*}
			f'(x) &= \frac{(7x^6 + \cot(x))(1) - (x+5)(42x^5 - \csc^2(x))}{(7x^6 + \cot(x))^2}  \\
			&= \frac{7x^6 + \cot(x) - (x+5)(42x^5 - \csc^2(x))}{(7x^6 + \cot(x))^2}.
			\end{align*}
			\end{freeResponse}
			
			
			
	%part b
	\item  $f(x) = \sin(x) \cos(x)$
			\begin{freeResponse}
			$$f'(x) = (\cos(x))(\cos(x)) + (\sin(x))(-\sin(x)) = \cos^2(x) - \sin^2(x).$$
			\end{freeResponse}
			
			
			
	%part c
	\item  $f(x) = \frac{e^x \tan(x)}{\sec(x) + 2}$
			\begin{freeResponse}
			\begin{align*}
			f'(x) &= \frac{(\sec(x)+2)(e^x \tan(x) + e^x \sec^2(x)) - e^x \tan(x) (\sec(x) \tan(x))}{(\sec(x) + 2)^2}  \\
			&= \frac{e^x[(\sec(x) + 2)(\tan(x) + \sec^2(x)) - \sec(x) \tan^2(x)]}{(\sec(x) + 2)^2}.
			\end{align*}
			\end{freeResponse}
			
			
			
	%part d
	\item  $f(x) = \sin(x) \cos(x) e^{3x}$
			\begin{freeResponse}
			\begin{align*}
			f'(x) &= \ddx[\sin(x) \cos(x)] e^{3x} + (\sin(x) \cos(x)) \ddx(e^{3x})  \\
			&= (\cos^2(x) - \sin^2(x))e^{3x} + 3e^{3x} \sin(x) \cos(x)  \\
			&= e^{3x}(\cos^2(x) + 3\sin(x) \cos(x) - \sin^2(x)).
			\end{align*}
			\end{freeResponse}
			
			
			
	\end{enumerate}
		
\end{problem}	

%problem 6
\begin{problem}
Let $f(x) = \sin(x)$ and $g(x) = \cos(x)$.  Can you compute $f^{(48)}(x)$, $g^{(42)}(x)$, and $g^{(39)}(x)$ ?  

\begin{freeResponse}
Let's find the first several derivatives of $f(x)=sin(x)$.

	\begin{align*}
	f^{(0)}(x)&=\sin(x)\\
	f^{(1)}(x)&=\cos(x)\\
	f^{(2)}(x)&=-\sin(x)\\
	f^{(3)}(x)&=-\cos(x)\\
	f^{(4)}(x)&=\sin(x)
	\end{align*}

  Recall that, for $n$ a nonnegative integer,
  $$f^{(4n)}(x) = f(x) = \sin(x) \quad \text{and} \quad g^{(4n)}(x) = g(x) = \cos(x).$$
  Thus: \\ 
  $f^{(48)}(x) =f(x)= \sin(x)$ \\
 $g^{(42)}(x) =g^{(40+2)}(x)=g^{(40)^{(2)}}(x)= g^{(2)}(x) = g''(x) = - \cos(x)$  \\
$g^{(39)}(x) =g^{(36+3)}(x)=g^{(36)^{(3)}}(x)= g^{(3)}(x) = g'''(x) = \sin(x)$  



\end{freeResponse}

\end{problem}
	


		
%problem7

\begin{problem} \hfil

\begin{enumerate}
	\item Let $f(x)= \frac{x^2-5x}{(x+5)\sin(x)}$.  Compute $f'(x)$

	\begin{freeResponse}	
	$f'(x)= \frac{(2x-5)(x+5)\sin(x)-(x^2-5x)(\sin(x)+(x+5)\cos(x))}{(x+5)^2 \cdot\sin^2(x)}$

	\end{freeResponse}

	\item Evaluate the limit. $\lim_{x \to 5} \frac{x^2-5x}{\sin(x-5)}$

	\begin{freeResponse}
	\begin{align*}
	\lim_{x \to 5} \frac{x^2-5x}{\sin(x-5)}=\lim_{x \to 5} \frac{x(x-5)}{\sin(x-5)}\\
	&=\lim_{x \to 5}(x) \cdot  \lim_{x \to 5}\frac{x-5}{\sin(x-5)}\\\\
	&=\lim_{x \to 5}(x) \cdot  \lim_{x \to 5}\frac{1}{\frac{\sin(x-5)}{x-5}}\\\\
	&=\lim_{x \to 5}(x) \cdot \frac{1}{ \lim_{x \to 5}\frac{\sin(x-5)}{x-5}}\\\\
	& \text{if we let}\ u=x-5\ \text{we have:}\\
	&=\lim_{x \to 5}(x) \cdot  \frac{1}{\lim_{u \to 0}\frac{\sin(u)}{u}}\\\\
	&= 5(1)=5
	\end{align*}
	\end{freeResponse}

\end{enumerate}
\end{problem}		
		
%problem8		
\begin{problem}
Find values for $a$, $b$, and $c$ so that the following function is differentiable everywhere.

$f(x) =   \left\{ \begin{array}{cl}
	a \sin(x) + b \cos(x)		 	&	\qquad \text{if } x < 0					\\
	ax^2 + bx + c   				&	\qquad \text{if } x \geq 0	 \end{array} \right.  $
		\begin{freeResponse}
		The function $f$ is differentiable for $x<0$ since $f$ is a combination of trigonometric functions.  $f$ is also differentiable $x>0$ since $f$ is a polynomial on that interval.  We need to focus on $x=0$.  Since $f$ is differentiable everywhere, $f$ must also be differentiable at $x=0$ and therefore continuous at $x=0$.  Therefore: \\\\
We need that $\lim_{x \to 0^-} f(x) = \lim_{x \to 0^+} f(x)=\lim_{x \to 0} f(x)=f(0)$.  Observe that
		
		\begin{itemize}
		
		\item $\lim_{x \to 0^-} f(x) 
		= \lim_{x \to 0^-} (a\sin(x) + b\cos(x))
		= b(1) = b$.
		
		\item  $ \lim_{x \to 0^+} f(x)
		= \lim_{x \to 0^+} (ax^2 + bx + c)
		= c$.
		
		\end{itemize}
		
		Thus, we must have that $b = c$\\\\
		Next, if $f'(0)$ exists, then:
		\begin{align*}
		f'(0)&=\lim_{x\to 0} \frac{f(x)-f(0)}{x-0}\\\\
		& \text{since}\ f(0)=a \cdot 0^2+b \cdot 0 +c =c\\\\
		& \lim_{x\to 0} \frac{f(x)-f(0)}{x-0}=\lim_{x\to 0} \frac{f(x)-c}{x}
		\end{align*}
		Since this limit exists, the left and right limits must be equal.
		
		\begin{align*}
		\lim_{x\to 0^+} \frac{f(x)-c}{x}&=\lim_{x\to 0^+} \frac{ax^2+bx+c-c}{x}\\
		&=\lim_{x\to 0^+} \frac{ax^2+bx}{x}\\
		&=\lim_{x\to 0^+} (ax+b)\\
		&=b
		\end{align*}
		\begin{align*}
		\lim_{x\to 0^-} \frac{f(x)-c}{x}&=\lim_{x\to 0^-} \frac{a \sin(x)+b \cos(x)-c}{x}\\
		&=\lim_{x\to 0^-} \frac{a \sin(x)}{x}+\lim_{x\to 0^-} \frac{b\cos(x)-c}{x}\\ \\
		& \text{we already found}\ b=c \ \text{so we have}\\
		&\lim_{x\to 0^-} \frac{a \sin(x)}{x}+\lim_{x\to 0^-} \frac{c\cdot \cos(x)-c}{x}\\
		&=a\cdot \lim_{x\to 0^-} \frac{ \sin(x)}{x}+c \cdot \lim_{x\to 0^-} \frac{\cos(x)-1}{x}\\ \\
		& \text{since}\ \lim_{x\to 0} \frac{\sin(x)}{x}=1\ \text{and}\  \lim_{x\to 0} \frac{\cos(x)-1}{x}=0 \ \text{we have}\\
		&a\cdot \lim_{x\to 0^-} \frac{\sin(x)}{x}+c \cdot \lim_{x\to 0^-} \frac{\cos(x)-1}{x}=a
		\end{align*}

		This means in order for the derivative to exist, $a=b=c$.
		\end{freeResponse}
\end{problem}	














\end{document} 


















