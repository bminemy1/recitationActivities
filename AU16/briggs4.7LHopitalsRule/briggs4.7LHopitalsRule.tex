\documentclass[nooutcomes]{ximera}

%% handout
%% space
%% newpage
%% numbers
%% nooutcomes

\newcommand{\RR}{\mathbb R}
\renewcommand{\d}{\,d}
\newcommand{\dd}[2][]{\frac{d #1}{d #2}}
\renewcommand{\l}{\ell}
\newcommand{\ddx}{\frac{d}{dx}}
\newcommand{\dfn}{\textbf}
\newcommand{\eval}[1]{\bigg[ #1 \bigg]}

\renewenvironment{freeResponse}{
\ifhandout\setbox0\vbox\bgroup\else
\begin{trivlist}\item[\hskip \labelsep\bfseries Solution:\hspace{2ex}]
\fi}
{\ifhandout\egroup\else
\end{trivlist}
\fi}
\usepackage{fullpage}



\title{4.7: L'Hopital's Rule}

\begin{document}
\begin{abstract}
\end{abstract}
\maketitle

%problem 1
\begin{problem}

Evaluate each limit.  Remember to state the form of the limit.
  \begin{enumerate}
  \item  $\lim_{x \to \infty}\left( \ln (1 + e^{-x}) \right)^x $
    \begin{freeResponse}
      Since $\lim_{x \to \infty} (1 + e^{-x}) = 1 + 0 = 1$ and $\ln (1) = 0$, this limit is of the form $0^{\infty}$.
      This is a determinate form which converges to $0$.
      Thus, $\lim_{x \to \infty} \left( \ln (1 + e^{-x}) \right)^x = 0$.
    \end{freeResponse}
    
  \item  $\lim_{x \to \infty} \left( \frac{1}{x} + 1 \right)^{\frac{1}{x}} $
    \begin{freeResponse}
      This limit is of the form $1^0$, which is a determinate form that converges to 1.
      Thus, $\lim_{x \to \infty} \left( \frac{1}{x} + 1 \right)^{\frac{1}{x}} = 1 $
    \end{freeResponse}
    
  \item  $\lim_{x \to \infty} \left( \frac{\arctan x}{x} \right) $
    \begin{freeResponse}
      Since $\lim_{x \to \infty} \arctan x = \frac{\pi}{2}$, this limit is of the form $\frac{\text{\#}}{\infty} \to 0$.
      Thus, $\lim_{x \to \infty} \left( \frac{\arctan x}{x} \right) = 0$  
    \end{freeResponse}
    
  \item  $\lim_{x \to \infty} (x-ln(x)) $
    \begin{freeResponse}
      This limit is of the form $\infty - \infty$, which is an indeterminate form.  We can rewrite this as:
   
      $$\lim_{x \to \infty} (x-ln(x))= \lim_{x \to \infty} \left(x\left(1-\frac{ln(x)}{x}\right)\right)$$
     
     We can see that $\lim_{x \to \infty}\left(\frac{ln(x)}{x}\right)$ is of the form $\frac{\infty}{\infty}$ so we can use L'Hopital's Rule.
     
      $$\stackrel{L.R.}{\implies} \lim_{x \to \infty}\left(\frac{1/x}{1}\right)=\lim_{x \to \infty}\left(\frac{1}{x}\right)=0$$
      
      Now we have: 
      $$\lim_{x \to \infty} \left(x\left(1-\frac{ln(x)}{x}\right)\right)$$
      This limit has the form $\infty \cdot 1$.  This is a determinate form, and, therefore,

	 $\lim_{x \to \infty} (x-ln(x)) =\infty$
   
    \end{freeResponse}

  \item  $\lim_{x \to \infty} \left( x \ln \left( \frac{1}{x} \right) \right) $
    \begin{freeResponse}
      As $x$ approaches $\infty$, $\frac{1}{x}$ approaches $0$ from the right.
      So  $$\lim_{x \to \infty} \ln \left( \frac{1}{x} \right) = - \infty $$
      Therefore, the limit in question is of the form $\infty \cdot - \infty$, which converges to $- \infty$.
      Thus, 
      $$ \lim_{x \to \infty} \left( x \ln \left( \frac{1}{x} \right) \right) = - \infty $$
    \end{freeResponse}

  \item  $\lim_{x \to 0^+} (\sin x \cot x ) $
    \begin{freeResponse}
      Since $\lim_{x \to 0^+} \cot x = \infty$, this limit is of the form $0 \cdot \infty$.
      This is an indeterminate form.
      
     Note: $\cot x = \frac{\cos x }{\sin x}$.  So
      $$\lim_{x \to 0^+} (\sin x \cot x ) = \lim_{x \to 0^+} \cos x = 1 .$$
    \end{freeResponse}
  \end{enumerate}
\end{problem}


%problem 2
\begin{problem}

  True or False: You can use L'H\^{o}pital's Rule to compute
  $\lim_{x \to 0} \frac{|x|}{x}$.
  \begin{freeResponse}
    False.
    The function $|x|$ is not differentiable at $x=0$, and so L'Hospital's Rule is not applicable.
  \end{freeResponse}
\end{problem}

%problem 3
\begin{problem}

  Circle the correct answer in each part:
  \begin{enumerate}
    \item
      Consider the limit $\lim_{x \to 0} (\cos x)^{\sin x}$.
      \begin{enumerate}
        \item
          Evaluate the limit.
          \begin{freeResponse}
            \textbf{The correct choice is (iii).}

            Evaluation of limit:
            \begin{align*}
              \lim_{x \to 0} \underbrace{(\cos x)^{\sin x}}_{\text{form $1^0$}} &= 1
            \end{align*}
          \end{freeResponse}
          \begin{enumerate}
            \item
              the limit DNE
            \item
              $e$
             \item
              $1$
            \item
              $\infty$
            \item
              $-\infty$
            \item
              $0$
            \item
              none of the previous answers is correct
          \end{enumerate}
        \item
          What Limit Law, rule or technique did you use to find this limit?
          \begin{freeResponse}
            The correct choice is (iv).
          \end{freeResponse}
          \begin{enumerate}
            \item
              The Squeeze Theorem;
            \item
              L'H\^{o}pital's Rule;
            \item
              The Product Law;
            \item
              evaluated the function at $x = 0$, since the function is continuous at $x = 0$;
            \item
              none of the previous answers is correct
          \end{enumerate}
      \end{enumerate}

    \item
      Evaluate the limit $\lim_{x \to 4^-} \frac{\ln x}{x - 4}$.
      \begin{freeResponse}
        The correct choice is (v).

        Evaluation of limit:
        \begin{align*}
          \lim_{x \to 4^-} \underbrace{\frac{\ln x}{x - 4}}_{\text{form $\ln(4)/0^-$}} &= - \infty
        \end{align*}
      \end{freeResponse}

      \begin{enumerate}
        \item
          the limit DNE
        \item
          $e$
        \item
          $1$
        \item
          $\infty$
        \item
          $-\infty$
        \item
          $0$
        \item
          none of the previous answers is correct
      \end{enumerate}

    \item
      Evaluate the limit $\lim_{x \to \infty} \frac{\ln x}{x - 4}$.
      \begin{freeResponse}
        The correct choice is (vi).

        Evaluation of limit:
        \begin{align*}
          \lim_{x \to \infty} \underbrace{\frac{\ln x}{x - 4}}_{\text{form $\infty/\infty$}} &\stackrel{L.H.}{=} \lim_{x \to \infty} \frac{1/x}{1} \\
          &= 0
        \end{align*}

      \end{freeResponse}

      \begin{enumerate}
        \item
          the limit DNE
        \item
          $e$
        \item
          $1$
        \item
          $\infty$
        \item
          $-\infty$
        \item
          $0$
        \item
          none of the previous answers is correct
      \end{enumerate}

    \item
      Consider the limit $\lim_{h \to 0} \frac{(2+h)^3 - 8}{h} = f'(2)$.
      Determine the function $f$.
      \begin{freeResponse}
        The correct choice is (ii).
      \end{freeResponse}
      \begin{enumerate}
       \item
         such a function DNE;
       \item
         $f(x) = x^3$;
       \item
         $f(x) = (2 + x)^3$;
       \item
         $f(x) = \frac{(2+x)^3}{x}$;
       \item
         none of the previous answers is correct
      \end{enumerate}
      
    \item
      Consider the limit $\lim_{x \to 0^+} \left( \frac{\sin x}{x} \right)^{|\ln x|}$.
      Determine the form of this limit.
      \begin{freeResponse}
        The correct choice is (v).
      \end{freeResponse}
      \begin{enumerate}
       \item
         $\frac{0}{0}$;
       \item
         $\frac{\infty}{\infty}$;
       \item
         $1^0$;
       \item
         $0^0$;
       \item
         $1^\infty$;
       \item
         $\infty^\infty$;
       \item
         none of the previous answers is correct
      \end{enumerate}
  \end{enumerate}
\end{problem}



%problem 4
\begin{problem}
  Determine the following limits.
  Use L'Hospital's Rule if applicable.
  \begin{enumerate}
  
    %part a  
  \item  $\lim_{x \to \infty} \frac{x}{\sqrt{x^2 + 1}}  $

    \begin{freeResponse}
    
    This limit is of the form:  $\frac{\infty}{\infty}$
      \begin{align*}
        \lim_{x \to \infty} \frac{x}{\sqrt{x^2 + 1}} &= \lim_{x \to \infty} \frac{x}{\sqrt{x^2 \left(1 + \frac{1}{x^2} \right)}} \\
                                                     &=  \lim_{x \to \infty} \frac{x}{|x| \sqrt{1 + \frac{1}{x^2} }} \\
                                                     &=  \lim_{x \to \infty} \frac{x}{x \sqrt{1 + \frac{1}{x^2} }} \\
                                                     &=  \lim_{x \to \infty} \frac{1}{\sqrt{1 + \frac{1}{x^2} }} \\
                                                     &= \frac{1}{\sqrt{1 + 0}} = 1
      \end{align*}
    \end{freeResponse}
    
    
    %part b
  \item  $\lim_{x \to - \infty} x^2 e^x $
    \begin{freeResponse}
    This limit is of the form: $\infty \cdot 0$
      \begin{align*}
        \lim_{x \to - \infty} x^2 e^x &= \lim_{x \to - \infty} \frac{x^2}{ e^{-x}} \; \left( \text{of the form } \frac{\infty}{\infty} \right) \\
                                      &\stackrel{L.R.}{=}  \lim_{x \to - \infty} \frac{2x}{- e^{-x}} \; \left( \text{of the form } \frac{\infty}{\infty} \right) \\
                                      &\stackrel{L.R.}{=}  \lim_{x \to - \infty} \frac{2}{ e^{-x}}  \\
                                      &= 0
      \end{align*}
      where ``L.R.'' above an equals sign means that that equality is due to ``L'Hospital's Rule''.  
    \end{freeResponse}
    
    
    
    % part c
  \item  $\lim_{x \to \infty} x^{\frac{1}{x}} $
    \begin{freeResponse}
    This limit is of the form: $\infty^0$
      \begin{align*}
        \lim_{x \to \infty} x^{\frac{1}{x}} &= \lim_{x \to \infty} e^{\ln \left( x^{\frac{1}{x}} \right) } \\
                                            &= \lim_{x \to \infty} e^{\frac{1}{x} \ln x } \\
                                            &= e^{ \lim_{x \to \infty} \frac{\ln x}{x} } \; \left( \text{limit is of the form } \frac{\infty}{\infty} \right) \\
                                            &\stackrel{L.R.}{=} e^{\lim_{x \to \infty}\frac{\frac{1}{x}}{1}} \\
                                            &= e^{\lim_{x \to \infty} \frac{1}{x}} \\
                                            &= e^0 = 1
      \end{align*}
    \end{freeResponse}
    
    %part d
    \item $\lim_{x \to \infty} \left(1+\frac{2}{x}\right)^x$
        \begin{freeResponse}
    This limit is of the form: $1^\infty$
     \begin{align*}
        \lim_{x \to \infty} \left(1+\frac{2}{x}\right)^x &= \lim_{x \to \infty} e^{\ln \left(\left(1+\frac{2}{x}\right)^x \right) } \\
                                            &= \lim_{x \to \infty} e^{x \ln \left(1+\frac{2}{x}\right) } \\
                                            &= e^{ \lim_{x \to \infty} \frac{\ln \left(1+\frac{2}{x}\right)}{1/x} } \; \left( \text{limit is of the form } \frac{0}{0} \right) \\
                                            &\stackrel{L.R.}{=} e^{\lim_{x \to \infty}\frac{-2x^{-2}\cdot \frac{1}{(1+2/x)}}{-x^{-2}}} \\
                                            &= e^{\lim_{x \to \infty} \left(2\left(\frac{1}{1+\frac{2}{x}}\right)\right)} \\
                                            &= e^2 
        \end{align*}
    \end{freeResponse}
    
    
        %part e
    \item $\lim_{x \to 0^+} (\sin\theta)^{\tan\theta}$
        \begin{freeResponse}
    This limit is of the form: $0^0$
    
     \begin{align*}
        \lim_{x \to 0^+} (\sin\theta)^{\tan\theta} &= \lim_{x \to 0^+} e^{\ln \left((\sin\theta)^{\tan\theta} \right) } \\
                                            &= \lim_{x \to 0^+} e^{\tan\theta \ln (\sin\theta) } \\
                                            &= e^{ \lim_{x \to 0^+} \frac{\ln (\sin\theta)}{\cot\theta} } \; \left( \text{limit is of the form } \frac{\infty}{\infty} \right) \\
                                            &\stackrel{L.R.}{=} e^{\lim_{x \to 0^+}\frac{\cos\theta \cdot \frac{1}{\sin\theta}}{-\csc^2\theta}} \\
                                            &= e^{\lim_{x \to 0^+} \left(\frac{\cos\theta}{\sin\theta}\cdot\frac{\sin^2\theta}{1}\right)} \\
                                              &= e^{\lim_{x \to 0^+} (\cos\theta\cdot\sin\theta)} \\
                                            &= e^0=1
        \end{align*}
    \end{freeResponse}
    

    
  \end{enumerate}
\end{problem}

%problem5
\begin{problem}
Use limits to compare growth rates.  State which function grows faster or that they have comparable growth rates.

\begin{enumerate}
	\item$b^x; x^x; b>1$
		\begin{freeResponse}
		$\lim_{x \to \infty}\frac{b^x}{(x^x}=\lim_{x \to \infty}\left(\frac{b}{x}\right)^x=0$.  Since this limit is of the form $0^{\infty}$ which is a determinate form and equals $0$.
		Therefore, $x^x$ grows faster.
		
		\end{freeResponse}
	
	\item $x^x; \left(\frac{x}{e}\right)^x$
		\begin{freeResponse}
		$\lim_{x \to \infty}\frac{x^x}{(x/e)^x}=\lim_{x \to \infty}\frac{x^x}{\frac{x^x}{e^x}}=\lim_{x \to \infty}{x^x}\cdot{\frac{e^x}{x^x}}=\lim_{x \to \infty}{e^x}=\infty$  Therefore, $x^x$ grows faster.
		
		\end{freeResponse}
	

	
	\item $x^3; x^3 \cdot \ln(x)$
		\begin{freeResponse}
		$\lim_{x \to \infty}\frac{x^3}{x^3 \cdot \ln(x)}=\lim_{x \to \infty}\frac{1}{\ln(x)}=0$.  
		Therefore, $x^3 \cdot \ln(x)$ grows faster.
		
		\end{freeResponse}
	\item $a^x; b^x; 0<a<b$
	\begin{freeResponse}
		$\lim_{x \to \infty}\frac{a^x}{b^x}=\lim_{x \to \infty}\left(\frac{a}{b}\right)^x=0$
		Therefore, $b^x$ grows faster
	
	
	\end{freeResponse}
	\item $\log_a(x); \log_b(x); 1<a<b$
	\begin{freeResponse}
	$\lim_{x \to \infty}\frac{\log_a(x)}{\log_b(x)}$\\
	By L'Hopital's Rule: $\lim_{x \to \infty}\frac{\frac{1}{x\ln(a)}}{\frac{1}{x\ln(b)}}=\lim_{x \to \infty}\left(\frac{1}{x\ln(a)} \cdot \frac{x\ln(b)}{1}\right)=\frac{\ln b}{\ln a}$\\
	Therefore, $\log_a(x)$ and $\log_b(x)$ grow at comparable rates.
	
		\end{freeResponse}
	\item $\ln^{3} (x); x^{1/2}$
		\begin{freeResponse}
	$\lim_{x \to \infty}\frac{\ln^{3}(x)}{x^{1/2}}$\\
	By L'Hopital's Rule:$\lim_{x \to \infty}\frac{\ln^{3}(x)}{x^{1/2}}=\lim_{x \to \infty}\frac{3\ln^2 (x) \cdot (1/x)}{(1/2)x^{-1/2}}=\lim_{x \to \infty}\frac{6 \cdot \ln^2 (x)}{x^{1/2}}$\\
	By L'Hopital's Rule:$\lim_{x \to \infty}\frac{12\ln (x) \cdot (1/x)}{(1/2)x^{-1/2}}=\lim_{x \to \infty}\frac{24 \cdot \ln (x)}{x^{1/2}}$\\
	By L'Hopital's Rule: $\lim_{x \to \infty}\frac{48}{x^{1/2}}=0$
	Therefore, $x^{1/2}$ grows faster
		\end{freeResponse}
	
		%part g	
	\item $x; \ln(x)\sqrt{x}$
		\begin{freeResponse}
	
	$\lim_{x \to \infty}\frac{x}{\ln (x)\sqrt{x}}=\lim_{x \to \infty}\frac{x^{1/2}}{\ln x}$\\
	By L'Hopital's Rule:$\lim_{x \to \infty}\frac{(1/2)x^{-1/2}}{x^{-1}}=\lim_{x \to \infty}{(1/2)x^{1/2}}=\infty$\\
		Therefore, $x$ grows faster
		\end{freeResponse}
		
		%part h
		\item Challenge: $x^{40}; 1.004^x$ (Hint: Use the substitution $x=lnt$.)
			\begin{freeResponse}
			\begin{align*}
			\lim_{x \to \infty}\frac{x^{40}}{1.004^x} &=\lim_{t \to \infty}\frac{\ln (t)^{40}}{1.004^{\ln(t)}}\\
			&=\lim_{t \to \infty}\frac{\ln (t)^{40}}{e^{\ln(1.004)^{\ln t}}}\\
			&=\lim_{t \to \infty}\frac{\ln (t)^{40}}{t^{\ln(1.004)}}\\
			&=\lim_{t \to \infty}\frac{40\ln(t)}{1.004t}\\
			&\stackrel{L.R.}{=} \lim_{t \to \infty}\frac{\frac{40}{t}}{1.004}\\
			&=0	
				\end{align*}
		Therefore, $1.004^x$ grows faster.
		\end{freeResponse}
\end{enumerate}
\end{problem}
\end{document} 
